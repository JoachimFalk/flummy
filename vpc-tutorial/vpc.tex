%%%%%%%%%%%%%%%%%%%%%%%%%%%%%%%%%%%%%%%%%%%%%%%%%%%%%%%%%%%%%%%%%%%%%%%%%%%%%%
%%
%%%%%%%%%%%%%%%%%%%%%%%%%%%%%%%%%%%%%%%%%%%%%%%%%%%%%%%%%%%%%%%%%%%%%%%%%%%%%%
\subsection{Functional Model}

%%%%%%%%%%%%%%%%%%%%%%%%%%%%%%%%%%%%%%%%%%%%%%%%%%%%%%%%%%%%%%%%%%%%%%%%%%%%%%
%%%%%%%%%%%%%%%%%%%%%%%%%%%%%%%%%%%%%%%%%%%%%%%%%%%%%%%%%%%%%%%%%%%%%%%%%%%%%%
\begin{frame}[t]
\mode<presentation>{\frametitle{\insertsubsection\ -- SysteMoC Recap}}
\begin{figure}
\centering
\resizebox{0.7\columnwidth}{!}{\input{vpc-systemoc-fig.tex}}
\end{figure}

\begin{itemize}
\item Write a functional model in SysteMoC
\item Two actors, Source and Sink, are connected via a FIFO queue
\item cf. ``SystemCoDesigner: System Programming Tutorial''
\end{itemize}

\end{frame}


%%%%%%%%%%%%%%%%%%%%%%%%%%%%%%%%%%%%%%%%%%%%%%%%%%%%%%%%%%%%%%%%%%%%%%%%%%%%%%
%%%%%%%%%%%%%%%%%%%%%%%%%%%%%%%%%%%%%%%%%%%%%%%%%%%%%%%%%%%%%%%%%%%%%%%%%%%%%%
 % fragile is mandatory for verbatim environments (lstlisting)
 % HACK: use singleslide for correct page numbering
 %  (page_counter+=2 else wise)
 %  this issue seems to be relevant for *pdflatex* only
 % NOTE: the intended usage for *singleslide* is to tell beamer that we use
 %   no overlays (animation)
\begin{frame}[fragile=singleslide]
\mode<presentation>{\frametitle{\insertsubsection}}
\begin{lstlisting}
//file vpc_source_sink.cpp
#include <iostream>
#include <systemoc/smoc_moc.hpp>
static const std::string MESSAGE = "Hello SysteMoC!";

class Source : public smoc_actor{
public:
  smoc_port_out<char> out;

  Source(sc_module_name name) : smoc_actor(name, start),
    count(0), size(MESSAGE.size()), message(MESSAGE) {
    start = GUARD(Source::hasToken) >>
      out(1)                        >>
      CALL(Source::src)             >> start;
  }
\end{lstlisting}
\begin{itemize}
\item the Source actor has an output port with data type \lstinline!char!
\item a single FSM transition is used produce one data token per actor firing
\end{itemize}
\end{frame}


%%%%%%%%%%%%%%%%%%%%%%%%%%%%%%%%%%%%%%%%%%%%%%%%%%%%%%%%%%%%%%%%%%%%%%%%%%%%%%
%%%%%%%%%%%%%%%%%%%%%%%%%%%%%%%%%%%%%%%%%%%%%%%%%%%%%%%%%%%%%%%%%%%%%%%%%%%%%%
\begin{frame}[fragile=singleslide]
\mode<presentation>{\frametitle{\insertsubsection}}
\begin{lstlisting}
//file vpc_source_sink.cpp cont'd
private:
  smoc_firing_state start;   // states
  unsigned int count, size;  // variables (functional state)
  const std::string message; //

  bool hasToken() const {
    return count < size;
  } // guard

  void src() {
    std::cerr << this->name() << "> @ " << sc_time_stamp()
        << "\tsend: \'" << message[count] << "\'" << std::endl;

    out[0] = message[count++];
  } // action
};
\end{lstlisting}
\begin{itemize}
\item a function \lstinline!src()! is used to produce the data tokens
\item a guard \lstinline!hasToken()! is used to check if the transition can be fired
\end{itemize}
\end{frame}


%%%%%%%%%%%%%%%%%%%%%%%%%%%%%%%%%%%%%%%%%%%%%%%%%%%%%%%%%%%%%%%%%%%%%%%%%%%%%%
%%%%%%%%%%%%%%%%%%%%%%%%%%%%%%%%%%%%%%%%%%%%%%%%%%%%%%%%%%%%%%%%%%%%%%%%%%%%%%
\begin{frame}[fragile=singleslide]
\mode<presentation>{\frametitle{\insertsubsection}}
\begin{lstlisting}
//file vpc_source_sink.cpp cont'd
class Sink : public smoc_actor {
public:
  smoc_port_in<char> in;

  Sink(sc_module_name name) // actor constructor
  : smoc_actor(name, start) {
    start = in(1) >> CALL(Sink::sink) >> start;
  }
private:
  smoc_firing_state start;

  void sink() {
    std::cout << this->name() << "> @ " << sc_time_stamp()
        << "\trecv: \'" << in[0] << "\'" << std::endl;
  }
};
\end{lstlisting}
\begin{itemize}
\item the Sink actor receives and consume the data token
\end{itemize}
\end{frame}


%%%%%%%%%%%%%%%%%%%%%%%%%%%%%%%%%%%%%%%%%%%%%%%%%%%%%%%%%%%%%%%%%%%%%%%%%%%%%%
%%%%%%%%%%%%%%%%%%%%%%%%%%%%%%%%%%%%%%%%%%%%%%%%%%%%%%%%%%%%%%%%%%%%%%%%%%%%%%
\begin{frame}[fragile=singleslide]
\mode<presentation>{\frametitle{\insertsubsection}}
\begin{lstlisting}
//file vpc_source_sink.cpp cont'd
class NetworkGraph : public smoc_graph {
public:
  NetworkGraph(sc_module_name name)
   : smoc_graph(name), source("Source"), sink("Sink") {
    smoc_fifo<char> fifo("queue", 4);
    fifo.connect(source.out).connect(sink.in); // connect actors
  }
private:
  Source source; // actors
  Sink sink;
};
int sc_main(int argc, char **argv){
  NetworkGraph top("top"); // create network graph
  smoc_scheduler_top sched(top);
  sc_start(); // start simulation (SystemC)
  return 0;
}
\end{lstlisting}
\begin{itemize}
\item Source and Sink are connected via a queue (queue size = 4)
\end{itemize}
\end{frame}


%%%%%%%%%%%%%%%%%%%%%%%%%%%%%%%%%%%%%%%%%%%%%%%%%%%%%%%%%%%%%%%%%%%%%%%%%%%%%%
%%%%%%%%%%%%%%%%%%%%%%%%%%%%%%%%%%%%%%%%%%%%%%%%%%%%%%%%%%%%%%%%%%%%%%%%%%%%%%
\begin{frame}[fragile=singleslide]
\mode<presentation>{\frametitle{\insertsubsection}}
\begin{lstlisting}
> ./vpc-src-sink

             SystemC 2.2.0 --- Dec 15 2008 13:19:05
        Copyright (c) 1996-2006 by all Contributors
                    ALL RIGHTS RESERVED            
top.Source> @ 0 s       send: 'H'
top.Sink> @ 0 s recv: 'H'
top.Source> @ 0 s       send: 'e'
top.Sink> @ 0 s recv: 'e'
top.Source> @ 0 s       send: 'l'
top.Sink> @ 0 s recv: 'l'
top.Source> @ 0 s       send: 'l'
top.Sink> @ 0 s recv: 'l'
top.Source> @ 0 s       send: 'o'
top.Sink> @ 0 s recv: 'o'
...
latency: 0 s
\end{lstlisting}
\begin{itemize}
\item Compile source code and run functional simulation
\item No timing is simulated in the functional model
\end{itemize}
\end{frame}


%%%%%%%%%%%%%%%%%%%%%%%%%%%%%%%%%%%%%%%%%%%%%%%%%%%%%%%%%%%%%%%%%%%%%%%%%%%%%%
%%
%%%%%%%%%%%%%%%%%%%%%%%%%%%%%%%%%%%%%%%%%%%%%%%%%%%%%%%%%%%%%%%%%%%%%%%%%%%%%%
\subsection{Architecture Model}


%%%%%%%%%%%%%%%%%%%%%%%%%%%%%%%%%%%%%%%%%%%%%%%%%%%%%%%%%%%%%%%%%%%%%%%%%%%%%%
%%%%%%%%%%%%%%%%%%%%%%%%%%%%%%%%%%%%%%%%%%%%%%%%%%%%%%%%%%%%%%%%%%%%%%%%%%%%%%
\begin{frame}[fragile=singleslide]
\mode<presentation>{\frametitle{\insertsubsection}}
\begin{itemize}
\item Write VPC configuaration (XML file \lstinline!src-snk.vpc.xml!)
\item \lstinline!.vpc.xml! is the preferred file name suffix
\item the document type definition is given in the file (\lstinline!vpc.dtd!)
\end{itemize}

\begin{lstlisting}
<?xml version="1.0"?>
<!DOCTYPE vpcconfiguration SYSTEM "vpc.dtd">
<vpcconfiguration>
 <resources>
  ...
 </resources>

 <mappings>
  ...
 </mappings>

 <topology>
  ...
 </topology>
</vpcconfiguration>
\end{lstlisting}
\end{frame}


%%%%%%%%%%%%%%%%%%%%%%%%%%%%%%%%%%%%%%%%%%%%%%%%%%%%%%%%%%%%%%%%%%%%%%%%%%%%%%
%%%%%%%%%%%%%%%%%%%%%%%%%%%%%%%%%%%%%%%%%%%%%%%%%%%%%%%%%%%%%%%%%%%%%%%%%%%%%%
\begin{frame}[fragile=singleslide]
\mode<presentation>{\frametitle{\insertsubsection}}
\begin{itemize}
\item \lstinline!<vpcconfiguration>! the XML root element of a VPC configuration
\item \lstinline!<vpcconfiguration>! cantains the nested elements
  \begin{itemize}
  \item \lstinline!<resources>!, \lstinline!<mappings>!, and \lstinline!<topology>!
  \end{itemize}
\end{itemize}

\begin{lstlisting}
<?xml version="1.0"?>
<!DOCTYPE vpcconfiguration SYSTEM "vpc.dtd">
<vpcconfiguration>
 <resources>
  ...
 </resources>

 <mappings>
  ...
 </mappings>

 <topology>
  ...
 </topology>
</vpcconfiguration>
\end{lstlisting}
\end{frame}


%%%%%%%%%%%%%%%%%%%%%%%%%%%%%%%%%%%%%%%%%%%%%%%%%%%%%%%%%%%%%%%%%%%%%%%%%%%%%%
%%%%%%%%%%%%%%%%%%%%%%%%%%%%%%%%%%%%%%%%%%%%%%%%%%%%%%%%%%%%%%%%%%%%%%%%%%%%%%
\begin{frame}[fragile=singleslide]
\mode<presentation>{\frametitle{\insertsubsection}}
\begin{itemize}
\item the \lstinline!<resources>! element has nested \lstinline!<component>! elements
\item an element \lstinline!<component>! defines a (Virtual Processing) Component
\item each Component has a name and a scheduler given by XML attributes (\lstinline!name! and \lstinline!scheduler!)
\end{itemize}

\begin{lstlisting}
 <resources>
  <component name="CPU" scheduler="FCFS">
  </component>

  <component name="Bus" scheduler="FCFS">
   <attribute type="transfer_delay" value="80 ns" />
  </component>

  <component name="Mem" scheduler="FCFS">
   <attribute type="transfer_delay" value="20 ns" />
  </component>
 </resources>
\end{lstlisting}
\end{frame}


%%%%%%%%%%%%%%%%%%%%%%%%%%%%%%%%%%%%%%%%%%%%%%%%%%%%%%%%%%%%%%%%%%%%%%%%%%%%%%
%%%%%%%%%%%%%%%%%%%%%%%%%%%%%%%%%%%%%%%%%%%%%%%%%%%%%%%%%%%%%%%%%%%%%%%%%%%%%%
\begin{frame}[fragile=singleslide]
\mode<presentation>{\frametitle{\insertsubsection -- Attributes}}
\begin{itemize}
\item components may have attributes (nested \lstinline!<attribute>! elements)
\item an attributtes has a type and a value (\lstinline!<attribute type="..." value="..." />!)
\end{itemize}

\begin{lstlisting}
 <resources>
  <component name="CPU" scheduler="FCFS">
  </component>

  <component name="Bus" scheduler="FCFS">
   <attribute type="transfer_delay" value="80 ns" />
  </component>

  <component name="Mem" scheduler="FCFS">
   <attribute type="transfer_delay" value="20 ns" />
  </component>
 </resources>
\end{lstlisting}
\end{frame}


%%%%%%%%%%%%%%%%%%%%%%%%%%%%%%%%%%%%%%%%%%%%%%%%%%%%%%%%%%%%%%%%%%%%%%%%%%%%%%
%%%%%%%%%%%%%%%%%%%%%%%%%%%%%%%%%%%%%%%%%%%%%%%%%%%%%%%%%%%%%%%%%%%%%%%%%%%%%%
\begin{frame}[fragile=singleslide]
\mode<presentation>{\frametitle{\insertsubsection -- TODO}}
\begin{itemize}
\item Write VPC configuaration as XML file ``src-snk.cmx.xml''
\end{itemize}

\begin{lstlisting}
<?xml version="1.0"?>
<!DOCTYPE vpcconfiguration SYSTEM "vpc.dtd">
<vpcconfiguration>
 <resources>
  <component name="CPU" scheduler="FCFS">
  </component>

  <component name="Bus" scheduler="FCFS">
   <attribute type="transfer_delay" value="80 ns" />
  </component>

  <component name="Mem" scheduler="FCFS">
   <attribute type="transfer_delay" value="20 ns" />
  </component>
 </resources>

 <mappings>
  <mapping source="top.Source" target="CPU">
    <timing dii="10 us" latency="10 us" />
    <timing fname="Source::src" dii="10 us" latency="10 us" />
  </mapping>
  <mapping source="top.Sink" target="CPU">
    <timing dii="10 us" latency="10 us" />
    <timing fname="Sink::sink" dii="10 us" latency="10 us" />
  </mapping>
 </mappings>

 <topology>
  <route  source="top.Source" destination="queue">
   <hop name="CPU">  </hop>
   <hop name="Bus">  </hop>
   <hop name="Mem">  </hop>
  </route>
  <route  source="queue" destination="top.Sink">
   <hop name="Mem">  </hop>
   <hop name="Bus">  </hop>
   <hop name="CPU">  </hop>
  </route>
 </topology>
</vpcconfiguration>

\end{lstlisting}
\end{frame}

