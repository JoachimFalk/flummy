\mode<article>
{
}

\mode<presentation>
{
\usetheme{SystemCoDesigner}
\usenavigationsymbolstemplate{} % uncomment to get rid of navigation symbols

}


%\let\Tiny=\tiny
\usepackage[utf8]{inputenc}
\usepackage{pslatex}
\usepackage{graphicx}
\definecolor{lightgray}{gray}{.9}

\usepackage{makeidx}
\makeindex
\mode<presentation>
{
%\newenvironment{theindex}
%               {\if@twocolumn
%                  \@restonecolfalse
%                \else
%                  \@restonecoltrue
%                \fi
%                \twocolumn[\section*{\indexname}]%
%                \@mkboth{\MakeUppercase\indexname}%
%                        {\MakeUppercase\indexname}%
%                \thispagestyle{plain}\parindent\z@
%                \parskip\z@ \@plus .3\p@\relax
%                \columnseprule \z@
%                \columnsep 35\p@
%                \let\item\@idxitem}
%               {\if@restonecol\onecolumn\else\clearpage\fi}

\newenvironment{theindex}%          environment name
{\let\item\idxitem}% begin code
{}%                    end code
\newcommand\idxitem{\par \hspace*{40pt}}
\newcommand\subitem{\idxitem \hspace*{20pt}}
\newcommand\subsubitem{\idxitem \hspace*{30pt}}
\newcommand\indexspace{\par}
%\renewcommand{\seename}{see}
}

\usepackage{expdlist}
\usepackage{listings}
\lstset{language=C++,showstringspaces=false,breaklines=true,basicstyle=\ttfamily}
\lstset{backgroundcolor=\color{lightgray}}

\mode<presentation>
{
\lstset{commentstyle=\scriptsize\selectfont}
\lstset{basicstyle=\scriptsize\ttfamily\selectfont}
}
%\lstset{commentstyle=\color{blue}}
%\lstset{stringstyle=\color{green}}
%\lstset{keywordstyle=\color{red}}
%\lstset{emph={bool,int,unsigned,char,true,false,void}}
%\lstset{emphstyle=\color{orange}}
\lstset{emph={[2]\#include,\#define,\#ifdef,\#endif}}
\lstset{emphstyle={[2]\color{blue}}}

% macros
\newcommand{\SysteMoC}{\emph{SysteMoC}}
\newcommand{\SystemCoDesigner}{\emph{SystemCoDesigner}}
\newcommand{\concat}{{}^{\smallfrown}}
\newcommand{\length}{\#}



\title{SystemCoDesigner: System Programming Tutorial}
\author{}

%\institute[Hardware/Software Co-Design]{Hardware/Software Co-Design\\University of Erlangen-Nuremberg}
\institute[Hardware/Software Co-Design]{Documentation and examples are available online:\\ http://www.mycodesign.com/research/scd}
\mode<presentation>
{
\university{Friedrich-Alexander University of Erlangen-Nuremberg}
}
\date{Version 1.1}

\AtBeginSubsection[]
{
   \begin{frame}
       \frametitle{Outline}
       \tableofcontents[currentsection,currentsubsections,hideothersubsections]
   \end{frame}
}

\begin{document}

\begin{frame}
  \titlepage
\end{frame}

\begin{frame}
  \frametitle{Outline}
  \tableofcontents[hideallsubsections]
\end{frame}



\section{Introduction}
%%%%%%%%%%%%%%%%%%%%%%%%%%%%%%%%%%%%%%%%%%%%%%%%%%%%%%%%%%%%%%%%%%%%%%%%%%%%%%
%%%%%%%%%%%%%%%%%%%%%%%%%%%%%%%%%%%%%%%%%%%%%%%%%%%%%%%%%%%%%%%%%%%%%%%%%%%%%%
\begin{frame}[t]
\mode<presentation>{\frametitle{\insertsubsection\ }}
\begin{itemize}
\item \SystemCoDesigner{} uses \SysteMoC{} and \VPC\ for functional and performance simulation
\end{itemize}
\begin{itemize}
\item \SysteMoC{} allows for functional modeling and simulation
\end{itemize}

\begin{itemize}
\item \VPC\ (VPC) allow for performance modeling and simulation
\end{itemize}

\begin{itemize}
\item Together, \SysteMoC{} and \VPC\, allow for combined functional and performance simulation
\end{itemize}

\begin{itemize}
\item Both, \SysteMoC{} and \VPC\, are implemented as individual libraries on top of SystemC
\end{itemize}

\end{frame}


%%%%%%%%%%%%%%%%%%%%%%%%%%%%%%%%%%%%%%%%%%%%%%%%%%%%%%%%%%%%%%%%%%%%%%%%%%%%%%
%%%%%%%%%%%%%%%%%%%%%%%%%%%%%%%%%%%%%%%%%%%%%%%%%%%%%%%%%%%%%%%%%%%%%%%%%%%%%%
\begin{frame}[t]
\mode<presentation>{\frametitle{\insertsection\ -- VPC }}
\begin{itemize}
\item A \VPC\ ...
\item ... is a SystemC module
\item ... models task execution times
\item ... models resource contention, scheduling, and arbitration
\end{itemize}

\begin{itemize}
\item A \VPC\ ...
\item ... is not an Instruction Set Simulator
\item ... is not Software
\item ... is not Hardware
\end{itemize}

\begin{itemize}
\item But, a \VPC\ ...
\item ... allows to model a Processor or a HW component
\item ... allows to model a communication resource
\end{itemize}

\end{frame}



%%%%%%%%%%%%%%%%%%%%%%%%%%%%%%%%%%%%%%%%%%%%%%%%%%%%%%%%%%%%%%%%%%%%%%%%%%%%%%
%%%%%%%%%%%%%%%%%%%%%%%%%%%%%%%%%%%%%%%%%%%%%%%%%%%%%%%%%%%%%%%%%%%%%%%%%%%%%%
\begin{frame}[t]
\mode<presentation>{\frametitle{\insertsection\ -- \SysteMoC\ }}

\begin{figure}
\centering
\resizebox{0.7\columnwidth}{!}{\input{vpc-systemoc-fig.tex}}
\end{figure}

\begin{itemize}
\item A functional model in \SysteMoC\ is given by ...
\item ... a set of actors (e.g., Source and Sink)
\item ... a set of communication queues connecting actors 
\end{itemize}
\end{frame}



%%%%%%%%%%%%%%%%%%%%%%%%%%%%%%%%%%%%%%%%%%%%%%%%%%%%%%%%%%%%%%%%%%%%%%%%%%%%%%
%%%%%%%%%%%%%%%%%%%%%%%%%%%%%%%%%%%%%%%%%%%%%%%%%%%%%%%%%%%%%%%%%%%%%%%%%%%%%%
\begin{frame}[t]
\mode<presentation>{\frametitle{\VPC\ }}

\begin{figure}
\centering
\resizebox{0.7\columnwidth}{!}{\input{vpc-systemoc-fig.tex}}
\end{figure}

\begin{itemize}
\item An architecture model in VPC is given by ...
\item ... a set of components (e.g., CPU, Bus, and Mem )
\item ... a mapping of actors and queues to the set of components
\end{itemize}
\end{frame}











\section{Programming Examples}
\subsection{Hello World}
%%%%%%%%%%%%%%%%%%%%%%%%%%%%%%%%%%%%%%%%%%%%%%%%%%%%%%%%%%%%%%%%%%%%%%%%%%%%%%
%%%%%%%%%%%%%%%%%%%%%%%%%%%%%%%%%%%%%%%%%%%%%%%%%%%%%%%%%%%%%%%%%%%%%%%%%%%%%%
\begin{frame}
\mode<presentation>{\frametitle{\insertsubsection\ -- Objectives}}
\begin{itemize}
\item You will see a state-of-the-art ``Hello World'' application ...
\item ...  modeled in SysteMoC.
\end{itemize}
\end{frame}




%%%%%%%%%%%%%%%%%%%%%%%%%%%%%%%%%%%%%%%%%%%%%%%%%%%%%%%%%%%%%%%%%%%%%%%%%%%%%%
%%%%%%%%%%%%%%%%%%%%%%%%%%%%%%%%%%%%%%%%%%%%%%%%%%%%%%%%%%%%%%%%%%%%%%%%%%%%%%
 % fragile is mandratory for verbatim environments (lstlisting)
 % HACK: use singleslide for correct page numbering
 %  (pagecounter+=2 else wise)
 %  this issue seems to be relevent for *pdflatex* only
 % NOTE: the intended usage for *singleslide* is to tell beamer that we use
 %   no overlays (animation)
\begin{frame}[fragile=singleslide]
\mode<presentation>{\frametitle{\insertsubsection}}
\begin{lstlisting}
//file hello.cpp
#include <iostream>
#include <systemoc/smoc_moc.hpp>

class HelloActor: public smoc_actor {
private:
  void src() {   // action
    std::cout << "Actor " << this->name() << " says:\n"
              << "Hello SysteMoC" << std::endl;
  }

  smoc_firing_state start, end;   // FSM states
public:
  // actor constructor
  HelloActor(sc_module_name name)
    : smoc_actor(name, start) {
    // FSM definition:
    //  transition from start to end calling action src
    start = CALL(HelloActor::src) >> end;
  }
};
\end{lstlisting}
\end{frame}

%%%%%%%%%%%%%%%%%%%%%%%%%%%%%%%%%%%%%%%%%%%%%%%%%%%%%%%%%%%%%%%%%%%%%%%%%%%%%%
%%%%%%%%%%%%%%%%%%%%%%%%%%%%%%%%%%%%%%%%%%%%%%%%%%%%%%%%%%%%%%%%%%%%%%%%%%%%%%
\begin{frame}[fragile=singleslide]
\mode<presentation>{\frametitle{\insertsubsection}}
\begin{lstlisting}
//file hello.cpp cont'd

class HelloNetworkGraph: public smoc_graph {
private:
  // actors
  HelloActor     helloActor;
public:
  // networkgraph constructor
  HelloNetworkGraph(sc_module_name name)
    : smoc_graph(name),
      helloActor("HalloActor") // create actor HelloWorld
  { }
};

int sc_main (int argc, char **argv) {
  // create networkgraph
  smoc_top_moc<HelloNetworkGraph> top("top");

  sc_start();  // start simulation (SystemC)
  return 0;
}
\end{lstlisting}
\end{frame}

%%%%%%%%%%%%%%%%%%%%%%%%%%%%%%%%%%%%%%%%%%%%%%%%%%%%%%%%%%%%%%%%%%%%%%%%%%%%%%
%%%%%%%%%%%%%%%%%%%%%%%%%%%%%%%%%%%%%%%%%%%%%%%%%%%%%%%%%%%%%%%%%%%%%%%%%%%%%%
\begin{frame}[fragile=singleslide]
\mode<presentation>{\frametitle{\insertsubsection}}
\begin{itemize}
\item run simulation
\begin{lstlisting}
./hello
\end{lstlisting}
\item simulation output
\begin{lstlisting}
             SystemC 2.2.0 --- Dec 15 2008 11:10:07
        Copyright (c) 1996-2006 by all Contributors
                    ALL RIGHTS RESERVED
Actor top.HalloActor says:
Hello SysteMoC
SystemC: simulation stopped by user.
\end{lstlisting}
\end{itemize}
\end{frame}






\subsection{Actors and Graphs}
%%%%%%%%%%%%%%%%%%%%%%%%%%%%%%%%%%%%%%%%%%%%%%%%%%%%%%%%%%%%%%%%%%%%%%%%%%%%%%
%%%%%%%%%%%%%%%%%%%%%%%%%%%%%%%%%%%%%%%%%%%%%%%%%%%%%%%%%%%%%%%%%%%%%%%%%%%%%%
\begin{frame}
\mode<presentation>{\frametitle{\insertsubsection\ -- Objectives}}
\begin{itemize}
\item You will learn to ...
\item ... write actors.
\item ... instantiate actors in a network graph.
\item ... run functional simulation.
\end{itemize}
\end{frame}




%%%%%%%%%%%%%%%%%%%%%%%%%%%%%%%%%%%%%%%%%%%%%%%%%%%%%%%%%%%%%%%%%%%%%%%%%%%%%%
%%%%%%%%%%%%%%%%%%%%%%%%%%%%%%%%%%%%%%%%%%%%%%%%%%%%%%%%%%%%%%%%%%%%%%%%%%%%%%
\begin{frame}[fragile=singleslide]
\mode<presentation>{\frametitle{\insertsubsection\ -- Actors}}
\index{actor|(}
\index{smoc_actor@\lstinline{smoc_actor}}
\begin{itemize}
\item include SysteMoC library header
\begin{lstlisting}
#include <systemoc/smoc_moc.hpp>
\end{lstlisting}
\item an actor is a C++ class derived from base class \lstinline!smoc_actor!
\begin{lstlisting}
class HelloActor: public smoc_actor {
\end{lstlisting}
\item an action is a member function of the actors class
\item actions shall be \lstinline!private! $\rightarrow$ prohibit execution by others
\begin{lstlisting}
private:
  void src() {   // action
    std::cout << "Actor " << this->name() << " says:\n"
              << "Hello World" << std::endl;
  }
\end{lstlisting}
\end{itemize}
\end{frame}





%%%%%%%%%%%%%%%%%%%%%%%%%%%%%%%%%%%%%%%%%%%%%%%%%%%%%%%%%%%%%%%%%%%%%%%%%%%%%%
%%%%%%%%%%%%%%%%%%%%%%%%%%%%%%%%%%%%%%%%%%%%%%%%%%%%%%%%%%%%%%%%%%%%%%%%%%%%%%
\begin{frame}[fragile=singleslide]
\mode<presentation>{\frametitle{\insertsubsection\ -- Actors}}
\index{state}
\index{smoc_firing_state@\lstinline{smoc_firing_state}}
\index{smoc_actor@\lstinline{smoc_actor}}
\begin{itemize}
\item an actor has a finite set of states
\begin{lstlisting}
  smoc_firing_state state_a, state_b;   // FSM states
\end{lstlisting}
\item like objects of a class in C++, we create instances of an actor
\item constructors are responsible for creating a certain actor instance
\item provide actor name and start state to base class \lstinline!smoc_actor!
\begin{lstlisting}
  HelloActor(sc_module_name name)  // actor constructor
    : smoc_actor(name, state_a) {
\end{lstlisting}
\end{itemize}
\end{frame}




%%%%%%%%%%%%%%%%%%%%%%%%%%%%%%%%%%%%%%%%%%%%%%%%%%%%%%%%%%%%%%%%%%%%%%%%%%%%%%
%%%%%%%%%%%%%%%%%%%%%%%%%%%%%%%%%%%%%%%%%%%%%%%%%%%%%%%%%%%%%%%%%%%%%%%%%%%%%%
\begin{frame}[fragile=singleslide]
\mode<presentation>{\frametitle{\insertsubsection\ -- Actors}}
\index{FSM}
\index{CALL@\lstinline{CALL}}
\begin{itemize}
\item a finite state machine (FSM) defines transitions between states
\item e.g. a transition from \lstinline!state_a! to \lstinline!state_b!
\item if this transition is taken, the action \lstinline!HelloActor::src! is executed
\begin{lstlisting}
  HelloActor(sc_module_name name)  // actor constructor
    : smoc_actor(name, state_a) {

  // FSM definition:
  //  transition from state_a to state_b calling action src
  state_a = CALL(HelloActor::src) >> state_b;

}
\end{lstlisting}
\end{itemize}
\index{actor|)}
\end{frame}





%%%%%%%%%%%%%%%%%%%%%%%%%%%%%%%%%%%%%%%%%%%%%%%%%%%%%%%%%%%%%%%%%%%%%%%%%%%%%%
%%%%%%%%%%%%%%%%%%%%%%%%%%%%%%%%%%%%%%%%%%%%%%%%%%%%%%%%%%%%%%%%%%%%%%%%%%%%%%
\begin{frame}[fragile=singleslide]
\mode<presentation>{\frametitle{\insertsubsection\ -- Graphs}}
\index{network graph}
\index{smoc_graph@\lstinline{smoc_graph}}
\begin{itemize}
\item a network graph is derived from base class \lstinline!smoc_graph!
\begin{lstlisting}
class HelloNetworkGraph: public smoc_graph {
\end{lstlisting}
\item our network graph has an instance of the \lstinline!HelloActor!
\begin{lstlisting}
private:
  // actors
  HelloActor     helloActor;
\end{lstlisting}
\item a name for the graph has to be passed to base class \lstinline!smoc_graph!
\item we need to call \lstinline!HelloActor!s constructor
\begin{lstlisting}
  // network graph constructor
  HelloNetworkGraph(sc_module_name name)
    : smoc_graph(name),
      helloActor("HelloActor") // create actor HelloWorld
  { }
};
\end{lstlisting}
\end{itemize}
\end{frame}


\subsection{Simulation}
%%%%%%%%%%%%%%%%%%%%%%%%%%%%%%%%%%%%%%%%%%%%%%%%%%%%%%%%%%%%%%%%%%%%%%%%%%%%%%
%%%%%%%%%%%%%%%%%%%%%%%%%%%%%%%%%%%%%%%%%%%%%%%%%%%%%%%%%%%%%%%%%%%%%%%%%%%%%%
\begin{frame}[fragile=singleslide]
\mode<presentation>{\frametitle{\insertsubsection\ }}
\index{simulation}
\begin{itemize}
\item function \lstinline!sc_main! is the entry point for simulation (cf. SystemC)
\item the network graph is instantiated
\item parameters are passed to the graph constructor (e.g.``top'')
\item call \lstinline!sc_start! to start simulation (cf. SystemC)
\begin{lstlisting}
int sc_main (int argc, char **argv) {
  // create network graph
  HelloNetworkGraph top("top");
  smoc_scheduler_top sched(top);

  sc_start();  // start simulation (SystemC)
  return 0;
}
\end{lstlisting}
\end{itemize}
\end{frame}

\subsection{Ports and Channels}
%%%%%%%%%%%%%%%%%%%%%%%%%%%%%%%%%%%%%%%%%%%%%%%%%%%%%%%%%%%%%%%%%%%%%%%%%%%%%%
%%%%%%%%%%%%%%%%%%%%%%%%%%%%%%%%%%%%%%%%%%%%%%%%%%%%%%%%%%%%%%%%%%%%%%%%%%%%%%
\begin{frame}
\mode<presentation>{\frametitle{\insertsubsection\ -- Objectives}}
\begin{itemize}
\item You will learn to ...
\item ... use ports.
\item ... connect actors via channels.
\end{itemize}
\end{frame}




%%%%%%%%%%%%%%%%%%%%%%%%%%%%%%%%%%%%%%%%%%%%%%%%%%%%%%%%%%%%%%%%%%%%%%%%%%%%%%
%%%%%%%%%%%%%%%%%%%%%%%%%%%%%%%%%%%%%%%%%%%%%%%%%%%%%%%%%%%%%%%%%%%%%%%%%%%%%%
\begin{frame}[fragile=singleslide]
\mode<presentation>{\frametitle{\insertsubsection\ -- Sink Actor}}
\index{port|(}
\begin{lstlisting}
class Sink: public smoc_actor {
public:
  // ports:
  smoc_port_in<char> in;

  Sink(sc_module_name name)   // actor constructor
    : smoc_actor(name, start) {
    // FSM definition:
    start =
      in(1)                 >>
      CALL(Sink::sink)      >> start;
  }
private:
  smoc_firing_state start;  // FSM states

  void sink() {
    std::cout << this->name() << " recv: \""
              << in[0] << "\"" << std::endl;
  }
};
\end{lstlisting}
\end{frame}






%%%%%%%%%%%%%%%%%%%%%%%%%%%%%%%%%%%%%%%%%%%%%%%%%%%%%%%%%%%%%%%%%%%%%%%%%%%%%%
%%%%%%%%%%%%%%%%%%%%%%%%%%%%%%%%%%%%%%%%%%%%%%%%%%%%%%%%%%%%%%%%%%%%%%%%%%%%%%
\begin{frame}[fragile=singleslide]
\mode<presentation>{\frametitle{\insertsubsection\ -- Input Port}}
%\index{actor!input port|see{input port}}
\index{input port}
\begin{itemize}
\item create an input port
\item ports have a data type (e.g. \lstinline!char!)
\begin{lstlisting}
  smoc_port_in<char> in;
\end{lstlisting}
\item a single object of the particular data type is termed ``a token''
\item declare to read one token in FSM transition
\begin{lstlisting}
    start =
      in(1)                 >>
      CALL(Sink::sink)      >> start;
\end{lstlisting}
\item write data in action
\begin{lstlisting}
  void sink() {
    std::cout << this->name() << " recv: \'"
              << in[0] << "\'" << std::endl;
  }
\end{lstlisting}
\end{itemize}
\end{frame}




%%%%%%%%%%%%%%%%%%%%%%%%%%%%%%%%%%%%%%%%%%%%%%%%%%%%%%%%%%%%%%%%%%%%%%%%%%%%%%
%%%%%%%%%%%%%%%%%%%%%%%%%%%%%%%%%%%%%%%%%%%%%%%%%%%%%%%%%%%%%%%%%%%%%%%%%%%%%%
\begin{frame}[fragile=singleslide]
\mode<presentation>{\frametitle{\insertsubsection\ -- Source Actor}}
\begin{itemize}
\item using output ports is similar
\end{itemize}
\begin{lstlisting}
class Source: public smoc_actor {
public:
  // ports:
  smoc_port_out<char> out;

  Source(sc_module_name name)
    : smoc_actor(name, start) {
    start = 
      out(1)                   >>
      CALL(Source::src)        >> start;
  }
private:
  smoc_firing_state start;  // FSM states

  void src() {
    std::cout << this->name() << " send: \'X\'" << std::endl;
    out[0] = 'X';
  }
};
\end{lstlisting}
\end{frame}






%%%%%%%%%%%%%%%%%%%%%%%%%%%%%%%%%%%%%%%%%%%%%%%%%%%%%%%%%%%%%%%%%%%%%%%%%%%%%%
%%%%%%%%%%%%%%%%%%%%%%%%%%%%%%%%%%%%%%%%%%%%%%%%%%%%%%%%%%%%%%%%%%%%%%%%%%%%%%
\begin{frame}[fragile=singleslide]
\mode<presentation>{\frametitle{\insertsubsection\ -- Output Port}}
%\index{actor!output port|see{output port}}
\index{output port}
\index{port|)}
\begin{itemize}
\item create an output with data type \lstinline!char!
\begin{lstlisting}
smoc_port_out<char> out;
\end{lstlisting}
\item declare to write one token in FSM transition
\begin{lstlisting}
  start = 
    out(1)                   >>
    CALL(Source::src)        >> start;
\end{lstlisting}
\item access data in action
\begin{lstlisting}
void src() {
  std::cout << this->name() << " send: \'X\'" << std::endl;
  out[0] = 'X';
}
\end{lstlisting}
\end{itemize}
\end{frame}




%%%%%%%%%%%%%%%%%%%%%%%%%%%%%%%%%%%%%%%%%%%%%%%%%%%%%%%%%%%%%%%%%%%%%%%%%%%%%%
%%%%%%%%%%%%%%%%%%%%%%%%%%%%%%%%%%%%%%%%%%%%%%%%%%%%%%%%%%%%%%%%%%%%%%%%%%%%%%
\begin{frame}[fragile=singleslide]
\mode<presentation>{\frametitle{\insertsubsection\ -- Connect Actors}}
\index{connectNodePorts@\lstinline{connectNodePorts}|(}
\begin{lstlisting}
class NetworkGraph: public smoc_graph {
public:
  NetworkGraph(sc_module_name name)  // network graph constructor
    : smoc_graph(name),
      source("Source"),             // create actors
      sink("Sink") {
    connectNodePorts(source.out, sink.in); // connect actors
  }
private:
  Source         source;   // actors
  Sink           sink;
};

int sc_main (int argc, char **argv) {
  smoc_top_moc<NetworkGraph> top("top"); // create network graph

  sc_start();   // start simulation (SystemC)
  return 0;
}
\end{lstlisting}
\end{frame}




%%%%%%%%%%%%%%%%%%%%%%%%%%%%%%%%%%%%%%%%%%%%%%%%%%%%%%%%%%%%%%%%%%%%%%%%%%%%%%
%%%%%%%%%%%%%%%%%%%%%%%%%%%%%%%%%%%%%%%%%%%%%%%%%%%%%%%%%%%%%%%%%%%%%%%%%%%%%%
\begin{frame}[fragile=singleslide]
\mode<presentation>{\frametitle{\insertsubsection\ -- FIFO Queues}}
\index{connectNodePorts@\lstinline{connectNodePorts}|)}
\begin{itemize}
\item connect a pair of ports (input, output) using a FIFO queue
\item connected ports have to use the same data type
\item queues have default size ``1'' (one data token)
\begin{lstlisting}
    connectNodePorts(source.out, sink.in);
\end{lstlisting}
\item set queue size explicitly
\begin{lstlisting}
    connectNodePorts<23>(source.out, sink.in);
\end{lstlisting}
\item more advanced channel features (later)
\begin{itemize}
\item channel types
\item initial data tokens
\end{itemize}
\end{itemize}
\end{frame}




%%%%%%%%%%%%%%%%%%%%%%%%%%%%%%%%%%%%%%%%%%%%%%%%%%%%%%%%%%%%%%%%%%%%%%%%%%%%%%
%%%%%%%%%%%%%%%%%%%%%%%%%%%%%%%%%%%%%%%%%%%%%%%%%%%%%%%%%%%%%%%%%%%%%%%%%%%%%%
\begin{frame}[fragile=singleslide]
\mode<presentation>{\frametitle{\insertsubsection}}
\begin{itemize}
\item simulation output
\begin{lstlisting}

             SystemC 2.2.0 --- Dec 15 2008 11:10:07
        Copyright (c) 1996-2006 by all Contributors
                    ALL RIGHTS RESERVED
top.Source send: 'X'
top.Sink recv: 'X'
top.Source send: 'X'
top.Sink recv: 'X'
top.Source send: 'X'
top.Sink recv: 'X'
...
\end{lstlisting}
\item simulation runs infinitely
\end{itemize}
\end{frame}






\subsection{Guards and Actions}
%%%%%%%%%%%%%%%%%%%%%%%%%%%%%%%%%%%%%%%%%%%%%%%%%%%%%%%%%%%%%%%%%%%%%%%%%%%%%%
%%%%%%%%%%%%%%%%%%%%%%%%%%%%%%%%%%%%%%%%%%%%%%%%%%%%%%%%%%%%%%%%%%%%%%%%%%%%%%
\begin{frame}
\mode<presentation>{\frametitle{\insertsubsection\ -- Objectives}}
\begin{itemize}
\item You will learn to ...
\item ... write and use guards.
\item ... write and use actions.
\end{itemize}
\end{frame}




%%%%%%%%%%%%%%%%%%%%%%%%%%%%%%%%%%%%%%%%%%%%%%%%%%%%%%%%%%%%%%%%%%%%%%%%%%%%%%
%%%%%%%%%%%%%%%%%%%%%%%%%%%%%%%%%%%%%%%%%%%%%%%%%%%%%%%%%%%%%%%%%%%%%%%%%%%%%%
\begin{frame}[fragile=singleslide]
\mode<presentation>{\frametitle{\insertsubsection\ -- SourceActor}}
\begin{lstlisting}
static const char MESSAGE [] = "0123456789";

class Source: public smoc_actor {
public:
  smoc_port_out<char> out;
  Source(sc_module_name name) : smoc_actor(name, start),
    count(0), size(strlen(MESSAGE)), message(MESSAGE) {
    start = 
      GUARD(Source::hasToken)  >>
      out(1)                   >>
      CALL(Source::src)        >> start;
  }
private:
  smoc_firing_state start;

  unsigned int count, size;  // variables (functional state)
  const char* message;       //

  bool hasToken() const{ return count<size; } // guard
  void src() { out[0] = message[count++]; }   // action
};

\end{lstlisting}
\end{frame}






%%%%%%%%%%%%%%%%%%%%%%%%%%%%%%%%%%%%%%%%%%%%%%%%%%%%%%%%%%%%%%%%%%%%%%%%%%%%%%
%%%%%%%%%%%%%%%%%%%%%%%%%%%%%%%%%%%%%%%%%%%%%%%%%%%%%%%%%%%%%%%%%%%%%%%%%%%%%%
\begin{frame}[fragile=singleslide]
\mode<presentation>{\frametitle{\insertsubsection\ -- Guards}}
\index{guard}
\index{GUARD@\lstinline{GUARD}}
\begin{itemize}
\item a guard is a \lstinline!const! member function returning a boolean value
\begin{lstlisting}
  bool hasToken() const{
    return count<size;
  } // guard
\end{lstlisting}
\item guards enable/disable transitions (true/false)
\item guards must not change variable values or token in channels
\item refer to guards via \lstinline!GUARD(..)! macro 
\begin{lstlisting}
    start = 
      GUARD(Source::hasToken)  >>
      out(1)                   >>
      CALL(Source::src)        >> start;
\end{lstlisting}
\item use guards for control flow (see below)
\end{itemize}
\end{frame}







%%%%%%%%%%%%%%%%%%%%%%%%%%%%%%%%%%%%%%%%%%%%%%%%%%%%%%%%%%%%%%%%%%%%%%%%%%%%%%
%%%%%%%%%%%%%%%%%%%%%%%%%%%%%%%%%%%%%%%%%%%%%%%%%%%%%%%%%%%%%%%%%%%%%%%%%%%%%%
\begin{frame}[fragile=singleslide]
\mode<presentation>{\frametitle{\insertsubsection\ -- Variables}}
\index{variables}
\begin{itemize}
\item variables ...
\item ... are \lstinline!private! class member of an actor
\item ... can be used to store data
\item ... represent a functional state of an actor (in contrast to FSM state)
\begin{lstlisting}
  unsigned int count, size;
  const char* message;
\end{lstlisting}
\end{itemize}
\end{frame}






%%%%%%%%%%%%%%%%%%%%%%%%%%%%%%%%%%%%%%%%%%%%%%%%%%%%%%%%%%%%%%%%%%%%%%%%%%%%%%
%%%%%%%%%%%%%%%%%%%%%%%%%%%%%%%%%%%%%%%%%%%%%%%%%%%%%%%%%%%%%%%%%%%%%%%%%%%%%%
\begin{frame}[fragile=singleslide]
\mode<presentation>{\frametitle{\insertsubsection\ -- Actions}}
\index{actions}
\begin{itemize}
\item actions ...
\item ... are used to read/write data on input/output ports
\item ... modify variables
\begin{lstlisting}
  void src() {
    out[0] = message[count++];
  }
\end{lstlisting}
\item guards access variables read-only (mandatory \lstinline!const! modifier)
\item actions are allowed to modify variables
\end{itemize}
\end{frame}






%%%%%%%%%%%%%%%%%%%%%%%%%%%%%%%%%%%%%%%%%%%%%%%%%%%%%%%%%%%%%%%%%%%%%%%%%%%%%%
%%%%%%%%%%%%%%%%%%%%%%%%%%%%%%%%%%%%%%%%%%%%%%%%%%%%%%%%%%%%%%%%%%%%%%%%%%%%%%
\begin{frame}[fragile=singleslide]
\mode<presentation>{\frametitle{\insertsubsection\ -- Multiple Access}}
\index{port}
\begin{itemize}
\item you can write/read data more than once (overwrite/re-read)
\item e.g. read input twice
\begin{lstlisting}
  void sink() {
    char squareInput = in[0] * in[0];

    char x = in[0];
    char y = in[0]; // re-read
    assert(x == y);
  }
\end{lstlisting}
\item e.g. write a default value first
\begin{lstlisting}
  void src() {
    out[0] = 'X';    // default
    if(count<size){
      out[0] = message[count++]; //overwrite
    }
  }
\end{lstlisting}
\end{itemize}
\end{frame}










%%%%%%%%%%%%%%%%%%%%%%%%%%%%%%%%%%%%%%%%%%%%%%%%%%%%%%%%%%%%%%%%%%%%%%%%%%%%%%
%%%%%%%%%%%%%%%%%%%%%%%%%%%%%%%%%%%%%%%%%%%%%%%%%%%%%%%%%%%%%%%%%%%%%%%%%%%%%%
\begin{frame}[fragile=singleslide]
\mode<presentation>{\frametitle{\insertsubsection}}
\begin{itemize}
\item simulation output (using Sink actor from previous example)
\begin{lstlisting}
top.Sink recv: "0"
top.Sink recv: "1"
top.Sink recv: "2"
top.Sink recv: "3"
top.Sink recv: "4"
top.Sink recv: "5"
top.Sink recv: "6"
top.Sink recv: "7"
top.Sink recv: "8"
top.Sink recv: "9"
SystemC: simulation stopped by user.
\end{lstlisting}
\item Source actor sends a finite number of characters only
\item simulation terminates when no actor can be activated
\end{itemize}
\end{frame}





%%%%%%%%%%%%%%%%%%%%%%%%%%%%%%%%%%%%%%%%%%%%%%%%%%%%%%%%%%%%%%%%%%%%%%%%%%%%%%
%%%%%%%%%%%%%%%%%%%%%%%%%%%%%%%%%%%%%%%%%%%%%%%%%%%%%%%%%%%%%%%%%%%%%%%%%%%%%%
%\begin{frame}[fragile=singleslide=singleslide]
%\mode<presentation>{\frametitle{\insertsubsection}}
%\begin{itemize}
%\item ...
%\end{itemize}
%\begin{lstlisting}
%
%\end{lstlisting}
%\end{frame}






\subsection{Communication Rates}
%%%%%%%%%%%%%%%%%%%%%%%%%%%%%%%%%%%%%%%%%%%%%%%%%%%%%%%%%%%%%%%%%%%%%%%%%%%%%%
%%%%%%%%%%%%%%%%%%%%%%%%%%%%%%%%%%%%%%%%%%%%%%%%%%%%%%%%%%%%%%%%%%%%%%%%%%%%%%
\begin{frame}
\mode<presentation>{\frametitle{\insertsubsection\ -- Objectives}}
\begin{itemize}
\item former examples consume/produce (read/write) only single tokens
\item consumption and production rates may be different from one
\end{itemize}
\begin{itemize}
\item You will learn to ...
\item ... use consumption and production rates.
\end{itemize}
\end{frame}




%%%%%%%%%%%%%%%%%%%%%%%%%%%%%%%%%%%%%%%%%%%%%%%%%%%%%%%%%%%%%%%%%%%%%%%%%%%%%%
%%%%%%%%%%%%%%%%%%%%%%%%%%%%%%%%%%%%%%%%%%%%%%%%%%%%%%%%%%%%%%%%%%%%%%%%%%%%%%
\begin{frame}[fragile=singleslide]
\mode<presentation>{\frametitle{\insertsubsection\ -- Source Actor }}
\begin{lstlisting}
static const char MESSAGE []   = "0123456789";
class Source: public smoc_actor {
public:
  smoc_port_out<char> out;
  Source(sc_module_name name) : smoc_actor(name, start),
    count(0), size(strlen(MESSAGE)), message(MESSAGE) {
    start = 
      GUARD(Source::hasToken)  >>
      out(2)                   >>
      CALL(Source::src)        >> start;
  }
private:
  smoc_firing_state start;
  unsigned int count, size;  // variables (functional state)
  const char* message;       //

  bool hasToken() const{ return count<size; } // guard
  void src() {                                // action
    out[0] = message[count++];
    out[1] = message[count++];
  }};
\end{lstlisting}
\end{frame}





%%%%%%%%%%%%%%%%%%%%%%%%%%%%%%%%%%%%%%%%%%%%%%%%%%%%%%%%%%%%%%%%%%%%%%%%%%%%%%
%%%%%%%%%%%%%%%%%%%%%%%%%%%%%%%%%%%%%%%%%%%%%%%%%%%%%%%%%%%%%%%%%%%%%%%%%%%%%%
\begin{frame}[fragile=singleslide]
\mode<presentation>{\frametitle{\insertsubsection\ -- Source Actor}}
\begin{itemize}
\item the Source actor shall produce two data tokens at once
\item we declare to produce $X$ data tokens in a transition  using \lstinline!out(!$X$\lstinline!)!
\begin{lstlisting}
    start = 
      GUARD(Source::hasToken)  >>
      out(2)                   >>
      CALL(Source::src)        >> start;
\end{lstlisting}
\item use the array operator (\lstinline![]!) to write two data tokens
\begin{lstlisting}
  void src() {
    out[0] = message[count++];
    out[1] = message[count++];
  }
\end{lstlisting}
\item similar to arrays addressing range is $0, \dots, X-1$
\end{itemize}
\end{frame}





%%%%%%%%%%%%%%%%%%%%%%%%%%%%%%%%%%%%%%%%%%%%%%%%%%%%%%%%%%%%%%%%%%%%%%%%%%%%%%
%%%%%%%%%%%%%%%%%%%%%%%%%%%%%%%%%%%%%%%%%%%%%%%%%%%%%%%%%%%%%%%%%%%%%%%%%%%%%%
\begin{frame}[fragile=singleslide]
\mode<presentation>{\frametitle{\insertsubsection\ -- Simulation}}
\begin{itemize}
\item simulation output (using Sink actor from previous example)
\end{itemize}
\begin{lstlisting}
             SystemC 2.2.0 --- Dec 15 2008 11:10:07
        Copyright (c) 1996-2006 by all Contributors
                    ALL RIGHTS RESERVED
SystemC: simulation stopped by user.
\end{lstlisting}
\begin{itemize}
\item Oh oh, nothing happens!
\item writing two tokens requires a free space for (at least) two tokens
\item we need to increase the queue size (implicit size was ``1'')
\begin{lstlisting}
    connectNodePorts<4>(source.out, sink.in);
\end{lstlisting}
\item minimum size of $2$ is mandatory (but actor would run in lockstep)
\item using larger sized queues may decouple execution of actors
\end{itemize}
\end{frame}





%%%%%%%%%%%%%%%%%%%%%%%%%%%%%%%%%%%%%%%%%%%%%%%%%%%%%%%%%%%%%%%%%%%%%%%%%%%%%%
%%%%%%%%%%%%%%%%%%%%%%%%%%%%%%%%%%%%%%%%%%%%%%%%%%%%%%%%%%%%%%%%%%%%%%%%%%%%%%
\begin{frame}[fragile=singleslide]
\mode<presentation>{\frametitle{\insertsubsection\ -- Simulation}}
\begin{itemize}
\item simulation output (using Sink actor from previous example)
\item and increased queue size
\end{itemize}
\begin{lstlisting}
             SystemC 2.2.0 --- Dec 15 2008 11:10:07
        Copyright (c) 1996-2006 by all Contributors
                    ALL RIGHTS RESERVED
top.Sink recv: "0"
top.Sink recv: "1"
top.Sink recv: "2"
top.Sink recv: "3"
top.Sink recv: "4"
top.Sink recv: "5"
top.Sink recv: "6"
top.Sink recv: "7"
top.Sink recv: "8"
top.Sink recv: "9"
SystemC: simulation stopped by user.
\end{lstlisting}
\end{frame}





%%%%%%%%%%%%%%%%%%%%%%%%%%%%%%%%%%%%%%%%%%%%%%%%%%%%%%%%%%%%%%%%%%%%%%%%%%%%%%
%%%%%%%%%%%%%%%%%%%%%%%%%%%%%%%%%%%%%%%%%%%%%%%%%%%%%%%%%%%%%%%%%%%%%%%%%%%%%%
\begin{frame}[fragile=singleslide]
\mode<presentation>{\frametitle{\insertsubsection\ -- Queue Access}}
\begin{center}
\resizebox{0.95\columnwidth}{!}{\input{queue-fig.tex}}
\end{center}
\begin{itemize}
\item a transition refers to $X$ input port $i$ using \lstinline!i(!$X$\lstinline!)!
\item enough tokens $\Rightarrow$ transition may be fired $\Rightarrow$ action executed once
\item access data in FIFO order using \lstinline!i[!$n$\lstinline!]!; $n \in {0, \dots, X-1}$
\item you can write/read data more than once (overwrite/re-read)
\item writing data is similar (using output ports)
\item \lstinline!o[0]! / \lstinline!i[0]! refers first free position / first available token in a queue
\end{itemize}
\end{frame}






\subsection{Initial Tokens}
%%%%%%%%%%%%%%%%%%%%%%%%%%%%%%%%%%%%%%%%%%%%%%%%%%%%%%%%%%%%%%%%%%%%%%%%%%%%%%
%%%%%%%%%%%%%%%%%%%%%%%%%%%%%%%%%%%%%%%%%%%%%%%%%%%%%%%%%%%%%%%%%%%%%%%%%%%%%%
\begin{frame}
\mode<presentation>{\frametitle{\insertsubsection\ -- Objectives}}
\begin{itemize}
\item data flow graphs may contain cyclic dependencies
\item initial token are a common way to break cyclic dependencies
\end{itemize}
\begin{itemize}
\item You will learn to ...
\item ... initialize queus.
\end{itemize}
\end{frame}


%%%%%%%%%%%%%%%%%%%%%%%%%%%%%%%%%%%%%%%%%%%%%%%%%%%%%%%%%%%%%%%%%%%%%%%%%%%%%%
%%%%%%%%%%%%%%%%%%%%%%%%%%%%%%%%%%%%%%%%%%%%%%%%%%%%%%%%%%%%%%%%%%%%%%%%%%%%%%
\begin{frame}
\mode<presentation>{\frametitle{\insertsubsection\ -- Cyclic Dependencies}}
\begin{figure}
\centering
\resizebox{0.9\columnwidth}{!}{\input{init-token-fig.tex}}
\end{figure}
\begin{itemize}
\item each actor forwards input tokens to output port
\item network graph contains a cyclic dependency
\item without initialization none of the actors would become active
\item we may use an initial token to break the cyclic dependency
\end{itemize}
\end{frame}





%%%%%%%%%%%%%%%%%%%%%%%%%%%%%%%%%%%%%%%%%%%%%%%%%%%%%%%%%%%%%%%%%%%%%%%%%%%%%%
%%%%%%%%%%%%%%%%%%%%%%%%%%%%%%%%%%%%%%%%%%%%%%%%%%%%%%%%%%%%%%%%%%%%%%%%%%%%%%
\begin{frame}[fragile=singleslide]
\mode<presentation>{\frametitle{\insertsubsection\ -- Example}}
\begin{lstlisting}
class Forward: public smoc_actor {
public:
  smoc_port_in<int> in;
  smoc_port_out<int> out;

  Forward(sc_module_name name) : smoc_actor(name, start){
    start = 
      in(1)                    >>
      out(1)                   >>
      CALL(Forward::forward)   >> start;
  }
private:
  smoc_firing_state start;

  void forward() {
    std::cout << this->name() << " forward: \""
              << in[0] << "\"" << std::endl;
    out[0] = in[0];
  }
};
\end{lstlisting}
\end{frame}





%%%%%%%%%%%%%%%%%%%%%%%%%%%%%%%%%%%%%%%%%%%%%%%%%%%%%%%%%%%%%%%%%%%%%%%%%%%%%%
%%%%%%%%%%%%%%%%%%%%%%%%%%%%%%%%%%%%%%%%%%%%%%%%%%%%%%%%%%%%%%%%%%%%%%%%%%%%%%
\begin{frame}[fragile=singleslide]
\mode<presentation>{\frametitle{\insertsubsection\ -- Example}}
\begin{lstlisting}
class NetworkGraph: public smoc_graph {
protected:
  // actors
  Forward         ping;
  Forward         pong;
public:
  // network graph constructor
  NetworkGraph(sc_module_name name)
    : smoc_graph(name),
      // create actors
      ping("Ping"),
      pong("Pong")
  {
    smoc_fifo<int> initFifo(1);
    initFifo << 42;
    connectNodePorts(ping.out, pong.in);
    connectNodePorts(pong.out, ping.in, initFifo);
  }
};
\end{lstlisting}
\end{frame}





%%%%%%%%%%%%%%%%%%%%%%%%%%%%%%%%%%%%%%%%%%%%%%%%%%%%%%%%%%%%%%%%%%%%%%%%%%%%%%
%%%%%%%%%%%%%%%%%%%%%%%%%%%%%%%%%%%%%%%%%%%%%%%%%%%%%%%%%%%%%%%%%%%%%%%%%%%%%%
\begin{frame}[fragile=singleslide]
\mode<presentation>{\frametitle{\insertsubsection\ -- Simulation}}
\begin{lstlisting}
             SystemC 2.2.0 --- Dec 15 2008 11:10:07
        Copyright (c) 1996-2006 by all Contributors
                    ALL RIGHTS RESERVED
top.Ping forward: "42"
top.Pong forward: "42"
top.Ping forward: "42"
top.Pong forward: "42"
...
\end{lstlisting}
\end{frame}

%%%%%%%%%%%%%%%%%%%%%%%%%%%%%%%%%%%%%%%%%%%%%%%%%%%%%%%%%%%%%%%%%%%%%%%%%%%%%%
%%%%%%%%%%%%%%%%%%%%%%%%%%%%%%%%%%%%%%%%%%%%%%%%%%%%%%%%%%%%%%%%%%%%%%%%%%%%%%
\begin{frame}[fragile=singleslide]
\mode<presentation>{\frametitle{\insertsubsection\ -- Syntax}}
\begin{itemize}
\item explicit creation of a FIFO initializer TODO: name it QUEUE??
\item give queue size as constructor parameter
\begin{lstlisting}
    smoc_fifo<int> initFifo(1);
\end{lstlisting}
\item push initial value to initializer
\begin{lstlisting}
    initFifo << 42;
\end{lstlisting}
\item pass initializer when constructing the queue
\begin{lstlisting}
    connectNodePorts(pong.out, ping.in, initFifo);
\end{lstlisting}
\item initializer may be reused for creating identical initialized queues
\end{itemize}
\end{frame}




\section{Advanced Data Flow Modeling}
\subsection{Hierarchical States}
%%%%%%%%%%%%%%%%%%%%%%%%%%%%%%%%%%%%%%%%%%%%%%%%%%%%%%%%%%%%%%%%%%%%%%%%%%%%%%
%%%%%%%%%%%%%%%%%%%%%%%%%%%%%%%%%%%%%%%%%%%%%%%%%%%%%%%%%%%%%%%%%%%%%%%%%%%%%%
\begin{frame}
\mode<presentation>{\frametitle{\insertsubsection\ -- Objectives}}
\begin{itemize}
\item You will learn how to refine states of an FSM into hierarchical states by means of ...
\item ... XOR decomposition,
\item ... AND decomposition, and
\item ... Junction states
\end{itemize}
\end{frame}


%%%%%%%%%%%%%%%%%%%%%%%%%%%%%%%%%%%%%%%%%%%%%%%%%%%%%%%%%%%%%%%%%%%%%%%%%%%%%%
%%%%%%%%%%%%%%%%%%%%%%%%%%%%%%%%%%%%%%%%%%%%%%%%%%%%%%%%%%%%%%%%%%%%%%%%%%%%%%
\begin{frame}[fragile=singleslide]
\mode<presentation>{\frametitle{Non-Hierarchical Example}}
\begin{lstlisting}
#include <iostream>
#include <systemoc/smoc_moc.hpp>

class Actor : public smoc_actor {
public:
  enum CmdType { CMD_GO, CMD_STOP };
  smoc_port_in<CmdType> inCtl;

  Actor(sc_module_name name)
    : smoc_actor(name, stopped) {
    smoc_firing_state run; // states can be declared locally

    stopped =
         inCtl(1) && GUARD(Actor::isCmd)(CMD_GO) >>
         CALL(Actor::go) >> run;
    run =
        inCtl(1) && GUARD(Actor::isCmd)(CMD_STOP) >>
        CALL(Actor::stop) >> stopped;
  }
  ...
\end{lstlisting}
\end{frame}

%%%%%%%%%%%%%%%%%%%%%%%%%%%%%%%%%%%%%%%%%%%%%%%%%%%%%%%%%%%%%%%%%%%%%%%%%%%%%%
%%%%%%%%%%%%%%%%%%%%%%%%%%%%%%%%%%%%%%%%%%%%%%%%%%%%%%%%%%%%%%%%%%%%%%%%%%%%%%
\begin{frame}[fragile=singleslide]
\mode<presentation>{\frametitle{Non-Hierarchical Example}}
\begin{lstlisting}
  ...
private:
  smoc_firing_state stopped;

  bool isCmd(CmdType cmd) const
    { return inCtl[0] == cmd; }
  void go()
    { std::cout << this->name() << ": Go" << std::endl; }
  void stop()
    { std::cout << this->name() << ": Stop" << std::endl; }
};
\end{lstlisting}
\begin{itemize}
\item This initial example contains two non-hierarchical states \texttt{stopped} and \texttt{run} as known from previous examples
\item Note that states (except the initial state) can be declared locally inside the constructor
\item In the following, we will refine the \texttt{run} state into a nested FSM by means of hierarchical states
\end{itemize}
\end{frame}


%%%%%%%%%%%%%%%%%%%%%%%%%%%%%%%%%%%%%%%%%%%%%%%%%%%%%%%%%%%%%%%%%%%%%%%%%%%%%%
%%%%%%%%%%%%%%%%%%%%%%%%%%%%%%%%%%%%%%%%%%%%%%%%%%%%%%%%%%%%%%%%%%%%%%%%%%%%%%
\begin{frame}[fragile=singleslide]
\mode<presentation>{\frametitle{\insertsubsection\ -- XOR decomposition}}
\begin{itemize}
\item First step: Replace the type of a state by the desired hierarchical state type, namely \texttt{smoc\_xor\_state} or \texttt{smoc\_and\_state}.  In this example, we will refine the \texttt{run} state into an XOR state:
\begin{lstlisting}
  Actor(...) : smoc_actor(..., stopped) {
    smoc_xor_state run;
    ...
  }
\end{lstlisting}
\item Second step: Instantiate the child states of the XOR state. Note that these can consist of hierarchical and non-hierarchical states:
\begin{lstlisting}
    ...    
    smoc_xor_state    run;
    smoc_firing_state waitMsg;
    smoc_and_state    storeData; // will be refined later
    smoc_firing_state sendAck;
    ...
\end{lstlisting}
\end{itemize}
\end{frame}

%%%%%%%%%%%%%%%%%%%%%%%%%%%%%%%%%%%%%%%%%%%%%%%%%%%%%%%%%%%%%%%%%%%%%%%%%%%%%%
%%%%%%%%%%%%%%%%%%%%%%%%%%%%%%%%%%%%%%%%%%%%%%%%%%%%%%%%%%%%%%%%%%%%%%%%%%%%%%
\begin{frame}[fragile=singleslide]
\mode<presentation>{\frametitle{\insertsubsection\ -- XOR decomposition}}
\begin{itemize}
\item Third step: Add the child states to the XOR state. Note that the initial child state, i.e., the state which should be active when the XOR state is entered, must be added via \texttt{parent.init(child)}. All other states are added via \texttt{parent.add(child)}:
\begin{lstlisting}
  smoc_xor_state    run;
  smoc_firing_state waitMsg; // initial child state
  smoc_and_state    storeData;
  smoc_firing_state sendAck;

  run.init(waitMsg);
  run.add(storeData);
  run.add(sendAck);
\end{lstlisting}
\item Last step: Add transitions to the child states:
\begin{lstlisting}
  waitMsg = inLink(1) >> CALL(Actor::processMsg)
    >> storeData;
  sendAck = outLink(1) >> CALL(Actor::sendAckMsg)
    >> waitMsg;
\end{lstlisting}
\end{itemize}
\end{frame}

%%%%%%%%%%%%%%%%%%%%%%%%%%%%%%%%%%%%%%%%%%%%%%%%%%%%%%%%%%%%%%%%%%%%%%%%%%%%%%
%%%%%%%%%%%%%%%%%%%%%%%%%%%%%%%%%%%%%%%%%%%%%%%%%%%%%%%%%%%%%%%%%%%%%%%%%%%%%%
\begin{frame}[fragile=singleslide]
\mode<presentation>{\frametitle{\insertsubsection\ -- XOR decomposition}}
\begin{itemize}
\item XOR states must have exactly one initial state
\item States which are not added to any hierarchical state are automatically added to the FSM (like \texttt{stopped} and \texttt{run} in this example), which can also be seen as an XOR state (with \texttt{stopped} being the initial state in this example)
\item Transitions added to the XOR state itself are added (recursively) to all child states. In this case, the only outgoing transition of the \texttt{run} state allows for leaving the receive/send loop at any time: 
\begin{lstlisting}
  run =
      inCtl(1) && GUARD(Actor::isCmd)(CMD_STOP) >>
      CALL(Actor::stop) >> stopped;
\end{lstlisting}
\item History connectors are not supported at this time
\end{itemize}
\end{frame}

%%%%%%%%%%%%%%%%%%%%%%%%%%%%%%%%%%%%%%%%%%%%%%%%%%%%%%%%%%%%%%%%%%%%%%%%%%%%%%
%%%%%%%%%%%%%%%%%%%%%%%%%%%%%%%%%%%%%%%%%%%%%%%%%%%%%%%%%%%%%%%%%%%%%%%%%%%%%%
\begin{frame}[fragile=singleslide]
\mode<presentation>{\frametitle{\insertsubsection\ -- AND decomposition}}
\begin{itemize}
\item An XOR state has exactly one active child state at any one time
\item In an AND state, all child states are active at any one time. In the following, the child states of an AND state will be called \emph{partitions}
\item A \emph{partition} can be a hierarchical or non-hierarchical state
\item The state of an AND state with $N$ partitions $P_1,\ldots,P_N$ is given by the tuple $(s_1,s_2,\ldots,s_N)$, such that $\forall i, 1 \leq i \leq N: s_i $ is a valid state from partition $P_i$. This is also called a \emph{product state}
\item The initial state of an AND state is the product state whose components are in turn the initial states of the corresponding partitions
\item In \SysteMoC, an AND state can be declared as follows: 
\begin{lstlisting}
  smoc_and_state storeData;
\end{lstlisting}
\end{itemize}
\end{frame}

%%%%%%%%%%%%%%%%%%%%%%%%%%%%%%%%%%%%%%%%%%%%%%%%%%%%%%%%%%%%%%%%%%%%%%%%%%%%%%
%%%%%%%%%%%%%%%%%%%%%%%%%%%%%%%%%%%%%%%%%%%%%%%%%%%%%%%%%%%%%%%%%%%%%%%%%%%%%%
\begin{frame}[fragile=singleslide]
\mode<presentation>{\frametitle{\insertsubsection\ -- AND decomposition}}
\begin{itemize}
\item Partitions are added to AND states in the same manner as child states are added to XOR states:
\begin{lstlisting}
  smoc_and_state storeData;
  smoc_xor_state sendDmaWriteReqs;
  smoc_xor_state recvDmaWriteAcks;
  // add two partitions to storeData:
  storeData.add(sendDmaWriteReqs)
           .add(recvDmaWriteAcks);

  // add child states to the first partition:
  smoc_firing_state sendDmaWriteReq; 
  smoc_firing_state sentAllDmaWriteReqs; 
  sendDmaWriteReqs.init(sendDmaWriteReq)
                  .add(sentAllDmaWriteReqs);

  // add child states to the second partition:
  smoc_firing_state recvDmaWriteAck; 
  smoc_firing_state recvdAllDmaWriteAcks; 
  recvDmaWriteAcks.init(recvDmaWriteAck)
                  .add(recvdAllDmaWriteAcks);
\end{lstlisting}
\end{itemize}
\end{frame}

%%%%%%%%%%%%%%%%%%%%%%%%%%%%%%%%%%%%%%%%%%%%%%%%%%%%%%%%%%%%%%%%%%%%%%%%%%%%%%
%%%%%%%%%%%%%%%%%%%%%%%%%%%%%%%%%%%%%%%%%%%%%%%%%%%%%%%%%%%%%%%%%%%%%%%%%%%%%%
\begin{frame}[fragile=singleslide]
\mode<presentation>{\frametitle{\insertsubsection\ -- AND decomposition}}
\begin{itemize}
\item Transitions can be added to partitions as usual. Note that transitions between partitions are not allowed:
\begin{lstlisting}
  sendDmaWriteReq =
       outMem(1) && GUARD(Actor::sendMoreDmaWriteReqs)
    >> CALL(Actor::sendDmaWriteReq)
    >> sendDmaWriteReq
  |    !GUARD(Actor::sendMoreDmaWriteReqs)
    >> sentAllDmaWriteReqs;
  
  recvDmaWriteAck =
       inMem(1) && GUARD(Actor::isDmaWriteAck)
    >> CALL(Actor::processDmaWriteAck)
    >> recvDmaWriteAck
  |    !GUARD(Actor::pendingDmaWriteAcks)
    >> recvdAllDmaWriteAcks;

  // not allowed (will throw a ModelingException):
  // sendDmaWriteReq = ... >> recvDmaWriteAck;
\end{lstlisting}
\end{itemize}
\end{frame}

%%%%%%%%%%%%%%%%%%%%%%%%%%%%%%%%%%%%%%%%%%%%%%%%%%%%%%%%%%%%%%%%%%%%%%%%%%%%%%
%%%%%%%%%%%%%%%%%%%%%%%%%%%%%%%%%%%%%%%%%%%%%%%%%%%%%%%%%%%%%%%%%%%%%%%%%%%%%%
\begin{frame}[fragile=singleslide]
\mode<presentation>{\frametitle{\insertsubsection\ -- AND decomposition}}
\begin{itemize}
\item A transition whose target state is an AND state may specify for each partition which state should be active when entering the AND state:
\begin{lstlisting}
  waitMsg = inLink(1) >> CALL(Actor::processMsg)
    >> (sendDmaWriteReq, recvDmaWriteAck);
\end{lstlisting}
\item If no specific state is given for a partition, the default initial state will be the active state:
\begin{lstlisting}
  waitMsg = inLink(1) >> CALL(Actor::processMsg)
    >> storeData;
\end{lstlisting}
\item Both transitions are equivalent, as both \texttt{sendDmaWriteReq} and \texttt{recvDmaWriteAck} are the initial states
      of their corresponding partitions
\item In fact, the target state can be any valid (partial) product state
\end{itemize}
\end{frame}

%%%%%%%%%%%%%%%%%%%%%%%%%%%%%%%%%%%%%%%%%%%%%%%%%%%%%%%%%%%%%%%%%%%%%%%%%%%%%%
%%%%%%%%%%%%%%%%%%%%%%%%%%%%%%%%%%%%%%%%%%%%%%%%%%%%%%%%%%%%%%%%%%%%%%%%%%%%%%
\begin{frame}[fragile=singleslide]
\mode<presentation>{\frametitle{\insertsubsection\ -- AND decomposition}}
\begin{itemize}
\item It is also possible for the source state to be a (partial) product state. However, some restrictions apply
\item A transition must be appended to exactly one state. All other states in the (partial) product state
      must be marked with the \texttt{IN} keyword:
\begin{lstlisting}
  (s1, IN(s2), IN(s3), ...) = ...;
\end{lstlisting}
\item In this case, the transition from \texttt{s1}  will be enabled if the FSM is also in states \texttt{s2} and \texttt{s3}.
\item The \texttt{IN} keyword can also be negated:
\begin{lstlisting}
  (s1, IN(s2), !IN(s3), ...) = ...;
\end{lstlisting}
\item In this case, the transition from \texttt{s1}  will be enabled if the FSM is in state \texttt{s2} but \emph{not} in \texttt{s3}.
\end{itemize}
\end{frame}

%%%%%%%%%%%%%%%%%%%%%%%%%%%%%%%%%%%%%%%%%%%%%%%%%%%%%%%%%%%%%%%%%%%%%%%%%%%%%%
%%%%%%%%%%%%%%%%%%%%%%%%%%%%%%%%%%%%%%%%%%%%%%%%%%%%%%%%%%%%%%%%%%%%%%%%%%%%%%
\begin{frame}[fragile=singleslide]
\mode<presentation>{\frametitle{\insertsubsection\ -- AND decomposition}}
\begin{itemize}
\item Like an XOR state, an AND state will be left if the target state of a transition is no child state of the AND state...
\begin{lstlisting}
  (sentAllDmaWriteReqs, IN(recvdAllDmaWriteAcks)) = ... >> sendAck
\end{lstlisting}
\item ...or if the source state of a transition is the AND state itself:
\begin{lstlisting}
  (storeData, IN(sentAllDmaWriteReqs), IN(recvdAllDmaWriteAcks)) = ... >> sendAck
\end{lstlisting}
\end{itemize}
\end{frame}

%%%%%%%%%%%%%%%%%%%%%%%%%%%%%%%%%%%%%%%%%%%%%%%%%%%%%%%%%%%%%%%%%%%%%%%%%%%%%%
%%%%%%%%%%%%%%%%%%%%%%%%%%%%%%%%%%%%%%%%%%%%%%%%%%%%%%%%%%%%%%%%%%%%%%%%%%%%%%
\begin{frame}[fragile=singleslide]
\mode<presentation>{\frametitle{\insertsubsection\ -- Junction states}}
\begin{itemize}
\item Junction states as known from UML can be used to chain together multiple transitions
\item They are not part of the state hierarchy and need not be added to any states
\begin{lstlisting}
  smoc_firing_state a, b, c;
  smoc_junction_state j;
    
  a = CALL(Actor::actionA) >> j;
  b = CALL(Actor::actionB) >> j;
  j = CALL(Actor::actionC) >> c; 
\end{lstlisting}
\item This example is equivalent to the following FSM fragment without junction states:
\begin{lstlisting}
  smoc_firing_state a, b, c;
    
  a = CALL(Actor::actionA) >> CALL(Actor::actionC) >> c;
  b = CALL(Actor::actionB) >> CALL(Actor::actionC) >> c;
\end{lstlisting}
\end{itemize}
\end{frame}

%%%%%%%%%%%%%%%%%%%%%%%%%%%%%%%%%%%%%%%%%%%%%%%%%%%%%%%%%%%%%%%%%%%%%%%%%%%%%%
%%%%%%%%%%%%%%%%%%%%%%%%%%%%%%%%%%%%%%%%%%%%%%%%%%%%%%%%%%%%%%%%%%%%%%%%%%%%%%
\begin{frame}[fragile=singleslide]
\mode<presentation>{\frametitle{\insertsubsection\ -- Junction states}}
\begin{itemize}
\item Note that a transition can have multiple actions. In this case, the order of
      execution is equal to the order of appearance
\item All pairs $(t_{in},t_{out})$ of transitions where $t_{in}$ is a transition entering the junction state
      and $t_{out}$ is a transition leaving the junction state will be transformed into a single compound transition
\item Actions of transition $t_{out}$ will be executed prior to the actions of transition $t_{in}$
\item All guards of a compound transition are executed prior to any action of a compound transition
\item For example, if variables are modified by transition $t_{in}$, guards of transition $t_{out}$ 
      will see the \emph{old} values of these variables, and not the modified values
\end{itemize}
\end{frame}

\subsection{Dynamic FSM Construction}
%%%%%%%%%%%%%%%%%%%%%%%%%%%%%%%%%%%%%%%%%%%%%%%%%%%%%%%%%%%%%%%%%%%%%%%%%%%%%%
%%%%%%%%%%%%%%%%%%%%%%%%%%%%%%%%%%%%%%%%%%%%%%%%%%%%%%%%%%%%%%%%%%%%%%%%%%%%%%
\begin{frame}
\mode<presentation>{\frametitle{\insertsubsection\ -- Objectives}}
\begin{itemize}
\item You will learn how to dynamically construct the FSM of an actor...
\item ... by dynamically adding transitions to states and
\item ... by passing and returning states to and from functions
\end{itemize}
\end{frame}


%%%%%%%%%%%%%%%%%%%%%%%%%%%%%%%%%%%%%%%%%%%%%%%%%%%%%%%%%%%%%%%%%%%%%%%%%%%%%%
%%%%%%%%%%%%%%%%%%%%%%%%%%%%%%%%%%%%%%%%%%%%%%%%%%%%%%%%%%%%%%%%%%%%%%%%%%%%%%
\begin{frame}[fragile=singleslide]
\mode<presentation>{\frametitle{\insertsubsection}}
\begin{itemize}
\item Transitions can be added to states after some transitions have already been added:
\begin{lstlisting}
  smoc_firing_state a, b, c, d;
  
  // overwrite any transitions already added to a
  a =
      GUARD(Actor::guardA) >> CALL(Actor::actionA) >> a
    | GUARD(Actor::guardB) >> CALL(Actor::actionB) >> b;
  
  // append some more transitions to a
  a |= 
      GUARD(Actor::guardC) >> CALL(Actor::actionC) >> c
    | GUARD(Actor::guardD) >> CALL(Actor::actionD) >> d;  
\end{lstlisting}
\item Note that whereas \texttt{a = <transition list>} first removes all transitions from a and then
      adds \texttt{<transition list>} to a, \texttt{a |= <transition list>} simply appends \texttt{<transition list>}
      to a
\end{itemize}
\end{frame}

%%%%%%%%%%%%%%%%%%%%%%%%%%%%%%%%%%%%%%%%%%%%%%%%%%%%%%%%%%%%%%%%%%%%%%%%%%%%%%
%%%%%%%%%%%%%%%%%%%%%%%%%%%%%%%%%%%%%%%%%%%%%%%%%%%%%%%%%%%%%%%%%%%%%%%%%%%%%%
\begin{frame}[fragile=singleslide]
\mode<presentation>{\frametitle{\insertsubsection}}
\begin{itemize}
\item States can be passed to functions via pointers or references:
\begin{lstlisting}
  void a(smoc_firing_state& s);
  void b(smoc_firing_state::Ref s); // equivalent to a
  
  void c(const smoc_firing_state& s);
  void d(smoc_firing_state::ConstRef s); // equivalent to c

  void e(smoc_firing_state* a);
  void f(smoc_firing_state::Ptr a); // equivalent to e

  void g(const smoc_firing_state* a);
  void h(smoc_firing_state::ConstPtr a); // equivalent to g
\end{lstlisting}
\item Due to states being reference counted, using \texttt{::Ref}, \texttt{::ConstRef}, \texttt{::Ptr} and \texttt{::ConstPtr} is the preferred way
      of passing around references or pointers to states
\end{itemize}
\end{frame}

%%%%%%%%%%%%%%%%%%%%%%%%%%%%%%%%%%%%%%%%%%%%%%%%%%%%%%%%%%%%%%%%%%%%%%%%%%%%%%
%%%%%%%%%%%%%%%%%%%%%%%%%%%%%%%%%%%%%%%%%%%%%%%%%%%%%%%%%%%%%%%%%%%%%%%%%%%%%%
\begin{frame}[fragile=singleslide]
\mode<presentation>{\frametitle{\insertsubsection}}
\begin{itemize}
\item States can be returned from functions via pointers or references:
\begin{lstlisting}
  smoc_firing_state::Ptr a() {
    smoc_firing_state s;
    ...
    return &s;
  }
  
  smoc_firing_state::Ref b() {
    smoc_firing_state s;
    ...
    return s;
  }
\end{lstlisting}
\item Note that when using \texttt{::Ref}, \texttt{::ConstRef}, \texttt{::Ptr} or \texttt{::ConstPtr} as the return type of a function,
      returning a pointer or reference to a state declared locally inside the function is perfectly legal
\end{itemize}
\end{frame}


%% TODO: (martin) complete following sections
%% for a first release version we excluded sections in draft state
%
%\section{Advanced Data Flow Modeling}
%\subsection{Data Flow MoCs}
%%%%%%%%%%%%%%%%%%%%%%%%%%%%%%%%%%%%%%%%%%%%%%%%%%%%%%%%%%%%%%%%%%%%%%%%%%%%%%%
%%%%%%%%%%%%%%%%%%%%%%%%%%%%%%%%%%%%%%%%%%%%%%%%%%%%%%%%%%%%%%%%%%%%%%%%%%%%%%
\begin{frame}
\mode<presentation>{\frametitle{\insertsubsection\ }}
  ...
\end{frame}










%%%%%%%%%%%%%%%%%%%%%%%%%%%%%%%%%%%%%%%%%%%%%%%%%%%%%%%%%%%%%%%%%%%%%%%%%%%%%%
%%%%%%%%%%%%%%%%%%%%%%%%%%%%%%%%%%%%%%%%%%%%%%%%%%%%%%%%%%%%%%%%%%%%%%%%%%%%%%
\begin{frame}
\mode<presentation>{\frametitle{\insertsection\ -- Hierarchy}}
\begin{figure}
\centering
\resizebox{0.7\columnwidth}{!}{\input{mocs-fig.tex}}
\end{figure}
\begin{itemize}
\item BDF and larger: Turing complete
\item SysteMoC $\sim$ RPN
\end{itemize}
\end{frame}


%\subsection{Merging Streams}
%%%%%%%%%%%%%%%%%%%%%%%%%%%%%%%%%%%%%%%%%%%%%%%%%%%%%%%%%%%%%%%%%%%%%%%%%%%%%%%
%%%%%%%%%%%%%%%%%%%%%%%%%%%%%%%%%%%%%%%%%%%%%%%%%%%%%%%%%%%%%%%%%%%%%%%%%%%%%%
\begin{frame}
\mode<presentation>{\frametitle{\insertsubsection\ -- Objectives}}
\begin{itemize}
\item lets have a look on some traps with data flow modeling
\end{itemize}
\begin{itemize}
\item You will learn to ...
\item ... use data flow models.
\item ... beware or at least be aware of deadlocks.
\end{itemize}
\end{frame}







%%%%%%%%%%%%%%%%%%%%%%%%%%%%%%%%%%%%%%%%%%%%%%%%%%%%%%%%%%%%%%%%%%%%%%%%%%%%%%
%%%%%%%%%%%%%%%%%%%%%%%%%%%%%%%%%%%%%%%%%%%%%%%%%%%%%%%%%%%%%%%%%%%%%%%%%%%%%%
\begin{frame}
\mode<presentation>{\frametitle{\insertsubsection\ -- Alternating Merge}}
\begin{figure}
\centering
\resizebox{0.9\columnwidth}{!}{\input{merge-fig.tex}}
\end{figure}
\begin{itemize}
\item the example consists of ...
\item ... two source actors, each producing a sequence of tokens
\begin{itemize}
\item one actor sends ``ABCDEFGHIJKLMNOPQRSTUVWXYZ''
\item the other actor produces the sequence ``012345''
\end{itemize}
\item ... a merge actor forwarding input data in alternating order
\item ... and a source to print out received data.
\end{itemize}
\end{frame}





%%%%%%%%%%%%%%%%%%%%%%%%%%%%%%%%%%%%%%%%%%%%%%%%%%%%%%%%%%%%%%%%%%%%%%%%%%%%%%
%%%%%%%%%%%%%%%%%%%%%%%%%%%%%%%%%%%%%%%%%%%%%%%%%%%%%%%%%%%%%%%%%%%%%%%%%%%%%%
\begin{frame}[fragile=singleslide]
\mode<presentation>{\frametitle{\insertsubsection\ -- Alternating Merge}}
\begin{lstlisting}
static std::string MESSAGE_ABC = "ABCDEFGHIJKLMNOPQRSTUVWXYZ";
static std::string MESSAGE_123 = "012345";

class Source: public smoc_actor {
public:
  smoc_port_out<char> out;
  Source(sc_module_name n, std::string m): smoc_actor(n, start),
    count(0), size(m.size()), message(m) {
    start = 
      GUARD(Source::hasToken)  >>
      out(1)                   >>
      CALL(Source::src)        >> start;
  }
private:
  smoc_firing_state start;
  unsigned int count, size;
  std::string  message;

  bool hasToken() const{ return count<size; }
  void src() { out[0] = message[count++];  }
};
\end{lstlisting}
\end{frame}





%%%%%%%%%%%%%%%%%%%%%%%%%%%%%%%%%%%%%%%%%%%%%%%%%%%%%%%%%%%%%%%%%%%%%%%%%%%%%%
%%%%%%%%%%%%%%%%%%%%%%%%%%%%%%%%%%%%%%%%%%%%%%%%%%%%%%%%%%%%%%%%%%%%%%%%%%%%%%
\begin{frame}[fragile=singleslide]
\mode<presentation>{\frametitle{\insertsubsection\ -- Alternating Merge}}
\begin{lstlisting}
template<typename T>
class Alternate: public smoc_actor {
public:
  smoc_port_in<T> in0, in1;
  smoc_port_out<T> out;
private:
  void forward0() {out[0] = in0[0];}
  void forward1() {out[0] = in1[0];}

  smoc_firing_state one, zero;
public:
  Alternate(sc_module_name name)
    : smoc_actor(name, one) {
    one =
      in0(1)    >>   out(1)     >>
      CALL(Alternate::forward0) >> zero;
    zero =
      in1(1)    >>   out(1)     >>
      CALL(Alternate::forward1) >> one;
  }
};
\end{lstlisting}
\end{frame}





%%%%%%%%%%%%%%%%%%%%%%%%%%%%%%%%%%%%%%%%%%%%%%%%%%%%%%%%%%%%%%%%%%%%%%%%%%%%%%
%%%%%%%%%%%%%%%%%%%%%%%%%%%%%%%%%%%%%%%%%%%%%%%%%%%%%%%%%%%%%%%%%%%%%%%%%%%%%%
\begin{frame}[fragile=singleslide]
\mode<presentation>{\frametitle{\insertsubsection\ -- Alternating Merge}}
\begin{lstlisting}
class NetworkGraph: public smoc_graph {
protected:
  Source           source0;
  Source           source1;
  Alternate<char>  alternate;
  Sink             sink;
public:
  NetworkGraph(sc_module_name name)
    : smoc_graph(name),
      source0("Source0", MESSAGE_123),
      source1("Source1", MESSAGE_ABC),
      alternate("Alternate"),
      sink("Sink") {
    connectNodePorts(source0.out, alternate.in0);
    connectNodePorts(source1.out, alternate.in1);
    connectNodePorts(alternate.out, sink.in);
  }
};
\end{lstlisting}
\end{frame}





%%%%%%%%%%%%%%%%%%%%%%%%%%%%%%%%%%%%%%%%%%%%%%%%%%%%%%%%%%%%%%%%%%%%%%%%%%%%%%
%%%%%%%%%%%%%%%%%%%%%%%%%%%%%%%%%%%%%%%%%%%%%%%%%%%%%%%%%%%%%%%%%%%%%%%%%%%%%%
\begin{frame}[fragile=singleslide]
\mode<presentation>{\frametitle{\insertsubsection\ -- Alternating Merge}}
\begin{lstlisting}
             SystemC 2.2.0 --- Dec 15 2008 11:10:07
        Copyright (c) 1996-2006 by all Contributors
                    ALL RIGHTS RESERVED
top.Sink recv: "0"
top.Sink recv: "A"
top.Sink recv: "1"
top.Sink recv: "B"
top.Sink recv: "2"
top.Sink recv: "C"
top.Sink recv: "3"
top.Sink recv: "D"
top.Sink recv: "4"
top.Sink recv: "E"
top.Sink recv: "5"
top.Sink recv: "F"
SystemC: simulation stopped by user.
\end{lstlisting}
\begin{itemize}
\item OK, source receives digits and letters alternating
\item sink does not receive rest of letters (``GH...'') 
\item FSM of actor Alternate  is forced to receive a digit character
\end{itemize}
\end{frame}







%%%%%%%%%%%%%%%%%%%%%%%%%%%%%%%%%%%%%%%%%%%%%%%%%%%%%%%%%%%%%%%%%%%%%%%%%%%%%%
%%%%%%%%%%%%%%%%%%%%%%%%%%%%%%%%%%%%%%%%%%%%%%%%%%%%%%%%%%%%%%%%%%%%%%%%%%%%%%
\begin{frame}
\mode<presentation>{\frametitle{\insertsubsection\ -- Non-deterministic Merge}}
\begin{figure}
\centering
\resizebox{0.9\columnwidth}{!}{\input{fair-merge-fig.tex}}
\end{figure}
\begin{itemize}
\item the NDMerge actor has two independent transitions
\item each transaction forwards individual input port tokens to output
\item having tokens on both input ports enables both transitions
\item merging should not stall if one input port runs out of token
\end{itemize}
\end{frame}





%%%%%%%%%%%%%%%%%%%%%%%%%%%%%%%%%%%%%%%%%%%%%%%%%%%%%%%%%%%%%%%%%%%%%%%%%%%%%%
%%%%%%%%%%%%%%%%%%%%%%%%%%%%%%%%%%%%%%%%%%%%%%%%%%%%%%%%%%%%%%%%%%%%%%%%%%%%%%
\begin{frame}[fragile=singleslide]
\mode<presentation>{\frametitle{\insertsubsection\ -- Non-deterministic Merge}}
\begin{lstlisting}
template<typename T>
class NDMerge: public smoc_actor {
public:
  smoc_port_in<T> in0, in1;
  smoc_port_out<T> out;
private:
  void forward0() {out[0] = in0[0];}
  void forward1() {out[0] = in1[0];}

  smoc_firing_state merge;
public:
  NDMerge(sc_module_name name)
    : smoc_actor(name, merge) {
    merge =
      in0(1)    >>  out(1)    >>
      CALL(NDMerge::forward0) >> merge
    |
      in1(1)    >>  out(1)    >>
      CALL(NDMerge::forward1) >> merge;
  }
};
\end{lstlisting}
\end{frame}





%%%%%%%%%%%%%%%%%%%%%%%%%%%%%%%%%%%%%%%%%%%%%%%%%%%%%%%%%%%%%%%%%%%%%%%%%%%%%%
%%%%%%%%%%%%%%%%%%%%%%%%%%%%%%%%%%%%%%%%%%%%%%%%%%%%%%%%%%%%%%%%%%%%%%%%%%%%%%
\begin{frame}[fragile=singleslide]
\mode<presentation>{\frametitle{\insertsubsection\ -- Non-deterministic Merge}}
\begin{lstlisting}
             SystemC 2.2.0 --- Dec 15 2008 11:10:07
        Copyright (c) 1996-2006 by all Contributors
                    ALL RIGHTS RESERVED
top.Sink recv: "0"
top.Sink recv: "1"
top.Sink recv: "A"
top.Sink recv: "B"
top.Sink recv: "2"
top.Sink recv: "3"
top.Sink recv: "C"
top.Sink recv: "D"
top.Sink recv: "4"
top.Sink recv: "5"
top.Sink recv: "E"
top.Sink recv: "F"
top.Sink recv: "G"
top.Sink recv: "H"
...
top.Sink recv: "Y"
top.Sink recv: "Z"
SystemC: simulation stopped by user.
\end{lstlisting}
\end{frame}





%%%%%%%%%%%%%%%%%%%%%%%%%%%%%%%%%%%%%%%%%%%%%%%%%%%%%%%%%%%%%%%%%%%%%%%%%%%%%%
%%%%%%%%%%%%%%%%%%%%%%%%%%%%%%%%%%%%%%%%%%%%%%%%%%%%%%%%%%%%%%%%%%%%%%%%%%%%%%
\begin{frame}
\mode<presentation>{\frametitle{\insertsubsection\ -- Non-deterministic Merge}}
\begin{itemize}
\item both transitions are active (enough token/free space)
\item one transition is selected non-deterministically and is executed
\item the streams in the example may be executed in any order
\end{itemize}
\end{frame}





%\subsection{MJPEG-Example}
%%%%%%%%%%%%%%%%%%%%%%%%%%%%%%%%%%%%%%%%%%%%%%%%%%%%%%%%%%%%%%%%%%%%%%%%%%%%%%%
%%%%%%%%%%%%%%%%%%%%%%%%%%%%%%%%%%%%%%%%%%%%%%%%%%%%%%%%%%%%%%%%%%%%%%%%%%%%%%
\begin{frame}
\mode<presentation>{\frametitle{\insertsubsection\ }}
  ...
\end{frame}




%
%\section{Programming Guidelines}
%\subsection{How-to}
%%%%%%%%%%%%%%%%%%%%%%%%%%%%%%%%%%%%%%%%%%%%%%%%%%%%%%%%%%%%%%%%%%%%%%%%%%%%%%%
%%%%%%%%%%%%%%%%%%%%%%%%%%%%%%%%%%%%%%%%%%%%%%%%%%%%%%%%%%%%%%%%%%%%%%%%%%%%%%
\begin{frame}[fragile=singleslide]
\mode<presentation>{\frametitle{\insertsubsection\ -- Inline Guards}}
\begin{itemize}
\item simple guards may be inlined
\item in a previous example we used a guard function ...
\begin{lstlisting}
  bool hasToken() const{
    return count<size;
  }
\end{lstlisting}
\item ... and referred to the guard inside the FSM 
\begin{lstlisting}
       GUARD(Source::hasToken)  >>
\end{lstlisting}
\item instead we can in-line this simple guards
\begin{lstlisting}
      (VAR(count)<VAR(size))   >>
\end{lstlisting}
\item in a FSMs transition we can directly access values of variables using \lstinline!VAR(...)! the macro
\end{itemize}
\end{frame}





%%%%%%%%%%%%%%%%%%%%%%%%%%%%%%%%%%%%%%%%%%%%%%%%%%%%%%%%%%%%%%%%%%%%%%%%%%%%%%
%%%%%%%%%%%%%%%%%%%%%%%%%%%%%%%%%%%%%%%%%%%%%%%%%%%%%%%%%%%%%%%%%%%%%%%%%%%%%%
\begin{frame}[fragile=singleslide]
\mode<presentation>{\frametitle{\insertsubsection\ -- Use Guards for Control Flow}}
\begin{itemize}
\item model ``if/else'' like constructs
\item use \lstinline|GUARD(..)| and  \lstinline|!GUARD(..)| for different transitions
\item please note the ``\lstinline|!|''
\begin{lstlisting}
    state = 
      GUARD(Actor::testGuard)  >>  // case IF
      out(1)                   >>
      CALL(Actor::processA)    >> state
    |
      !GUARD(Actor::testGuard) >>  // case IF NOT
      out(1)                   >>
      CALL(Actor::processB)    >> state
      ;
\end{lstlisting}
\end{itemize}
\end{frame}





%%%%%%%%%%%%%%%%%%%%%%%%%%%%%%%%%%%%%%%%%%%%%%%%%%%%%%%%%%%%%%%%%%%%%%%%%%%%%%
%%%%%%%%%%%%%%%%%%%%%%%%%%%%%%%%%%%%%%%%%%%%%%%%%%%%%%%%%%%%%%%%%%%%%%%%%%%%%%
\begin{frame}[fragile=singleslide]
\mode<presentation>{\frametitle{\insertsubsection\ -- Use Guards for Control Flow}}
\begin{itemize}
\item to be precise, the previous example is not an ``if/else'' construct
\item it is a ``if/if not'' construct
\item you cannot model a explicit ``else'' transition
\item neither an exclusive default transition
\item instead you have to model an explicit ``if not'' transition (see ``else if'' constructs on next slide)
\item transition have no priority ...
\item ``if/else if/else'' constructs in software do have a priority
\end{itemize}
\end{frame}





%%%%%%%%%%%%%%%%%%%%%%%%%%%%%%%%%%%%%%%%%%%%%%%%%%%%%%%%%%%%%%%%%%%%%%%%%%%%%%
%%%%%%%%%%%%%%%%%%%%%%%%%%%%%%%%%%%%%%%%%%%%%%%%%%%%%%%%%%%%%%%%%%%%%%%%%%%%%%
\begin{frame}[fragile=singleslide]
\mode<presentation>{\frametitle{\insertsubsection\ -- Use Guards for Control Flow}}
\begin{itemize}
\item a similarly way for ``if/else if'' constructs
\end{itemize}
\begin{lstlisting}
 (GUARD(Source::testGuard1) && GUARD(Source::testGuard2))   >>
 ...
 (GUARD(Source::testGuard1) && !GUARD(Source::testGuard2))  >>
 ...
 (!GUARD(Source::testGuard1) && GUARD(Source::testGuard2))  >>
 ...
 (!GUARD(Source::testGuard1) && !GUARD(Source::testGuard2)) >>
 ...
\end{lstlisting}
\end{frame}











%%%%%%%%%%%%%%%%%%%%%%%%%%%%%%%%%%%%%%%%%%%%%%%%%%%%%%%%%%%%%%%%%%%%%%%%%%%%%%
%%%%%%%%%%%%%%%%%%%%%%%%%%%%%%%%%%%%%%%%%%%%%%%%%%%%%%%%%%%%%%%%%%%%%%%%%%%%%%
\begin{frame}[fragile=singleslide]
\mode<presentation>{\frametitle{\insertsubsection\ -- Compatibility}}
\begin{itemize}
\item the old syntax of network graph instantiation (using \lstinline!smoc_top_moc!)
\end{itemize}
\begin{lstlisting}
  smoc_top_moc<HelloNetworkGraph> top("top"); // old syntax
\end{lstlisting}
\begin{itemize}
\item has been replaced by the following (using \lstinline! smoc_scheduler_top!)
\end{itemize}
\begin{lstlisting}
  HelloNetworkGraph top("top"); // new syntax
  smoc_scheduler_top sched(top);
\end{lstlisting}
\end{frame}





%%%%%%%%%%%%%%%%%%%%%%%%%%%%%%%%%%%%%%%%%%%%%%%%%%%%%%%%%%%%%%%%%%%%%%%%%%%%%%
%%%%%%%%%%%%%%%%%%%%%%%%%%%%%%%%%%%%%%%%%%%%%%%%%%%%%%%%%%%%%%%%%%%%%%%%%%%%%%
\begin{frame}
\mode<presentation>{\frametitle{\insertsubsection\ -- TODOs}}
\begin{itemize}
\item nesting graphs
\item dynamic instantiation (actor, channel, FSM)

\end{itemize}
\end{frame}



%\subsection{Naming Conventions}
%%%%%%%%%%%%%%%%%%%%%%%%%%%%%%%%%%%%%%%%%%%%%%%%%%%%%%%%%%%%%%%%%%%%%%%%%%%%%%%
%%%%%%%%%%%%%%%%%%%%%%%%%%%%%%%%%%%%%%%%%%%%%%%%%%%%%%%%%%%%%%%%%%%%%%%%%%%%%%
\begin{frame}
\mode<presentation>{\frametitle{\insertsubsection\ }}
\begin{itemize}
\item TODO
\end{itemize}
\end{frame}





%%%%%%%%%%%%%%%%%%%%%%%%%%%%%%%%%%%%%%%%%%%%%%%%%%%%%%%%%%%%%%%%%%%%%%%%%%%%%%
% Section Remarks
%%%%%%%%%%%%%%%%%%%%%%%%%%%%%%%%%%%%%%%%%%%%%%%%%%%%%%%%%%%%%%%%%%%%%%%%%%%%%%
\section{Remarks}
\begin{frame}
  \frametitle{Outline}
  \tableofcontents[currentsection,hideallsubsections]
\end{frame}


%%%%%%%%%%%%%%%%%%%%%%%%%%%%%%%%%%%%%%%%%%%%%%%%%%%%%%%%%%%%%%%%%%%%%%%%%%%%%%
% Contact
%%%%%%%%%%%%%%%%%%%%%%%%%%%%%%%%%%%%%%%%%%%%%%%%%%%%%%%%%%%%%%%%%%%%%%%%%%%%%%
\begin{frame}
\mode<presentation>{\frametitle{Contact}}
\begin{description}[\breaklabel\setleftmargin{60pt}\setlabelstyle{\color{beamer@SystemCoDesigner@color}}]
\item[Contact Persons:]
Christian Haubelt, Jürgen Teich
\item[Email:] codesigner@mycodesign.com
\item[Address:]
Hardware/Software Co-Design\\
Department of Computer Science\\
University of Erlangen-Nuremberg\\
Am Weichselgarten 3\\
91058 Erlangen, Germany
%\item[History:]
%Version 1.0:  2009/10/01
\end{description}
\end{frame}




%%%%%%%%%%%%%%%%%%%%%%%%%%%%%%%%%%%%%%%%%%%%%%%%%%%%%%%%%%%%%%%%%%%%%%%%%%%%%%
% Credits
%%%%%%%%%%%%%%%%%%%%%%%%%%%%%%%%%%%%%%%%%%%%%%%%%%%%%%%%%%%%%%%%%%%%%%%%%%%%%%
\begin{frame}
\mode<presentation>{\frametitle{Credits}}
\begin{description}[\breaklabel\setleftmargin{60pt}\setlabelstyle{\color{beamer@SystemCoDesigner@color}}]
\item[SysteMoC Development Team:]
Joachim Falk, Jens Gladigau, Martin Streubühr, Christian Zebelein
\item[SystemCoDesigner Contributors:]
Joachim Falk, Michael Glass, Jens Gladigau, Joachim Keinert, Martin Lukasiewycz, Felix Reimann, Thomas Schlichter, Thilo Streichert, Martin Streubühr, Christian Zebelein, Christian Haubelt, Jürgen Teich
\end{description}
\end{frame}


%%%%%%%%%%%%%%%%%%%%%%%%%%%%%%%%%%%%%%%%%%%%%%%%%%%%%%%%%%%%%%%%%%%%%%%%%%%%%%
% Contact
%%%%%%%%%%%%%%%%%%%%%%%%%%%%%%%%%%%%%%%%%%%%%%%%%%%%%%%%%%%%%%%%%%%%%%%%%%%%%%
\begin{frame}
\mode<presentation>{\frametitle{Document Info}}
\begin{description}[\breaklabel\setleftmargin{60pt}\setlabelstyle{\color{beamer@SystemCoDesigner@color}}]
\item[Authors:]
Martin Streubühr, Christian Zebelein, Christian Haubelt
\item[Document Release:]
Jul. 28, 2010
\item[Version History:]
Jul. 28, 2010: Version 1.1\\
Oct.1, 2009: Version 1.0
\end{description}
\end{frame}




%%%%%%%%%%%%%%%%%%%%%%%%%%%%%%%%%%%%%%%%%%%%%%%%%%%%%%%%%%%%%%%%%%%%%%%%%%%%%%
% Index
%%%%%%%%%%%%%%%%%%%%%%%%%%%%%%%%%%%%%%%%%%%%%%%%%%%%%%%%%%%%%%%%%%%%%%%%%%%%%%
\begin{frame}
\mode<presentation>{\frametitle{Index}}
{\footnotesize
\printindex
}
\end{frame}


\end{document}

