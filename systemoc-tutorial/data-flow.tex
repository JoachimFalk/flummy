%%%%%%%%%%%%%%%%%%%%%%%%%%%%%%%%%%%%%%%%%%%%%%%%%%%%%%%%%%%%%%%%%%%%%%%%%%%%%%
%%%%%%%%%%%%%%%%%%%%%%%%%%%%%%%%%%%%%%%%%%%%%%%%%%%%%%%%%%%%%%%%%%%%%%%%%%%%%%
\begin{frame}
\mode<presentation>{\frametitle{\insertsubsection\ -- Objectives}}
\begin{itemize}
\item lets have a look on some traps with data flow modeling
\end{itemize}
\begin{itemize}
\item You will learn to ...
\item ... use data flow models.
\item ... beware or at least be aware of deadlocks.
\end{itemize}
\end{frame}







%%%%%%%%%%%%%%%%%%%%%%%%%%%%%%%%%%%%%%%%%%%%%%%%%%%%%%%%%%%%%%%%%%%%%%%%%%%%%%
%%%%%%%%%%%%%%%%%%%%%%%%%%%%%%%%%%%%%%%%%%%%%%%%%%%%%%%%%%%%%%%%%%%%%%%%%%%%%%
\begin{frame}
\mode<presentation>{\frametitle{\insertsubsection\ -- Alternating Merge}}
\begin{figure}
\centering
\resizebox{0.9\columnwidth}{!}{\input{merge-fig.tex}}
\end{figure}
\begin{itemize}
\item the example consists of ...
\item ... two source actors, each producing a sequence of tokens
\begin{itemize}
\item one actor sends ``ABCDEFGHIJKLMNOPQRSTUVWXYZ''
\item the other actor produces the sequence ``012345''
\end{itemize}
\item ... a merge actor forwarding input data in alternating order
\item ... and a source to print out received data.
\end{itemize}
\end{frame}





%%%%%%%%%%%%%%%%%%%%%%%%%%%%%%%%%%%%%%%%%%%%%%%%%%%%%%%%%%%%%%%%%%%%%%%%%%%%%%
%%%%%%%%%%%%%%%%%%%%%%%%%%%%%%%%%%%%%%%%%%%%%%%%%%%%%%%%%%%%%%%%%%%%%%%%%%%%%%
\begin{frame}[fragile=singleslide]
\mode<presentation>{\frametitle{\insertsubsection\ -- Alternating Merge}}
\begin{lstlisting}
static std::string MESSAGE_ABC = "ABCDEFGHIJKLMNOPQRSTUVWXYZ";
static std::string MESSAGE_123 = "012345";

class Source: public smoc_actor {
public:
  smoc_port_out<char> out;
  Source(sc_module_name n, std::string m): smoc_actor(n, start),
    count(0), size(m.size()), message(m) {
    start = 
      GUARD(Source::hasToken)  >>
      out(1)                   >>
      CALL(Source::src)        >> start;
  }
private:
  smoc_firing_state start;
  unsigned int count, size;
  std::string  message;

  bool hasToken() const{ return count<size; }
  void src() { out[0] = message[count++];  }
};
\end{lstlisting}
\end{frame}





%%%%%%%%%%%%%%%%%%%%%%%%%%%%%%%%%%%%%%%%%%%%%%%%%%%%%%%%%%%%%%%%%%%%%%%%%%%%%%
%%%%%%%%%%%%%%%%%%%%%%%%%%%%%%%%%%%%%%%%%%%%%%%%%%%%%%%%%%%%%%%%%%%%%%%%%%%%%%
\begin{frame}[fragile=singleslide]
\mode<presentation>{\frametitle{\insertsubsection\ -- Alternating Merge}}
\begin{lstlisting}
template<typename T>
class Alternate: public smoc_actor {
public:
  smoc_port_in<T> in0, in1;
  smoc_port_out<T> out;
private:
  void forward0() {out[0] = in0[0];}
  void forward1() {out[0] = in1[0];}

  smoc_firing_state one, zero;
public:
  Alternate(sc_module_name name)
    : smoc_actor(name, one) {
    one =
      in0(1)    >>   out(1)     >>
      CALL(Alternate::forward0) >> zero;
    zero =
      in1(1)    >>   out(1)     >>
      CALL(Alternate::forward1) >> one;
  }
};
\end{lstlisting}
\end{frame}





%%%%%%%%%%%%%%%%%%%%%%%%%%%%%%%%%%%%%%%%%%%%%%%%%%%%%%%%%%%%%%%%%%%%%%%%%%%%%%
%%%%%%%%%%%%%%%%%%%%%%%%%%%%%%%%%%%%%%%%%%%%%%%%%%%%%%%%%%%%%%%%%%%%%%%%%%%%%%
\begin{frame}[fragile=singleslide]
\mode<presentation>{\frametitle{\insertsubsection\ -- Alternating Merge}}
\begin{lstlisting}
class NetworkGraph: public smoc_graph {
protected:
  Source           source0;
  Source           source1;
  Alternate<char>  alternate;
  Sink             sink;
public:
  NetworkGraph(sc_module_name name)
    : smoc_graph(name),
      source0("Source0", MESSAGE_123),
      source1("Source1", MESSAGE_ABC),
      alternate("Alternate"),
      sink("Sink") {
    connectNodePorts(source0.out, alternate.in0);
    connectNodePorts(source1.out, alternate.in1);
    connectNodePorts(alternate.out, sink.in);
  }
};
\end{lstlisting}
\end{frame}





%%%%%%%%%%%%%%%%%%%%%%%%%%%%%%%%%%%%%%%%%%%%%%%%%%%%%%%%%%%%%%%%%%%%%%%%%%%%%%
%%%%%%%%%%%%%%%%%%%%%%%%%%%%%%%%%%%%%%%%%%%%%%%%%%%%%%%%%%%%%%%%%%%%%%%%%%%%%%
\begin{frame}[fragile=singleslide]
\mode<presentation>{\frametitle{\insertsubsection\ -- Alternating Merge}}
\begin{lstlisting}
             SystemC 2.2.0 --- Dec 15 2008 11:10:07
        Copyright (c) 1996-2006 by all Contributors
                    ALL RIGHTS RESERVED
top.Sink recv: "0"
top.Sink recv: "A"
top.Sink recv: "1"
top.Sink recv: "B"
top.Sink recv: "2"
top.Sink recv: "C"
top.Sink recv: "3"
top.Sink recv: "D"
top.Sink recv: "4"
top.Sink recv: "E"
top.Sink recv: "5"
top.Sink recv: "F"
SystemC: simulation stopped by user.
\end{lstlisting}
\begin{itemize}
\item OK, source receives digits and letters alternating
\item sink does not receive rest of letters (``GH...'') 
\item FSM of actor Alternate  is forced to receive a digit character
\end{itemize}
\end{frame}







%%%%%%%%%%%%%%%%%%%%%%%%%%%%%%%%%%%%%%%%%%%%%%%%%%%%%%%%%%%%%%%%%%%%%%%%%%%%%%
%%%%%%%%%%%%%%%%%%%%%%%%%%%%%%%%%%%%%%%%%%%%%%%%%%%%%%%%%%%%%%%%%%%%%%%%%%%%%%
\begin{frame}
\mode<presentation>{\frametitle{\insertsubsection\ -- Non-deterministic Merge}}
\begin{figure}
\centering
\resizebox{0.9\columnwidth}{!}{\input{fair-merge-fig.tex}}
\end{figure}
\begin{itemize}
\item the NDMerge actor has two independent transitions
\item each transaction forwards individual input port tokens to output
\item having tokens on both input ports enables both transitions
\item merging should not stall if one input port runs out of token
\end{itemize}
\end{frame}





%%%%%%%%%%%%%%%%%%%%%%%%%%%%%%%%%%%%%%%%%%%%%%%%%%%%%%%%%%%%%%%%%%%%%%%%%%%%%%
%%%%%%%%%%%%%%%%%%%%%%%%%%%%%%%%%%%%%%%%%%%%%%%%%%%%%%%%%%%%%%%%%%%%%%%%%%%%%%
\begin{frame}[fragile=singleslide]
\mode<presentation>{\frametitle{\insertsubsection\ -- Non-deterministic Merge}}
\begin{lstlisting}
template<typename T>
class NDMerge: public smoc_actor {
public:
  smoc_port_in<T> in0, in1;
  smoc_port_out<T> out;
private:
  void forward0() {out[0] = in0[0];}
  void forward1() {out[0] = in1[0];}

  smoc_firing_state merge;
public:
  NDMerge(sc_module_name name)
    : smoc_actor(name, merge) {
    merge =
      in0(1)    >>  out(1)    >>
      CALL(NDMerge::forward0) >> merge
    |
      in1(1)    >>  out(1)    >>
      CALL(NDMerge::forward1) >> merge;
  }
};
\end{lstlisting}
\end{frame}





%%%%%%%%%%%%%%%%%%%%%%%%%%%%%%%%%%%%%%%%%%%%%%%%%%%%%%%%%%%%%%%%%%%%%%%%%%%%%%
%%%%%%%%%%%%%%%%%%%%%%%%%%%%%%%%%%%%%%%%%%%%%%%%%%%%%%%%%%%%%%%%%%%%%%%%%%%%%%
\begin{frame}[fragile=singleslide]
\mode<presentation>{\frametitle{\insertsubsection\ -- Non-deterministic Merge}}
\begin{lstlisting}
             SystemC 2.2.0 --- Dec 15 2008 11:10:07
        Copyright (c) 1996-2006 by all Contributors
                    ALL RIGHTS RESERVED
top.Sink recv: "0"
top.Sink recv: "1"
top.Sink recv: "A"
top.Sink recv: "B"
top.Sink recv: "2"
top.Sink recv: "3"
top.Sink recv: "C"
top.Sink recv: "D"
top.Sink recv: "4"
top.Sink recv: "5"
top.Sink recv: "E"
top.Sink recv: "F"
top.Sink recv: "G"
top.Sink recv: "H"
...
top.Sink recv: "Y"
top.Sink recv: "Z"
SystemC: simulation stopped by user.
\end{lstlisting}
\end{frame}





%%%%%%%%%%%%%%%%%%%%%%%%%%%%%%%%%%%%%%%%%%%%%%%%%%%%%%%%%%%%%%%%%%%%%%%%%%%%%%
%%%%%%%%%%%%%%%%%%%%%%%%%%%%%%%%%%%%%%%%%%%%%%%%%%%%%%%%%%%%%%%%%%%%%%%%%%%%%%
\begin{frame}
\mode<presentation>{\frametitle{\insertsubsection\ -- Non-deterministic Merge}}
\begin{itemize}
\item both transitions are active (enough token/free space)
\item one transition is selected non-deterministically and is executed
\item the streams in the example may be executed in any order
\end{itemize}
\end{frame}




