\section{SysteMoC semantic}

\subsection{Actor behavior}

An actor $A$ can be thought of as a relation $\mathbb{A}$ which maps sequences of token values $V^{**}$ on its input ports $A.I$ into sequences of token values on its output ports $A.O$, i.e., $\mathbb{A} \subseteq (V^{**})^{|A.I|} \times (V^{**})^{|A.O|}$.

\begin{eqnarray}
\mathrm{Fire}(v, q_\mathrm{firing}, q_\mathrm{func}) = \{ w_\mathrm{prod}.w_\mathrm{tail} %
 & |      & (q_\mathrm{firing}, k, f_\mathrm{action}, q'_\mathrm{firing}) \in \mathcal{T} \\
 & \wedge & k(v_\mathrm{cons}) \\
 & \wedge & \forall{v_\mathrm{prefix} \sqsubset v_\mathrm{cons}}: \neg k(v_\mathrm{prefix}) \\
 & \wedge & (w_\mathrm{prod}, q'_\mathrm{func}) = f_\mathrm{action}(v_\mathrm{cons},q_\mathrm{func}) \\
 & \wedge & v = v_\mathrm{cons}.v_\mathrm{tail} \\
 & \wedge & w_\mathrm{tail} = \mathrm{Fire}(v_\mathrm{tail}, q'_\mathrm{firing}, q'_\mathrm{func}) \}
\end{eqnarray}

$\mathbb{A} = \{ (v,w) | w \in \mathrm{Fire}(v, q_{_0\mathrm{firing}}, q_{_0\mathrm{func}}), v \in (V^{**})^{|A.I|} \}$

% \section{Kopplung zwischen SysteMoC und VPC}
% \label{sec:smoc-vpc}
% Die SysteMoC-Bibliothek erm�glicht es, einzelne Aktoren in Form von Zustandsmaschinen, die Funktionen aufrufen, zu beschreiben.
% Eine Anwendung, z.B. der InfiniBand-HCA besteht in der Regel aus einem Netzwerk solcher Aktoren.
% Ein solches Netzwerk kann durch Simulation bez�glich der Funktionalit�t untersucht werden.
% Das \emph{Virtual Processing Components} Framework~\cite{streubuehr:2005} erm�glicht die zus�tzliche Simulation des Zeitverhaltens unter Ressourcenbeschr�nkung.
% Hierf�r wird f�r jede Aktivierung eines Aktors eine dynamisch ermittelte Ausf�hrungszeit abgewartet.
% Diese Ausf�hrungszeit wird durch die Bindung an eine Ressource, sowie durch eine Schedulingstrategie bestimmt.
% Einzig notwendig hierf�r sind zwei Funktionsaufrufe (\code{getResource}, \code{compute}) aus der Anwendung heraus (vgl.~\cite{streubuehr:2005, fghsst:2005}).
% \par
% Zusammen mit einer Aktorbibliothek, wie der SysteMoC, k�nnen diese Aufrufe vollst�ndig transparent f�r den Nutzer eingef�gt werden.
% %Zusammen mit der SysteMoC-Bibliothek gestaltet sich die Kopplung wie folgt:
% Jedes mal wenn in einem Aktor eine Transition ausgef�hrt wird, entspricht dieses einer Aktivit�t.
% Bevor diese Funktionalit�t ausgef�hrt wird, werden die Funktionsaufrufe \code{getResource} und \code{compute} ausgef�hrt und somit die simulierte Zeit verbraucht.
% %\textbf{TODO: globales scheduling bzw. dessen Effekte erw�hnen??}
% \par
% Weiterhin zeichnen sich die SysteMoC Aktoren durch ihre funktionsakkurate Aus\-f�hr\-ungs\-sem\-an\-tik aus:
% Ein Aktor besitzt ausdr�cklich modellierte Zust�nde und Transitionen, wobei Transitionen die Aktivit�ten darstellen und Zustandswechsel verursachen.
% Das VPC wurde dahingehend erweitert, dass die Ausf�hrungszeit auch in Abh�ngigkeit zur ausgef�hrten Transition steht.
% Dementsprechend wurde die \code{compute}-Funktion um einen weiteren Parameter f�r die auszuf�hrende Transition erg�nzt.
% W�hrend die Bindung an Ressourcen weiterhin aktorakkurat stattfindet, wird die Ausf�hrungszeit funktionsakkurat modelliert.
% Wie schon bei der vorangegangenen Studie wird das VPC durch eine Konfiguration parametrisiert, und erm�glicht somit flexible Tests der Architektur.
% Durch die funktionsgetreue Zeitmodellierung, ist die Zahl der Konfigurationsparameter gewachsen.
% Dementsprechend wurde ein neues Konfigurationsformat auf XML Basis eingef�hrt, dargestellt und beschrieben in
% Abbildung \ref{resource-config}.
% 
% \begin{figure}
% \centering
% {\footnotesize
% \verbatiminput{mapping-demo.xml}}
% \caption{\label{resource-config}%
% Beispiel des neuen XML Konfigurationsformates, welches wie
% folgend aufgebaut ist:
% Innerhalb des \code{configuration}-Top-Level-Elements befinden sich die zwei Elemente
% (i) \code{resources}, innerhalb dessen die \code{component}-Elemente die virtuellen Komponenten spezifizieren,
% und (ii) \code{mappings}, welches \code{mapping}-Elemente enth�lt zur Abbildung aller Prozesse
% auf genau eine Komponente.
% Sowohl die \code{component}- als auch die \code{mapping}-Elemente werden mittels eingeschachtelter
% \code{attribute}-Elemente weiter Parametrisiert. Diese Parametrisierung bestimmt bei den
% \code{component}-Elementen den zu verwendenden Scheduler und bei den \code{mapping}-Elementen
% die Ausf�hrungszeiten der Aktionen auf der ausgew�hlten Komponente.
% %Beispielhaft definiert \code{<attribute type="ib\_m\_atu::forward\_mfetch" value="2"/>}
% %eine Ausf�hrungszeit von 2 $\mathrm{ns}$ f�r die \code{ib\_m\_atu::forward\_mfetch}
% %Aktion auf der gew�hlten Komponente.
% %Eine Standardausf�hrungszeit kann ebenfalls mittels \code{<attribute type="delay" value="2"/>} angegeben werden,
% %diese wird verwendet falls die ausgef�hrte Aktion nicht im \code{mapping}-Element
% %parametrisiert ist.
% }
% \end{figure}

\subsection{Network graph behavior}



% LocalWords:  SysteMoC
