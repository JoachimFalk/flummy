\chapter{SysteMoC semantic}

\section{Actor behavior}

As known from Kahn \cite{Kahn:1974} and Lee \cite{Lee98,LeeDenotialDF:1997} we use the sequence of tokens $\mathbf{v} \in V^{**}$ transmitted via a FIFO to mathematical describe it.
In order to simplify this description we will replace a bounded FIFO with two unbounded FIFOs.
Therefore each actor port $p \in \mathcal{P} = I \cup O$ is connected to a pair of FIFOs $(\mathbf{v}, \mathbf{f}) \in Z$ containing
a \emph{value FIFO} $\mathbf{v} \in V^{**}$ and
a \emph{free space FIFO} $\mathbf{f} \in \{\circ\}^{**}$, where each \emph{free space symbol} $\circ$ represents an available slot in the value FIFO to store a token. % as known from PETRI-nets \cite{XXXX}.
Accordingly an actor not only consumes sequences of tokens $V^{**}$ on its input ports $I$ and produce them on its output ports $O$ but also produce sequences of free space symbols $\{\circ\}^{**}$ on its input ports and consume them on its output ports, respectively.
%For notational simplicity we will assume that the free space symbol is contained in the set of all values, i.e., $\circ \in V$.
To reduce the mathematical clutter we will not distinguish between \emph{value sequences} and \emph{free space symbol sequences} and simply use $S$ to denote the set of all finite or infinite sequences of values or free space symbols, i.e., $S = V^{**} \cup \{\circ\}^{**}$.

Before we can continue to define the behavior of an actor some mathematical notations for manipulating tuples are needed:

A tuple, e.g., $\mathbf{s} = (\mathbf{v}_{i_1}, \mathbf{v}_{i_2}, \mathbf{v}_{o_1})$, can be viewed as a function, e.g., $\mathbf{s}: \mathbb{N}_3 \to S$, over the counting set $\mathbb{N}_n = \{1,2,\ldots,n\}$ from one to the tuple size $n$.
Additionally we can associate which each tuple an orderd set, e.g., $\{i_1, i_2, o_1\} \subseteq \mathcal{P}$, and view the tuple as a function over this set, e.g., $\mathbf{s}: \{i_1, i_2, o_1\} \to S$.
This allows us to access a tuple member either by its position inside the tuple, e.g., $\mathbf{s}(3) = \mathbf{v}_{o_1}$, or by an indexing element, e.g., the actor port $o_1$ and its associated tuple element $\mathbf{s}(o_1) = \mathbf{v}_{o_1}$.
Moreover we use the pointwise extension of the tuple member access operator to extract from a tuple its associated ordered sets, e.g., $\mathbf{s}.\mathbb{N} = \{1, 2, 3\}$ or $\mathbf{s}.\mathcal{P} = \{i_1, i_2, o_1\}$.

The position of a sequence in its tuple is equivalent to its name.
To achieve the equivalent of renaming or hiding sequences, the position of a sequence in a tuple must be changed or the sequence must be dropped from the tuple.
This is done by applying the \emph{projection} function to a tuple.

\begin{definition}[Projection]\label{def:tsm-projection}
A \emph{projection} $\pi{}_I: X^n \to X^m$ is a function which discards and reorders members $x \in X$ of a $n$-tuple according to an ordered set of indexes $I$ to form a new $m$-tuple, where $|I| = m \le n$.
In other words, given $\mathbf{x} \in X^n$ and $I = \{i_1,\ldots,i_m\}$ then $\pi{}_I(\mathbf{x}) = (\mathbf{x}(i_1), \mathbf{x}(i_2), \ldots, \mathbf{x}({i_m}))$.
Furthemore a projection $\pi{}_I$ can be generalized to sets of tuples $\mathbf{X} \subset X^n$, i.e., $\pi{}_I(\mathbf{X}) = \{\pi{}_I(\mathbf{x}) \mid \mathbf{x} \in \mathbf{X}\}$.
\end{definition}

As an example consider the tuple of sequences $\mathbf{s} = (\mathbf{v}_{i_1}, \mathbf{v}_{i_2}, \mathbf{v}_{o_1})$, and the ordered set of actor ports $P = \{o_1, i_1\}$ then the projection is given as $\pi{}_P(\mathbf{s}) = (\mathbf{v}_{o_1}, \mathbf{v}_{i_1})$.

With the mathematical notation defined above it is now possible to formally describe the behavior of an actor.
\begin{definition}[Actor behavior]\label{def:actor-behavior}
The behavior of an actor $a$ can be thought of as a relation $\mathbf{P} \subseteq Z^{|\mathcal{P}|}$ which maps a tuple of input sequences $\mathbf{s}_\mathrm{in} \in \mathbf{S}_\mathrm{in} \subset S^{|\mathcal{P}|}$ into a tuple of output sequences $\mathbf{s}_\mathrm{out} \in \mathbf{S}_\mathrm{out} \subset S^{|\mathcal{P}|}$.
Where $\mathbf{S}_\mathrm{in} = \pi{}_{I}(\mathbf{P}).\mathbf{v} \times \pi{}_{O}(\mathbf{P}).\mathbf{f}$ and
$\mathrm{S}_\mathrm{out} = \pi{}_{I}(\mathbf{P}).\mathbf{f} \times \pi{}_{O}(\mathbf{P}).\mathbf{v}$.
% i.e., $\mathbb{A} \subseteq  \mathbf{S}_\mathrm{in} \times \mathbf{S}_\mathrm{out}$, 
% partitioned into actor input ports $I$ and actor output ports $O$
\end{definition}

%Where the tuple of input sequences $\mathbf{s}_\mathrm{in}$ is constraint to be devided into value sequences $\pi{}_{I}(\mathbf{s}_\mathrm{in}) \in (V^{**})^{|I|}$ and free space symbol sequences $\pi{}_{O}(\mathbf{s}_\mathrm{in}) \in (\{\circ\}^{**})^{|O|}$.
% , each sequence associated with an actor input port, , each sequence associated with an actor output port.
%Furthermore the tuple of output sequences $\mathbf{s}_\mathrm{out}$ is similarly devided into value sequences $\pi{}_{O}(\mathbf{s}_\mathrm{out}) \in (V^{**})^{|O|}$ and free space symbol sequences $\pi{}_{I}(\mathbf{s}_\mathrm{out}) \in (\{\circ\}^{**})^{|I|}$.
% , each sequence associated with an actor output port, , each sequence associated with an actor input port.

The execution of an actor is divided into atomic \emph{firing steps}.
The possible firing steps an actor can take are dependent on the tuple $\mathbf{s}_\mathrm{in}$ of still unconsumed input sequences and the actor state $q = (q_\mathrm{func}, q_\mathrm{firing}) \in Q = \mathcal{F}.Q_\mathrm{func} \times \mathcal{R}.Q_\mathrm{firing}$.
The $\mathrm{Fire}$-function deduces these possible steps and returns the set of possible produced output sequences $\mathbf{s}_\mathrm{prod}$ and the resulting actor state $q'$, i.e., $\mathrm{Fire}: \mathbf{S}_\mathrm{in} \times Q \to 2^{\mathbf{S}_\mathrm{out} \times Q}$.
\footnote{We use $2^X$ to denote the powerset of $X$, the set of all subsets of $X$, i.e., $2^X = \bigcup_{X_\mathrm{subset} \subseteq X} \{X_\mathrm{subset}\}$.}
To decide if a transition can be taken its associated \emph{activation pattern} must be evaluated (\ref{eqn:transitionenabled}).

\begin{definition}[Activation pattern]\label{def:activation-pattern}
  An \emph{activation pattern} $k$ of an actor $a \in A$ is a is a boolean function depending on the available input sequences $\mathbf{S}_\mathrm{in}$ and the \emph{functionality state} $\mathcal{F}.S_\mathrm{func}$ of the actor, i.e., $k: \mathbf{S}_\mathrm{in} \times \mathcal{F}.S_\mathrm{func} \to \{\mathrm{true}, \mathrm{false}\}$.
  The activation pattern is used to decide if its associated transition can be taken $(\mathrm{true})$ or not $(\mathrm{false})$.
\end{definition}

To further express the behavior of a firing step we need to define the prefix order $\sqsubseteq$ on the set of sequences $S$, i.e., given $\mathbf{u}, \mathbf{v} \in S$ then $\mathbf{u} \sqsubseteq \mathbf{v} \equiv \mathbf{u}.\mathbb{N} \subseteq \mathbf{v}.\mathbb{N} \wedge \forall{n \in \mathbf{u}.\mathbb{N}}: \mathbf{u}(n) = \mathbf{v}(n)$.
As an example consider the two sequences $\mathbf{u} = (v_1, v_2, v_3) \in S$ and $\mathbf{v} = (v_1, v_2, v_3, v_4, v_5) \in S$ then $\mathbf{u}$ is a prefix of $\mathbf{v}$, i.e., $\mathbf{u} \sqsubseteq \mathbf{v}$.
This prefix order has a trivial pointwise extenstion to tuples of sequences, i.e., given $\mathbf{s}_1, \mathbf{s}_2 \in S^n$ then $\mathbf{s}_1 \sqsubseteq \mathbf{s}_2 \equiv \mathbf{s}_1.\mathbb{N} = \mathbf{s}_2.\mathbb{N} \wedge \forall{n \in \mathbf{s}_1.\mathbb{N}}: \mathbf{s}_1(n) \sqsubseteq \mathbf{s}_2(n)$.
Additionally we will use the $\length: S \to \mathbb{N}_\infty$ operator to denote the length of a sequence, e.g., given $\mathbf{u} = (v_1, v_2, v_3)$ then $\length\mathbf{u} = |\mathbf{u}.\mathbb{N}| = 3$,
and its pointwise extenstion to tuples of sequences $\length: S^n \to \mathbb{N}^n_\infty$, e.g., given $\mathbf{s} = (\mathbf{u}, \mathbf{v})$ then $\length\mathbf{s} = (3, 5)$.

With the notation given above we can now derive the following formall definition:

\begin{definition}[Firing step]\label{def:firing-step}
The execution of an actor is divided into atomic \emph{firing steps}.

\begin{eqnarray}
\mathrm{Fire}(\mathbf{s}_\mathrm{in}, q) = \{ & & \label{eqn:dependence} \\
 &        & (\mathbf{s}_\mathrm{prod}, q') \label{eqn:production} \\
 & |      & (q.q_\mathrm{firing}, k, f_\mathrm{action}, q'.q_\mathrm{firing}) \in \mathcal{R}.T \label{eqn:needtransition} \\
 & \wedge & \mathbf{s}_\mathrm{cons} \sqsubseteq \mathbf{s}_\mathrm{in} \label{eqn:consumeprefix} \\
 & \wedge & k(\mathbf{s}_\mathrm{cons}, q.q_\mathrm{func}) \label{eqn:transitionenabled} \\
 & \wedge & \forall{\mathbf{s}_\mathrm{prefix} \sqsubset \mathbf{s}_\mathrm{cons}}: \neg k(\mathbf{s}_\mathrm{prefix}, q.q_\mathrm{func}) \label{eqn:consume-glb}\\
 & \wedge & (\pi{}_{O}(\mathbf{s}_\mathrm{prod}), q'.q_\mathrm{func}) = f_\mathrm{action}(\pi{}_{I}(\mathbf{s}_\mathrm{cons}), q.q_\mathrm{func}) \label{eqn:action}\\
 & \wedge & \#\pi{}_{I}(\mathbf{s}_\mathrm{prod}) = \#\pi{}_{I}(\mathbf{s}_\mathrm{cons}) \label{eqn:freespacesymbols} \}
\end{eqnarray}

The possible firing steps an actor can take are dependent on \emph{the tuple of still unconsumed input sequences} $\mathbf{s}_\mathrm{in} \in \mathbf{S}_\mathrm{in}$ and the \emph{actor state} $q \in Q$ (\ref{eqn:dependence})
The $\mathrm{Fire}$-function deduces these steps and returns a set containing \emph{tuples of produced output sequences} $\mathbf{s}_\mathrm{prod}$ and the \emph{resulting actor state} $q'$ (\ref{eqn:production}).
The firing steps corresponds to the \emph{execution of a transition} $t \in \mathcal{R}.T$ of the \emph{firing FSM} $\mathcal{R}$ (\ref{eqn:needtransition}).
Each firing step of an actor consumes a tuple of finite sequences $\mathbf{s}_\mathrm{cons}$, which must be a prefix of the tuple of input sequences (\ref{eqn:consumeprefix}).
To decide if a transition can be taken its associated activation pattern must be evaluated (\ref{eqn:transitionenabled}).
Consume sequence must be smallest sequence which still satisfies activation pattern (\ref{eqn:consume-glb}).
Consequently a tuple of output sequences and a resulting actor state is produced (\ref{eqn:action}).
The number of free space symbols generated on the input ports equals the number of tokens consumed on them (\ref{eqn:freespacesymbols}).
\end{definition}

%Furthermore each firing step transforms the \emph{actor state} $q = (q_\mathrm{func}, q_\mathrm{firing}) \in Q = \mathcal{F}.Q_\mathrm{func} \times \mathcal{R}.Q_\mathrm{firing}$.
% sequence of tokens $\mathbf{v} \in V^{**}$ for each actor input port $i \in I$ and sequences of free space tokens $\mathbf{f} \in \{\circ\}^{**}$ for each actor output port $o \in O$,

%$z = (\mathbf{v}, \mathbf{f}) \in Z$, the data FIFO $z.\mathbf{v} \in V^{**}$ and the free space FIFO $z.\mathbf{s} \in \{\circ\}^{**}$.


%$a.\mathcal{P}' = [i_1, i'_1, i_2, i'_2, \ldots, i_{|a.I|}, i'_{|a.I|}, o_1, o'_1, o_2, o'_2, \ldots, o_{|a.O|}, o'_{|a.O|}]$
%$a.\mathcal{I}' = \{i_1, i_2, \ldots, i_{|a.I|}, o'_1, o'_2, \ldots, o'_{|a.O|}\}$
%$a.\mathcal{O}' = \{i'_1, i'_2, \ldots, i'_{|a.I|}, o_1, o_2, \ldots, o_{|a.O|}\}$
%$\mathbf{z} \in Z^{|\mathcal{P}|}$ actor ports $\mathcal{P}$
%$\mathrm{IN}(\mathbf{z}) = \pi{}_{I}(\mathbf{z}).\mathbf{v} \times \pi{}_{O}(\mathbf{z}).\mathbf{f}$
%$\mathrm{OUT}(\mathbf{z}) = \pi{}_{I}(\mathbf{z}).\mathbf{v} \times \pi{}_{O}(\mathbf{z}).\mathbf{f}$

%\begin{eqnarray}
%\mathrm{Fire}(\mathbf{z}, s_\mathrm{firing}, s_\mathrm{func}) = \{ (\mathbf{z}', s'_\mathrm{firing}, s'_\mathrm{func}) %
% & |      & (s_\mathrm{firing}, k, f_\mathrm{action}, s'_\mathrm{firing}) \in T \\
% & \wedge & k(\mathbf{v}_\mathrm{cons}) \\
% & \wedge & \forall{\mathbf{v}_\mathrm{prefix} \sqsubset \mathbf{v}_\mathrm{cons}}: \neg k(\mathbf{v}_\mathrm{prefix}) \\
% & \wedge & (\mathbf{w}_\mathrm{prod}, s'_\mathrm{func}) = f_\mathrm{action}(\mathbf{v}_\mathrm{cons},s_\mathrm{func}) \\
%\end{eqnarray}

%\begin{eqnarray}
%\mathrm{Fire}(\mathbf{v}, s_\mathrm{firing}, s_\mathrm{func}) = \{ \mathbf{w}_\mathrm{prod}\concat\mathbf{w}_\mathrm{tail} %
% & |      & (s_\mathrm{firing}, k, f_\mathrm{action}, s'_\mathrm{firing}) \in T \\
% & \wedge & k(\mathbf{v}_\mathrm{cons}) \\
% & \wedge & \forall{\mathbf{v}_\mathrm{prefix} \sqsubset \mathbf{v}_\mathrm{cons}}: \neg k(\mathbf{v}_\mathrm{prefix}) \\
% & \wedge & (\mathbf{w}_\mathrm{prod}, s'_\mathrm{func}) = f_\mathrm{action}(\mathbf{v}_\mathrm{cons},s_\mathrm{func}) \\
% & \wedge & \mathbf{v} = \mathbf{v}_\mathrm{cons}\concat\mathbf{v}_\mathrm{tail} \\
% & \wedge & \mathbf{w}_\mathrm{tail} = \mathrm{Fire}(\mathbf{v}_\mathrm{tail}, s'_\mathrm{firing}, s'_\mathrm{func}) \}
%\end{eqnarray}

%$\mathbb{A} = \{ (\mathbf{v},\mathbf{w}) | \mathbf{w} \in \mathrm{Fire}(\mathbf{v}, s_{_0\mathrm{firing}}, s_{_0\mathrm{func}}), \mathbf{v} \in (V^{**})^{|A.I|} \times (\{\circ\}^{**})^{|a.O|} \}$

% \section{Kopplung zwischen SysteMoC und VPC}
% \label{sec:smoc-vpc}
% Die SysteMoC-Bibliothek erm�glicht es, einzelne Aktoren in Form von Zustandsmaschinen, die Funktionen aufrufen, zu beschreiben.
% Eine Anwendung, z.B. der InfiniBand-HCA besteht in der Regel aus einem Netzwerk solcher Aktoren.
% Ein solches Netzwerk kann durch Simulation bez�glich der Funktionalit�t untersucht werden.
% Das \emph{Virtual Processing Components} Framework~\cite{streubuehr:2005} erm�glicht die zus�tzliche Simulation des Zeitverhaltens unter Ressourcenbeschr�nkung.
% Hierf�r wird f�r jede Aktivierung eines Aktors eine dynamisch ermittelte Ausf�hrungszeit abgewartet.
% Diese Ausf�hrungszeit wird durch die Bindung an eine Ressource, sowie durch eine Schedulingstrategie bestimmt.
% Einzig notwendig hierf�r sind zwei Funktionsaufrufe (\code{getResource}, \code{compute}) aus der Anwendung heraus (vgl.~\cite{streubuehr:2005, fghsst:2005}).
% \par
% Zusammen mit einer Aktorbibliothek, wie der SysteMoC, k�nnen diese Aufrufe vollst�ndig transparent f�r den Nutzer eingef�gt werden.
% %Zusammen mit der SysteMoC-Bibliothek gestaltet sich die Kopplung wie folgt:
% Jedes mal wenn in einem Aktor eine Transition ausgef�hrt wird, entspricht dieses einer Aktivit�t.
% Bevor diese Funktionalit�t ausgef�hrt wird, werden die Funktionsaufrufe \code{getResource} und \code{compute} ausgef�hrt und somit die simulierte Zeit verbraucht.
% %\textbf{TODO: globales scheduling bzw. dessen Effekte erw�hnen??}
% \par
% Weiterhin zeichnen sich die SysteMoC Aktoren durch ihre funktionsakkurate Aus\-f�hr\-ungs\-sem\-an\-tik aus:
% Ein Aktor besitzt ausdr�cklich modellierte Zust�nde und Transitionen, wobei Transitionen die Aktivit�ten darstellen und Zustandswechsel verursachen.
% Das VPC wurde dahingehend erweitert, dass die Ausf�hrungszeit auch in Abh�ngigkeit zur ausgef�hrten Transition steht.
% Dementsprechend wurde die \code{compute}-Funktion um einen weiteren Parameter f�r die auszuf�hrende Transition erg�nzt.
% W�hrend die Bindung an Ressourcen weiterhin aktorakkurat stattfindet, wird die Ausf�hrungszeit funktionsakkurat modelliert.
% Wie schon bei der vorangegangenen Studie wird das VPC durch eine Konfiguration parametrisiert, und erm�glicht somit flexible Tests der Architektur.
% Durch die funktionsgetreue Zeitmodellierung, ist die Zahl der Konfigurationsparameter gewachsen.
% Dementsprechend wurde ein neues Konfigurationsformat auf XML Basis eingef�hrt, dargestellt und beschrieben in
% Abbildung \ref{resource-config}.
% 
% \begin{figure}
% \centering
% {\footnotesize
% \verbatiminput{mapping-demo.xml}}
% \caption{\label{resource-config}%
% Beispiel des neuen XML Konfigurationsformates, welches wie
% folgend aufgebaut ist:
% Innerhalb des \code{configuration}-Top-Level-Elements befinden sich die zwei Elemente
% (i) \code{resources}, innerhalb dessen die \code{component}-Elemente die virtuellen Komponenten spezifizieren,
% und (ii) \code{mappings}, welches \code{mapping}-Elemente enth�lt zur Abbildung aller Prozesse
% auf genau eine Komponente.
% Sowohl die \code{component}- als auch die \code{mapping}-Elemente werden mittels eingeschachtelter
% \code{attribute}-Elemente weiter Parametrisiert. Diese Parametrisierung bestimmt bei den
% \code{component}-Elementen den zu verwendenden Scheduler und bei den \code{mapping}-Elementen
% die Ausf�hrungszeiten der Aktionen auf der ausgew�hlten Komponente.
% %Beispielhaft definiert \code{<attribute type="ib\_m\_atu::forward\_mfetch" value="2"/>}
% %eine Ausf�hrungszeit von 2 $\mathrm{ns}$ f�r die \code{ib\_m\_atu::forward\_mfetch}
% %Aktion auf der gew�hlten Komponente.
% %Eine Standardausf�hrungszeit kann ebenfalls mittels \code{<attribute type="delay" value="2"/>} angegeben werden,
% %diese wird verwendet falls die ausgef�hrte Aktion nicht im \code{mapping}-Element
% %parametrisiert ist.
% }
% \end{figure}

\section{Network graph behavior}



% LocalWords:  SysteMoC
