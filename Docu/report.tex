\newcommand{\graphicPostfix}{pdf}
\newcommand{\resdir}{../../../HscdTeXRes/}
\newcommand{\code}[1]{\texttt{#1}}
\newcommand{\SysteMoC}{{\bf SysteMoC}}
\newcommand{\SysteMoCV}{\SysteMoC{} {\bf 1.0}}

\input{\resdir format.tex}

\usepackage{multirow}
\usepackage{fancyvrb}

% Use a small font for the verbatim environment
\makeatletter  % makes '@' an ordinary character
\renewcommand{\verbatim@font}{%
  \ttfamily\small\catcode`\<=\active\catcode`\>=\active%
}
\makeatother   % makes '@' a special symbol again

% \begin{Verbatim}[fontsize=\tiny]


\begin{document}
\hscdtitle{Representing Models of Computation in SystemC}
{Joachim Falk, Christian Haubelt, J�rgen Teich}{\today}
\clearpage
\tableofcontents
\clearpage

\section{Introduction\label{intro}}
Due to rising design complexity it is necessary to increase
the level of abstraction at which systems are designed. In
today's embedded systems the specification is mostly mapped
into a set of interacting tasks and hardware modules, which 
interact by the use of shared variables and various ways of
message passing. To guarantee exclusive access to the shared
variables and resources mutually exclusive devices like semaphores,
mutexes and monitors are used. But this unstructured use
of communication types lead to hardly analysable systems.
However, by constraining the type of communication
used between tasks and the communication behavior of the tasks,
expressiveness is traded for analyzability.

\emph{Models of computation}
\cite{embsft:2002}, in the following called \emph{MoCs}, are
predefined types of communication and strategies for scheduling
the communicating tasks. Thus MoCs are comparable to design
patterns known from the area of software design \cite{gamma:1995}.
Limiting the expressiveness of an MoC enables:

\begin{enumerate}
\item Mathematically reasoning about communication patterns

\item Code generators to produce optimized code

\item Verification tools to check system properties automatically
\end{enumerate}

On the other hand, modern embedded systems design is still based on
specification languages which allow unstructured communication.
Even worse, nearly all specification languages allow for Turing
complete MoCs making analysis in general impossible.
To make industry benefit from the best of both worlds,
engineers must restrict themselves to use certain subsets of
specification languages allowing for analyzability, identification
and extraction of these MoCs from the source code automatically.

These report deal with the representation of some transformative
MoCs in SystemC. The rest of the report is structured as follows:
In Section~\ref{related-work}, we discuss related work. In
Section~\ref{concept-overview}, we present our concepts for
representing MoCs in SystemC. In Section~\ref{requirements-of-mocs},
we exemplify for some MoCs how they fit into
the concepts presented before. In Section~\ref{software-architecture}
an implementation of these concepts in SystemC called \SysteMoC{} is
presented. We follow this with an example model using the \SysteMoC{}
framework and conclude the report in Section~\ref{conclusion}.

\section{Related Work}\label{related-work}

\section{Concept Overview}\label{concept-overview}

A model of computation is defined \cite{embsft:2002} as the 
interaction policy between \emph{actors} in an \emph{actor-oriented} design
\cite{agha97abstracting:1997}.
Actors are objects which execute concurrently. They can only
communicate with the environment (other actors) through their ports.
The actor ports are connected with each other via a \emph{channel network}.
The data values, which are communicated over the channels
are abstracted. The values are called \emph{tokens} regardless
of their type or value.

Instead of a monolithic approach for representing an executable specification,
we choose in this report to separate the specification into several
independent concepts.
This is a refinement of actor-oriented design.
Actor-oriented design has actors which execute concurrently and only
communicate with each other via channels instead of method calls as known
in object-oriented design. In our methodology, the specification is more
detailed than in actor-oriented design.
One aspect of the separation is a devision of an actor
into control flow and data flow concepts. Another aspect is the
separation of the channel network into the structure of the network
and the type of channel used for communication.
In the following, the concepts into which the executable specification
is separated is discussed in depth:

\subsection{Node Functionality}

In our methodology the actor concept is still used, but is split into two parts.
There is the part called node functionality which is responsible
for transforming data values. The node functionality can be
thought of as a function $F: V^{m} \to V^{n}$  which maps a fixed set
of parameters to a fixed set of results. The node functionality must
therefore terminate and it must not communicate with
other actors.

\subsection{Node Interface}

The other part of the actor which determines the communication
behavior of the actor is called firing rules. The firing rules
determine for each actor port the number of tokens needed
that a node functionality can be invoked.
%%Firing rules can be invoked
%%Examples of firing 
The node interface is the set of communication operations an
actor has available to construct its firing rules.

\subsection{Network Graph Type}

To represent the executable specification not only information about
individual actors must be contained but also information about
the connections of the actor ports to each other and to uplevel ports.
This information is stored in the network graph \label{network-graph}
$N=(\mathcal{A},C,P_{i},P_{o},E)$ a 5-tuple containing
a set of actors $\mathcal{A}$, a set of channels $C$, a set of uplevel
input ports $P_{i}$, a set of uplevel output ports $P_{o}$ and a set of
directed edges $E \subseteq ((P_{i} \bigcup C) \times \mathcal{A}_{P_{i}})
\bigcup (\mathcal{A}_{P_{o}} \times (P_{o} \bigcup C))$.
Each actor $A \in \mathcal{A}$ can only communicate with other actors
through its set of dedicated actor input ports $A_{P_{i}}$ and
actor output ports $A_{P_{o}}$.
% $\forall{(X,Y)\in \mathcal{A^2}}: X \ne Y \Rightarrow X_{P} \cap Y_{P} = \emptyset$
Furthermore we define the set of all
actor input ports as $\mathcal{A}_{P_{i}} = \bigcup{}_{A \in \mathcal{A}} A_{P_{i}}$ and
the set of all actor output ports as
$\mathcal{A}_{P_{o}} = \bigcup{}_{A \in \mathcal{A}} A_{P_{o}}$.

There are two network graph types \label{network-graph-types} which are distinguished by our methodology:

\begin{enumerate}
\item Petri network graph \label{petri-network-graph}

  Petri network graphs are network graphs which are constrained that only
  one edge enters an actor input port
  $\forall{p \in \mathcal{A}_{P_{i}}}: |(C \times \{p\}) \bigcap E| \le 1$.
  An example of a petri network graph is displayed in Figure~\ref{ng-petri}.

\item Dataflow network graph \label{sdf-network-graph}

  Dataflow network graphs are petri network graphs with the additional constraint,
  that the indegree and outdegree of every channel in the graph must be one
  $\forall{c \in C}: |(\{c\} \times \mathcal{A}_{P_{i}}) \bigcap E| \le 1 \wedge
                    |(\mathcal{A}_{P_{o}} \times \{c\}) \bigcap E| \le 1$.
  An example of a Dataflow network graph is displayed in Figure~\ref{ng-sdf}.
\end{enumerate}

\begin{figure}
\centering
\includegraphics[scale=1]{ng-petri.\graphicPostfix}\\
\caption{\label{ng-petri}Petri network graph}
\end{figure}

\begin{figure}
\centering
\includegraphics[scale=1]{ng-sdf.\graphicPostfix}\\
\caption{\label{ng-sdf}Dataflow network graph}
\end{figure}

\subsection{Channel Kind}

The channel kind defines the communication semantic of the channel.
Because the communication semantic is not influenced by the
type of the values communicated, the actual type of a communication
channel is derived from the channel kind by parameterizing it with
the type of the communicated values. Examples for channel kinds are:

\begin{enumerate}
\item FIFO
  An unbound FIFO channel for communication
  (Nonblocking write and blocking read).
\item Rendezvous
  A rendezvous channel for communication % and barrier synchronisation
  (Blocking write and read).
\item Register
  For expressing communication via shared variables
  (Nonblocking write and read).
\end{enumerate}

\section{Revision of Different MoCs}\label{requirements-of-mocs}

The requirements of different MoCs can be separated into two distinct groups.
There are the execution requirements which must be fulfilled by
the programming system otherwise the MoC cannot be implemented in
this particular system.
On the other hand, there are the analysis requirements which
permit the extraction of information needed for the MoC specific analysis.

In the following Subsections, different MoCs are considered and their
requirements determined. For each MoC these requirements are than
mapped onto the concepts presented in the previous Section.

\subsection{Communicating Sequential Processes (CSP)}
\emph{Communicating Sequential Processes} \cite{csphoare:1985}
is a MoC which consist of concurrently executing processes
which communicate via unidirectional rendezvous channels.
As a difference to the CSP model of Hoare \cite{csphoare:1985}
which allows runtime creation of processes via parallel
composition and  recursion, the CSP model presented in this report
is constraint to a fixed predetermined number of processes.
This constraint was set to allow hardware synthesis from this
model.

The following requirements are the execution requirements to
implement CSP behavior:

\begin{enumerate}
\item A rendezvous channel for communication
\item A communication method for Hoare's so-called \emph{general choice}
\item Parallel executing processes
\end{enumerate}

To facilitate analysis of the CSP MoC additionally the following
constraints must be satisfied:
\begin{enumerate}
\item The indegree and outdegree of all channels in the network graph must be one.
\item The actors must only communicate over their ports which each other.
\end{enumerate}

Mapping of the CSP MoC requirements onto the concepts presented in
Section~\ref{concept-overview}:

\begin{itemize}

\item The node interface must provide a communication operator which
      generates a 'general choice' communication.

\item The connections needed for CSP are provided by the Dataflow network graph.

\item As channel kind rendezvous is selected.

\end{itemize}

\subsection{Dataflow (DF)}
Kahn Process Networks \cite{kahn:1974}
is a MoC which consist of concurrently executing processes
which communicate via unidirectional unbounded FIFO channels.
The input/output ends of a channel are connected to dedicated
processes.

The following requirements are the execution requirements to
implement KPN behavior:

\begin{enumerate}
\item An unbound FIFO channel for communication
\item A communication method to read from and write to FIFOs
\item Parallel executing processes
\end{enumerate}

To facilitate analysis of the KPN MoC additionally the following
constraints must be fulfilled:
\begin{enumerate}
\item The indegree and outdegree of all channels in the network graph must be one.
\item Assurance that the arrival sequence of tokens at
      their ports cannot be determined.
\item The actors must only communicate over their ports which each other.
\end{enumerate}

Mapping of the KPN MoC requirements onto the concepts presented in
Section~\ref{concept-overview}:

\begin{itemize}

\item The node interface must only provide a communication operator which
      generates parameterized read and write requests on the actor ports.

\item The connections needed for CSP are provided by the Dataflow network graph.

\item As channel kind FIFO is selected.

\end{itemize}

\subsection{Synchronous Dataflow (SDF)}
Synchronous Data Flow \cite{Lee87b:1987}
is a more constricted form of the KPN MoC.
It has all the execution requirements of the KPN MoC and
the additional analysis requirement that actor consumption and
production rates are constant.

The following requirements are the execution requirements to
implement SDF behavior:

\begin{enumerate}
\item An unbound FIFO channel for communication
\item A communication method to read from and write to FIFOs
\item Parallel executing processes
\end{enumerate}

To facilitate analysis of the SDF MoC additionally the following
constraints must be fulfilled:
\begin{enumerate}
\item The indegree and outdegree of all channels in the network graph must be one.
\item Assurance that the arrival sequence of tokens at
      their ports cannot be determined.
\item Assurance that consumption and production rates cannot
      be altered once set.
\item The actors must only communicate over their ports which each other.
\end{enumerate}

Mapping of the SDF MoC requirements onto the concepts presented in
Section~\ref{concept-overview}:

\begin{itemize}

\item The node interface must only provide a communication operator which
      generates read and write requests with a fixed set of actor ports.

\item The connections needed for CSP are provided by the Dataflow network graph.

\item As channel kind FIFO is selected.

\end{itemize}

\section{Software Architecture}\label{software-architecture}

Due to its high abstraction level and possibilities for both
hardware and software refinements we have chosen SystemC
\cite{systemc-lrm:2003} \cite{glms:2002} as our language for system design.
SystemC is a C++ based design framework.

In the following we propose a SystemC framework, called \SysteMoC, which
facilitates automatic extraction of the MoC of a SystemC design.
%Actors can have their own thread of control or be without. Actors
%with their own thread of control will be called in the following
%active actors and actors without a dedicated thread of control
%passive actors.
Actors in SystemC are C++ classes which are derived from the base class \code{sc\_module}.
However it is impossible, in the general case, to extract the kind of communication
behavior these actors will exhibit. The \SysteMoC{} framework must now provide the
execution requirements for each MoC while still allowing to extract the information
needed for analysis. This analysis is made possible by dividing the actor into its
node functionality and its node interface.
Furthermore the \SysteMoC{} framework must provide a way for these
actors to be connected to each other via their ports. This connections
are provided by the network graph.

\begin{figure}
\centering
\begin{verbatim}
template <typename T>
class m_adder // Actor m_adder
  : public hscd_fixed_transact_node // Node interface
{
public:
  hscd_port_in<T>  in;
  hscd_port_out<T> out;
private:
  // Node functionality
  void transform() {
    out[0] = in[0] + in[1];
    std::cout << "Adding " << in[0] << " + " << in[1]
              << " = " << out[0] << std::endl;
  }
  
  // Firing rules
  void process() {
    while (true) {
      transform();
      transact();
    }
  }
public:
  m_adder( sc_module_name name )
    :hscd_fixed_transact_node( name,
        in(2) & out(1) /* Firing rules */ ) {}
};
\end{verbatim}
\caption{\label{example-adder-actor}Example of an adder in the \SysteMoC{} framework}
\end{figure}

\begin{figure}
\centering
\begin{verbatim}
template <typename T>
class m_multiply // Actor m_multiply
  : public hscd_fixed_transact_node // Node interface
{
public:
  hscd_port_in<T>  in1;
  hscd_port_in<T>  in2;
  hscd_port_out<T> out;
private:
  // Node functionality
  void transform() {
    out[0] = in1[0] + in2[0];
    std::cout << "Multiplying" << in1[0] << " * " << in2[0]
              << " = " << out[0] << std::endl;
  }
  
  // Firing rules
  void process() {
    while (true) {
      transform();
      transact();
    }
  }
public:
  m_multiply( sc_module_name name )
    :hscd_fixed_transact_node( name,
        in1(1) & in2(1) & out(1) /* Firing rules */ ) {}
};
\end{verbatim}
\caption{\label{example-multiply-actor}Example of a multiplier in the \SysteMoC{} framework}
\end{figure}

\subsection{Node Functionality}
The node functionality is only used for algorithmic transformations of data values.
The node functionality of an actor is defined in certain member functions,
which are called by the firing rules if their requirements for input data is met.
This member functions are not allowed to call communication operations and
the input and output data values used are read from
and written to the actor ports. All actor ports are globally accessible
by the actor member functions and can therefore be used to get input data
and place output data for the node functionality
(See Figure~\ref{example-adder-actor} method \code{transform} for an example of a node functionality).

\subsection{Node Interface Hierarchy}
The problem of extracting the communication behavior of an actor is caused by the fact,
that the communication methods in SystemC are all accessible to all SystemC modules.
And execution of these methods is controlled by the Turing equivalent coding possibilities
in the actor. To partially redress these problems we introduce the concept of node interface
which determines what communication methods are available to an actor. This is implemented
by disallowing all communication methods on \SysteMoC{} channels and providing them instead
in a base class from which the actor is derived. Therefore the node interface of an actor
corresponds to the base class from which the actor is derived
(See Table~\ref{node-interface-c++} for mapping of the node interface names to their
corresponding C++ class types).
Communication over other media than \SysteMoC{} channels is forbidden but this cannot be enforced
because the full power of SystemC should be available for the node functionality part
of the actor. So we require form the \SysteMoC{} user the assurance that the
node functionality does terminate and does not communicate with other actors.

\begin{table}
\centering
\begin{tabular}{|l|l|}
\hline
 Node interface & \SysteMoC{} type \\
\hline \hline
 Choice Node          & \code{hscd\_choice\_node} \\
 Transact Node        & \code{hscd\_transact\_node} \\
 Fixed Transact Node  & \code{hscd\_fixed\_transact\_node} \\
\hline
\end{tabular}
\caption{\label{node-interface-c++}Node interfaces represented as \SysteMoC{} classes}
\end{table}

The following subset of the requirements enumerated in Section~\ref{requirements-of-mocs}
determines the node interface of the corresponding MoCs:

\begin{enumerate}
\item Choice Node
  \begin{itemize}
  \item  A communication method which implements Hoare's 'general choice' operator \cite{csphoare:1985}.
  \end  {itemize}
\item Transact Node
  \begin{itemize}
  \item A communication method to read from and write to ports.
  \item Assurance that the arrival sequence of tokens at
        their ports cannot be determined.
  \end  {itemize}
\item Fixed Transact Node
  \begin{itemize}
  \item A communication method to read from and write to ports.
  \item Assurance that the arrival sequence of tokens at
        their ports cannot be determined.
  \item Assurance that consumption and production rates cannot
        be altered once set
  \end  {itemize}
\end{enumerate}

These requirements build a hierarchy of decreasing capability. The node
interface of CSP is the most capable in the hierarchy and it
fulfills the execution requirements of all more constraint interfaces.
To fulfill the execution requirement of KPN the general choice operator
can be used to implement the communication method for KPN by constraining
the list of actor ports for the choice operation to one and therefore
eliminating the choice between different actor ports.
The execution requirement of SDF are the same than that of KPN and can
therefore be fulfilled by the node interface of KPN.

However the analysis requirements of the MoC in the hierarchy
get progressively more constraining and therefore the power
of the communication operations in the node interface must
be reduced accordingly to meet them.

\begin{figure}
\centering
\includegraphics[scale=0.5]{NodeInterfaceHierarchy.\graphicPostfix}\\
\caption{\label{node-interface-hierarchy}Node interface hierarchy as UML}
\end{figure}

The hierarchy of the requirements is transformed into a C++ inheritance
hierarchy with the CSP node interface at the root (See Figure~\ref{node-interface-c++}).
Less capable node interfaces inherit from more capable ones. The
less capable node interfaces use the communication operations from the base class node
interface to implement their own operations. The communication operations of the
base class are disabled by declaring them private.

\begin{enumerate}
\item CSP
  \begin{itemize}
  \item  \code{choice( a(1) $\arrowvert$ b(2) )}

    This communication operation gets a list of actor ports (e.g. $a$, $b$) which
    are parameterized with the number of tokens that must be communicated
    over the port. The operations communicates over that actor port which is first ready
    for communication. If no port is ready for communication the operation blocks until
    at least one port is ready. Should more than one actor port be ready for
    communication at the same instance one of them is chosen nondeterministically.
  \end  {itemize}
\item KPN
  \begin{itemize}
  \item \code{transact( a(1) \& b(2) )}

    This communication operation gets a list of actor ports (e.g. $a$, $b$) which
    are parameterized with the number of tokens that must be communicated
    over the port. The operations blocks until all requested tokens on the ports
    have been communicated.
  \end  {itemize}
\item SDF
  \begin{itemize}
  \item \code{transact(), hscd\_fixed\_transact\_node( ..., a(1) \& b(2) )}

    Actors are C++ classes which are derived of their node interface class.
    The consumption and production rates for SDF Actors must be fixed, therefore
    the node interface of a SDF-Actor is parameterized with the consumption and production
    rates of the SDF-Actor (See actor constructor in Figure~\ref{example-multiply-actor} for an
    example of fixed consumption and production rates). The communication operation \code{transact} is
    therefore missing the port list which is available to KPN nodes.
  \end  {itemize}
\end{enumerate}

\begin{figure}
\centering
\begin{verbatim}
class m_top2: public hscd_sdf_constraintset {
  public:
    hscd_port_in<int>  in1;
    hscd_port_in<int>  in2;
    hscd_port_out<int> out;
    
    m_top2()
      : hscd_sdf_constraintset()
    {
      m_adder<int>    &adder =
        registerNode(new m_adder<int>("adder"));
      m_multiply<int> &mult  =
        registerNode(new m_multiply<int>("multiply"));
      
      connectInterfacePorts( in1, adder.in1 );
      connectInterfacePorts( in2, mult.in1 );
      connectNodePorts( adder.out, mult.in2 );
      connectNodePorts( mult.out2, adder.in2,
        hscd_fifo<int>() << 13 /* Start marking */ );
      connectInterfacePorts( out, mult.out1 );
    }
};
\end{verbatim}
\caption{\label{example-sdf-constraintset}Example of a network graph for the SDF-Actor in the \SysteMoC{} framework}
\end{figure}

\subsection{Network Graph Type}

To determine the MoC of a SystemC model additionally to the information
required about the actors of the model information is required about the
type of the network graph (See Subsection~\ref{network-graph-types} for defined graph types).
The network graph is represented by a C++ class, in the following called
network graph class. This class is derived from a graph type class witch
provides the methods needed to assemble the network graph in the constructor
of the network graph class. Once the constructor has finished the network graph must
be fully assembled. No alteration at a later date is allowed.

The operator set provided by the graph type class together with runtime checks
in these methods enforces that in the network graph class
only a graph of the type determined by the graph type class can be assembled.
Therefore the network graph type of a network graph can be determined by looking
at the base class of the network graph class.
(See Table~\ref{network-graph-types-c++} for mapping of the network graph types to their
corresponding C++ class types).

\begin{table}
\centering
\begin{tabular}{|l|l|}
\hline
 Network graph type   & \SysteMoC{} type \\
\hline \hline
 Petri Choice Node    & \code{hscd\_graph\_petri} \\
 Transact Node        & \code{hscd\_graph\_sdf} \\
\hline
\end{tabular}
\caption{\label{network-graph-types-c++}Network graph types represented as \SysteMoC{} classes}
\end{table}

To construct the network graph the following methods are available for
composition of a network graph of the desired type:
\begin{enumerate}
\item Petri network graph

  \begin{itemize}
  \item registerNode:
    This method is used to add actor vertices to the network graph.
    Each registered actor must implement
    the parameterized node interface of the \code{hscd\_structure}
    template or an even less capable node interface.
    This constraint is enforced by the type signature
    of the registerNode method and the node interface hierarchy.

  \item registerChan:
    This method is used to add channel vertices to the network graph.
    Each registered channel must be of the parameterized channel kind.
    The constraint is enforced by the type signature
    of the registerChan method.

  \item connectChanPort:
    This method adds the edges between the actor ports and the
    channels.

  \item connectInterfacePorts

  \end{itemize}

\item Dataflow network graph

  \begin{itemize}
  \item registerNode: Same method as defined for petri network graphs.
  
  \item connectInterfacePorts: Same method as defined for petri network graphs.
  
  \item connectNodePorts:
    This method is a shorthand for adding a channel and
    connecting two actor ports via this channel. Only channels
    which are connected to a dedicated output actor port and
    a dedicated input actor port can be realized with it.
  
  \end{itemize}

\end{enumerate}

\subsection{Channel Kind}

As SystemC is an actor-oriented design framework, it also has the concepts
of actors and channels. So the channel kind concept of the \SysteMoC{}
framework can cleanly be implemented as a base class of a channel in SystemC.
The channel kind determines the communication semantic of a channel but not
the data type of the tokens. The channel type is a template parameterized
with the data type for the tokens and derived from the channel kind.
Identifying the channel kind with a base class instead of a template enables
easier type signature checks of C++ for the enforcement of a particular
channel kind of a channel instance.

Further differences between \SysteMoC{} channels and a SystemC channels consist of the
absence of user callable methods for communicating on the \SysteMoC{} channels.
Having those communication methods on the channels would contradict the separation
of node interface and channel kind. The mapping of channel kind to corresponding C++
class types is displayed in Table~\ref{channel-kind-c++}.

\begin{table}
\centering
\begin{tabular}{|l|l|}
\hline
 Channel kind & \SysteMoC{} type \\
\hline \hline
 FIFO         & \code{hscd\_fifo\_kind} \\
 Rendezvous   & \code{hscd\_rendezvous\_kind} \\
\hline
\end{tabular}
\caption{\label{channel-kind-c++}Channel kinds represented as \SysteMoC{} classes}
\end{table}

\subsection{Constraint Set Composition for MoCs}

The network graph is represented by the C++ template
\code{hscd\_structure} which has the parameters node interface
and channel kind. The network graph is assembled by the
constructor of the  \code{hscd\_structure} template.

The channel kind is the parameter for the
\code{hscd\_structure} template which determines
the semantic of all channels in the parameterized
\code{hscd\_structure} template.





The MoC is represented as an actor which is derived from the \code{hscd\_structure}
template parameterized with the node interface and the channel kind. This actor can only
instantiate sub actors of a predetermined kind and connect their ports with a  predetermined
kind of channel.

The MoC is now determined by the composition of the two concepts node interface
and channel kind as following:

%%\begin{tabular}{|c||c|c|c|}
%%\hline
%% Channel kind & \multicolumn{3}{c|}{ Node interface } \\
%%\hline
%%              & Choice Node         & Transact Node      & Fixed Transact Node \\
%%\hline \hline
%% Fifo         & No well known name  & KPN                & SDF \\
%%\hline
%% Rendezvous   & CSP                 & No well known name & No well known name \\
%%\hline
%%\end{tabular}

\begin{table}
\centering
\begin{tabular}{|l|l|l|l|}
\hline
 Channel kind & Node interface & Network graph type & Constraint set \\
\hline \hline
 Fifo         & Fixed Transact node & Dataflow network graph & SDF constraint set \\
 Fifo         & Transact node       & Dataflow network graph & Dataflow constraint set \\
%% Fifo         & Choice node         & Dataflow network graph & FIFO CSP constraint set \\
 Rendezvous   & Choice node         & Dataflow network graph & CSP constraint set \\
\hline
\end{tabular}
\caption{\label{constraintset-composition}
  Composition of constraint set out of channel kind, node interface
  and network graph type}
\end{table}

\begin{table}
\centering
\begin{tabular}{|l|p{6cm}|l|}
\hline
 Constraint set & Composition in C++ & \SysteMoC{} typedef alias \\
\hline \hline
 SDF constraint set &
  \code{hscd\_graph\_sdf< hscd\_fixed\_transact\_node, hscd\_fifo\_kind>} &
  \code{hscd\_sdf\_constraintset} \\
\hline
 Dataflow constraint set &
  \code{hscd\_graph\_sdf< hscd\_transact\_node, hscd\_fifo\_kind>} &
  \code{hscd\_df\_constraintset} \\
\hline
 CSP constraint set &
  \code{hscd\_graph\_sdf< hscd\_choice\_node, hscd\_rendezvous\_kind>} &
  \code{hscd\_csp\_constraintset} \\
\hline
\end{tabular}
\caption{\label{constraintset-c++}Constraint sets represented as \SysteMoC{}
  classes and their composition in C++}
\end{table}

\section{Examples of MoC in SystemC with SysteMoC}\label{systemoc-examples}

\section{Conclusion}\label{conclusion}



\clearpage
\appendix
\bibliographystyle{alpha}
\bibliography{literature}

\end{document}
