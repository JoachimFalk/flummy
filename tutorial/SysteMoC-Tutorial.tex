\mode<article>
{
}

\mode<presentation>
{
\usetheme{codesign}
\usenavigationsymbolstemplate{} % uncomment to get rid of navigation symbols

}


%\let\Tiny=\tiny
\usepackage[utf8]{inputenc}
\usepackage{pslatex}
\usepackage{graphicx}
\definecolor{lightgray}{gray}{.9}

\usepackage{listings}
\lstset{language=C++,showstringspaces=false,breaklines=true,basicstyle=\ttfamily}
\lstset{backgroundcolor=\color{lightgray}}

\mode<presentation>
{
\lstset{commentstyle=\scriptsize\selectfont}
\lstset{basicstyle=\scriptsize\ttfamily\selectfont}
}
%\lstset{commentstyle=\color{blue}}
%\lstset{stringstyle=\color{green}}
%\lstset{keywordstyle=\color{red}}
%\lstset{emph={bool,int,unsigned,char,true,false,void}}
%\lstset{emphstyle=\color{orange}}
\lstset{emph={[2]\#include,\#define,\#ifdef,\#endif}}
\lstset{emphstyle={[2]\color{blue}}}

\title{SystemCoDesigner: System Programming Tutorial}
\author[SCD Group]{Joachim Falk, Jens Gladigau, Martin Streubühr, Christian Zebelein\\ Christian Haubelt, Jürgen Teich\texorpdfstring{\\}{}
    {\tiny \href{mailto:systemoc-devel@codesign.informatik.uni-erlangen.de}{SysteMoC Mailing list $<$systemoc-devel@codesign.informatik.uni-erlangen.de$>$}}}
\institute[Hardware/Software Co-Design]{Hardware/Software Co-Design\\University of Erlangen-Nuremberg}
\mode<presentation>
{
\university{Friedrich-Alexander University of Erlangen-Nuremberg}
}
\date{08.07.2009}

\AtBeginSection[]
{
   \begin{frame}
       \frametitle{Outline}
       \tableofcontents[currentsection]
   \end{frame}
}

\begin{document}

\begin{frame}
  \titlepage
\end{frame}

\begin{frame}
  \frametitle{Outline}
  \tableofcontents
\end{frame}



\section{Introduction}
%%%%%%%%%%%%%%%%%%%%%%%%%%%%%%%%%%%%%%%%%%%%%%%%%%%%%%%%%%%%%%%%%%%%%%%%%%%%%%
%%%%%%%%%%%%%%%%%%%%%%%%%%%%%%%%%%%%%%%%%%%%%%%%%%%%%%%%%%%%%%%%%%%%%%%%%%%%%%
\begin{frame}[t]
\mode<presentation>{\frametitle{\insertsubsection\ }}
\begin{itemize}
\item \SystemCoDesigner{} uses \SysteMoC{} and \VPC\ for functional and performance simulation
\end{itemize}
\begin{itemize}
\item \SysteMoC{} allows for functional modeling and simulation
\end{itemize}

\begin{itemize}
\item \VPC\ (VPC) allow for performance modeling and simulation
\end{itemize}

\begin{itemize}
\item Together, \SysteMoC{} and \VPC\, allow for combined functional and performance simulation
\end{itemize}

\begin{itemize}
\item Both, \SysteMoC{} and \VPC\, are implemented as individual libraries on top of SystemC
\end{itemize}

\end{frame}


%%%%%%%%%%%%%%%%%%%%%%%%%%%%%%%%%%%%%%%%%%%%%%%%%%%%%%%%%%%%%%%%%%%%%%%%%%%%%%
%%%%%%%%%%%%%%%%%%%%%%%%%%%%%%%%%%%%%%%%%%%%%%%%%%%%%%%%%%%%%%%%%%%%%%%%%%%%%%
\begin{frame}[t]
\mode<presentation>{\frametitle{\insertsection\ -- VPC }}
\begin{itemize}
\item A \VPC\ ...
\item ... is a SystemC module
\item ... models task execution times
\item ... models resource contention, scheduling, and arbitration
\end{itemize}

\begin{itemize}
\item A \VPC\ ...
\item ... is not an Instruction Set Simulator
\item ... is not Software
\item ... is not Hardware
\end{itemize}

\begin{itemize}
\item But, a \VPC\ ...
\item ... allows to model a Processor or a HW component
\item ... allows to model a communication resource
\end{itemize}

\end{frame}



%%%%%%%%%%%%%%%%%%%%%%%%%%%%%%%%%%%%%%%%%%%%%%%%%%%%%%%%%%%%%%%%%%%%%%%%%%%%%%
%%%%%%%%%%%%%%%%%%%%%%%%%%%%%%%%%%%%%%%%%%%%%%%%%%%%%%%%%%%%%%%%%%%%%%%%%%%%%%
\begin{frame}[t]
\mode<presentation>{\frametitle{\insertsection\ -- \SysteMoC\ }}

\begin{figure}
\centering
\resizebox{0.7\columnwidth}{!}{\input{vpc-systemoc-fig.tex}}
\end{figure}

\begin{itemize}
\item A functional model in \SysteMoC\ is given by ...
\item ... a set of actors (e.g., Source and Sink)
\item ... a set of communication queues connecting actors 
\end{itemize}
\end{frame}



%%%%%%%%%%%%%%%%%%%%%%%%%%%%%%%%%%%%%%%%%%%%%%%%%%%%%%%%%%%%%%%%%%%%%%%%%%%%%%
%%%%%%%%%%%%%%%%%%%%%%%%%%%%%%%%%%%%%%%%%%%%%%%%%%%%%%%%%%%%%%%%%%%%%%%%%%%%%%
\begin{frame}[t]
\mode<presentation>{\frametitle{\VPC\ }}

\begin{figure}
\centering
\resizebox{0.7\columnwidth}{!}{\input{vpc-systemoc-fig.tex}}
\end{figure}

\begin{itemize}
\item An architecture model in VPC is given by ...
\item ... a set of components (e.g., CPU, Bus, and Mem )
\item ... a mapping of actors and queues to the set of components
\end{itemize}
\end{frame}










\section{Tutorial}
\subsection{Hello World}
%%%%%%%%%%%%%%%%%%%%%%%%%%%%%%%%%%%%%%%%%%%%%%%%%%%%%%%%%%%%%%%%%%%%%%%%%%%%%%
%%%%%%%%%%%%%%%%%%%%%%%%%%%%%%%%%%%%%%%%%%%%%%%%%%%%%%%%%%%%%%%%%%%%%%%%%%%%%%
\begin{frame}
\mode<presentation>{\frametitle{\insertsubsection\ -- Objectives}}
\begin{itemize}
\item You will see a state-of-the-art ``Hello World'' application ...
\item ...  modeled in SysteMoC.
\end{itemize}
\end{frame}




%%%%%%%%%%%%%%%%%%%%%%%%%%%%%%%%%%%%%%%%%%%%%%%%%%%%%%%%%%%%%%%%%%%%%%%%%%%%%%
%%%%%%%%%%%%%%%%%%%%%%%%%%%%%%%%%%%%%%%%%%%%%%%%%%%%%%%%%%%%%%%%%%%%%%%%%%%%%%
 % fragile is mandratory for verbatim environments (lstlisting)
 % HACK: use singleslide for correct page numbering
 %  (pagecounter+=2 else wise)
 %  this issue seems to be relevent for *pdflatex* only
 % NOTE: the intended usage for *singleslide* is to tell beamer that we use
 %   no overlays (animation)
\begin{frame}[fragile=singleslide]
\mode<presentation>{\frametitle{\insertsubsection}}
\begin{lstlisting}
//file hello.cpp
#include <iostream>
#include <systemoc/smoc_moc.hpp>

class HelloActor: public smoc_actor {
private:
  void src() {   // action
    std::cout << "Actor " << this->name() << " says:\n"
              << "Hello SysteMoC" << std::endl;
  }

  smoc_firing_state start, end;   // FSM states
public:
  // actor constructor
  HelloActor(sc_module_name name)
    : smoc_actor(name, start) {
    // FSM definition:
    //  transition from start to end calling action src
    start = CALL(HelloActor::src) >> end;
  }
};
\end{lstlisting}
\end{frame}

%%%%%%%%%%%%%%%%%%%%%%%%%%%%%%%%%%%%%%%%%%%%%%%%%%%%%%%%%%%%%%%%%%%%%%%%%%%%%%
%%%%%%%%%%%%%%%%%%%%%%%%%%%%%%%%%%%%%%%%%%%%%%%%%%%%%%%%%%%%%%%%%%%%%%%%%%%%%%
\begin{frame}[fragile=singleslide]
\mode<presentation>{\frametitle{\insertsubsection}}
\begin{lstlisting}
//file hello.cpp cont'd

class HelloNetworkGraph: public smoc_graph {
private:
  // actors
  HelloActor     helloActor;
public:
  // networkgraph constructor
  HelloNetworkGraph(sc_module_name name)
    : smoc_graph(name),
      helloActor("HalloActor") // create actor HelloWorld
  { }
};

int sc_main (int argc, char **argv) {
  // create networkgraph
  smoc_top_moc<HelloNetworkGraph> top("top");

  sc_start();  // start simulation (SystemC)
  return 0;
}
\end{lstlisting}
\end{frame}

%%%%%%%%%%%%%%%%%%%%%%%%%%%%%%%%%%%%%%%%%%%%%%%%%%%%%%%%%%%%%%%%%%%%%%%%%%%%%%
%%%%%%%%%%%%%%%%%%%%%%%%%%%%%%%%%%%%%%%%%%%%%%%%%%%%%%%%%%%%%%%%%%%%%%%%%%%%%%
\begin{frame}[fragile=singleslide]
\mode<presentation>{\frametitle{\insertsubsection}}
\begin{itemize}
\item run simulation
\begin{lstlisting}
./hello
\end{lstlisting}
\item simulation output
\begin{lstlisting}
             SystemC 2.2.0 --- Dec 15 2008 11:10:07
        Copyright (c) 1996-2006 by all Contributors
                    ALL RIGHTS RESERVED
Actor top.HalloActor says:
Hello SysteMoC
SystemC: simulation stopped by user.
\end{lstlisting}
\end{itemize}
\end{frame}






\subsection{Actors and Graphs}
%%%%%%%%%%%%%%%%%%%%%%%%%%%%%%%%%%%%%%%%%%%%%%%%%%%%%%%%%%%%%%%%%%%%%%%%%%%%%%
%%%%%%%%%%%%%%%%%%%%%%%%%%%%%%%%%%%%%%%%%%%%%%%%%%%%%%%%%%%%%%%%%%%%%%%%%%%%%%
\begin{frame}
\mode<presentation>{\frametitle{\insertsubsection\ -- Objectives}}
\begin{itemize}
\item You will learn to ...
\item ... write actors.
\item ... instantiate actors in a network graph.
\item ... run functional simulation.
\end{itemize}
\end{frame}




%%%%%%%%%%%%%%%%%%%%%%%%%%%%%%%%%%%%%%%%%%%%%%%%%%%%%%%%%%%%%%%%%%%%%%%%%%%%%%
%%%%%%%%%%%%%%%%%%%%%%%%%%%%%%%%%%%%%%%%%%%%%%%%%%%%%%%%%%%%%%%%%%%%%%%%%%%%%%
\begin{frame}[fragile=singleslide]
\mode<presentation>{\frametitle{\insertsubsection\ -- Actors}}
\index{actor|(}
\index{smoc_actor@\lstinline{smoc_actor}}
\begin{itemize}
\item include SysteMoC library header
\begin{lstlisting}
#include <systemoc/smoc_moc.hpp>
\end{lstlisting}
\item an actor is a C++ class derived from base class \lstinline!smoc_actor!
\begin{lstlisting}
class HelloActor: public smoc_actor {
\end{lstlisting}
\item an action is a member function of the actors class
\item actions shall be \lstinline!private! $\rightarrow$ prohibit execution by others
\begin{lstlisting}
private:
  void src() {   // action
    std::cout << "Actor " << this->name() << " says:\n"
              << "Hello World" << std::endl;
  }
\end{lstlisting}
\end{itemize}
\end{frame}





%%%%%%%%%%%%%%%%%%%%%%%%%%%%%%%%%%%%%%%%%%%%%%%%%%%%%%%%%%%%%%%%%%%%%%%%%%%%%%
%%%%%%%%%%%%%%%%%%%%%%%%%%%%%%%%%%%%%%%%%%%%%%%%%%%%%%%%%%%%%%%%%%%%%%%%%%%%%%
\begin{frame}[fragile=singleslide]
\mode<presentation>{\frametitle{\insertsubsection\ -- Actors}}
\index{state}
\index{smoc_firing_state@\lstinline{smoc_firing_state}}
\index{smoc_actor@\lstinline{smoc_actor}}
\begin{itemize}
\item an actor has a finite set of states
\begin{lstlisting}
  smoc_firing_state state_a, state_b;   // FSM states
\end{lstlisting}
\item like objects of a class in C++, we create instances of an actor
\item constructors are responsible for creating a certain actor instance
\item provide actor name and start state to base class \lstinline!smoc_actor!
\begin{lstlisting}
  HelloActor(sc_module_name name)  // actor constructor
    : smoc_actor(name, state_a) {
\end{lstlisting}
\end{itemize}
\end{frame}




%%%%%%%%%%%%%%%%%%%%%%%%%%%%%%%%%%%%%%%%%%%%%%%%%%%%%%%%%%%%%%%%%%%%%%%%%%%%%%
%%%%%%%%%%%%%%%%%%%%%%%%%%%%%%%%%%%%%%%%%%%%%%%%%%%%%%%%%%%%%%%%%%%%%%%%%%%%%%
\begin{frame}[fragile=singleslide]
\mode<presentation>{\frametitle{\insertsubsection\ -- Actors}}
\index{FSM}
\index{CALL@\lstinline{CALL}}
\begin{itemize}
\item a finite state machine (FSM) defines transitions between states
\item e.g. a transition from \lstinline!state_a! to \lstinline!state_b!
\item if this transition is taken, the action \lstinline!HelloActor::src! is executed
\begin{lstlisting}
  HelloActor(sc_module_name name)  // actor constructor
    : smoc_actor(name, state_a) {

  // FSM definition:
  //  transition from state_a to state_b calling action src
  state_a = CALL(HelloActor::src) >> state_b;

}
\end{lstlisting}
\end{itemize}
\index{actor|)}
\end{frame}





%%%%%%%%%%%%%%%%%%%%%%%%%%%%%%%%%%%%%%%%%%%%%%%%%%%%%%%%%%%%%%%%%%%%%%%%%%%%%%
%%%%%%%%%%%%%%%%%%%%%%%%%%%%%%%%%%%%%%%%%%%%%%%%%%%%%%%%%%%%%%%%%%%%%%%%%%%%%%
\begin{frame}[fragile=singleslide]
\mode<presentation>{\frametitle{\insertsubsection\ -- Graphs}}
\index{network graph}
\index{smoc_graph@\lstinline{smoc_graph}}
\begin{itemize}
\item a network graph is derived from base class \lstinline!smoc_graph!
\begin{lstlisting}
class HelloNetworkGraph: public smoc_graph {
\end{lstlisting}
\item our network graph has an instance of the \lstinline!HelloActor!
\begin{lstlisting}
private:
  // actors
  HelloActor     helloActor;
\end{lstlisting}
\item a name for the graph has to be passed to base class \lstinline!smoc_graph!
\item we need to call \lstinline!HelloActor!s constructor
\begin{lstlisting}
  // network graph constructor
  HelloNetworkGraph(sc_module_name name)
    : smoc_graph(name),
      helloActor("HelloActor") // create actor HelloWorld
  { }
};
\end{lstlisting}
\end{itemize}
\end{frame}


\subsection{Ports and Channels}
%%%%%%%%%%%%%%%%%%%%%%%%%%%%%%%%%%%%%%%%%%%%%%%%%%%%%%%%%%%%%%%%%%%%%%%%%%%%%%
%%%%%%%%%%%%%%%%%%%%%%%%%%%%%%%%%%%%%%%%%%%%%%%%%%%%%%%%%%%%%%%%%%%%%%%%%%%%%%
\begin{frame}
\mode<presentation>{\frametitle{\insertsubsection\ -- Objectives}}
\begin{itemize}
\item You will learn to ...
\item ... use ports.
\item ... connect actors via channels.
\end{itemize}
\end{frame}




%%%%%%%%%%%%%%%%%%%%%%%%%%%%%%%%%%%%%%%%%%%%%%%%%%%%%%%%%%%%%%%%%%%%%%%%%%%%%%
%%%%%%%%%%%%%%%%%%%%%%%%%%%%%%%%%%%%%%%%%%%%%%%%%%%%%%%%%%%%%%%%%%%%%%%%%%%%%%
\begin{frame}[fragile=singleslide]
\mode<presentation>{\frametitle{\insertsubsection\ -- Sink Actor}}
\index{port|(}
\begin{lstlisting}
class Sink: public smoc_actor {
public:
  // ports:
  smoc_port_in<char> in;

  Sink(sc_module_name name)   // actor constructor
    : smoc_actor(name, start) {
    // FSM definition:
    start =
      in(1)                 >>
      CALL(Sink::sink)      >> start;
  }
private:
  smoc_firing_state start;  // FSM states

  void sink() {
    std::cout << this->name() << " recv: \""
              << in[0] << "\"" << std::endl;
  }
};
\end{lstlisting}
\end{frame}






%%%%%%%%%%%%%%%%%%%%%%%%%%%%%%%%%%%%%%%%%%%%%%%%%%%%%%%%%%%%%%%%%%%%%%%%%%%%%%
%%%%%%%%%%%%%%%%%%%%%%%%%%%%%%%%%%%%%%%%%%%%%%%%%%%%%%%%%%%%%%%%%%%%%%%%%%%%%%
\begin{frame}[fragile=singleslide]
\mode<presentation>{\frametitle{\insertsubsection\ -- Input Port}}
%\index{actor!input port|see{input port}}
\index{input port}
\begin{itemize}
\item create an input port
\item ports have a data type (e.g. \lstinline!char!)
\begin{lstlisting}
  smoc_port_in<char> in;
\end{lstlisting}
\item a single object of the particular data type is termed ``a token''
\item declare to read one token in FSM transition
\begin{lstlisting}
    start =
      in(1)                 >>
      CALL(Sink::sink)      >> start;
\end{lstlisting}
\item write data in action
\begin{lstlisting}
  void sink() {
    std::cout << this->name() << " recv: \'"
              << in[0] << "\'" << std::endl;
  }
\end{lstlisting}
\end{itemize}
\end{frame}




%%%%%%%%%%%%%%%%%%%%%%%%%%%%%%%%%%%%%%%%%%%%%%%%%%%%%%%%%%%%%%%%%%%%%%%%%%%%%%
%%%%%%%%%%%%%%%%%%%%%%%%%%%%%%%%%%%%%%%%%%%%%%%%%%%%%%%%%%%%%%%%%%%%%%%%%%%%%%
\begin{frame}[fragile=singleslide]
\mode<presentation>{\frametitle{\insertsubsection\ -- Source Actor}}
\begin{itemize}
\item using output ports is similar
\end{itemize}
\begin{lstlisting}
class Source: public smoc_actor {
public:
  // ports:
  smoc_port_out<char> out;

  Source(sc_module_name name)
    : smoc_actor(name, start) {
    start = 
      out(1)                   >>
      CALL(Source::src)        >> start;
  }
private:
  smoc_firing_state start;  // FSM states

  void src() {
    std::cout << this->name() << " send: \'X\'" << std::endl;
    out[0] = 'X';
  }
};
\end{lstlisting}
\end{frame}






%%%%%%%%%%%%%%%%%%%%%%%%%%%%%%%%%%%%%%%%%%%%%%%%%%%%%%%%%%%%%%%%%%%%%%%%%%%%%%
%%%%%%%%%%%%%%%%%%%%%%%%%%%%%%%%%%%%%%%%%%%%%%%%%%%%%%%%%%%%%%%%%%%%%%%%%%%%%%
\begin{frame}[fragile=singleslide]
\mode<presentation>{\frametitle{\insertsubsection\ -- Output Port}}
%\index{actor!output port|see{output port}}
\index{output port}
\index{port|)}
\begin{itemize}
\item create an output with data type \lstinline!char!
\begin{lstlisting}
smoc_port_out<char> out;
\end{lstlisting}
\item declare to write one token in FSM transition
\begin{lstlisting}
  start = 
    out(1)                   >>
    CALL(Source::src)        >> start;
\end{lstlisting}
\item access data in action
\begin{lstlisting}
void src() {
  std::cout << this->name() << " send: \'X\'" << std::endl;
  out[0] = 'X';
}
\end{lstlisting}
\end{itemize}
\end{frame}




%%%%%%%%%%%%%%%%%%%%%%%%%%%%%%%%%%%%%%%%%%%%%%%%%%%%%%%%%%%%%%%%%%%%%%%%%%%%%%
%%%%%%%%%%%%%%%%%%%%%%%%%%%%%%%%%%%%%%%%%%%%%%%%%%%%%%%%%%%%%%%%%%%%%%%%%%%%%%
\begin{frame}[fragile=singleslide]
\mode<presentation>{\frametitle{\insertsubsection\ -- Connect Actors}}
\index{connectNodePorts@\lstinline{connectNodePorts}|(}
\begin{lstlisting}
class NetworkGraph: public smoc_graph {
public:
  NetworkGraph(sc_module_name name)  // network graph constructor
    : smoc_graph(name),
      source("Source"),             // create actors
      sink("Sink") {
    connectNodePorts(source.out, sink.in); // connect actors
  }
private:
  Source         source;   // actors
  Sink           sink;
};

int sc_main (int argc, char **argv) {
  smoc_top_moc<NetworkGraph> top("top"); // create network graph

  sc_start();   // start simulation (SystemC)
  return 0;
}
\end{lstlisting}
\end{frame}




%%%%%%%%%%%%%%%%%%%%%%%%%%%%%%%%%%%%%%%%%%%%%%%%%%%%%%%%%%%%%%%%%%%%%%%%%%%%%%
%%%%%%%%%%%%%%%%%%%%%%%%%%%%%%%%%%%%%%%%%%%%%%%%%%%%%%%%%%%%%%%%%%%%%%%%%%%%%%
\begin{frame}[fragile=singleslide]
\mode<presentation>{\frametitle{\insertsubsection\ -- FIFO Queues}}
\index{connectNodePorts@\lstinline{connectNodePorts}|)}
\begin{itemize}
\item connect a pair of ports (input, output) using a FIFO queue
\item connected ports have to use the same data type
\item queues have default size ``1'' (one data token)
\begin{lstlisting}
    connectNodePorts(source.out, sink.in);
\end{lstlisting}
\item set queue size explicitly
\begin{lstlisting}
    connectNodePorts<23>(source.out, sink.in);
\end{lstlisting}
\item more advanced channel features (later)
\begin{itemize}
\item channel types
\item initial data tokens
\end{itemize}
\end{itemize}
\end{frame}




%%%%%%%%%%%%%%%%%%%%%%%%%%%%%%%%%%%%%%%%%%%%%%%%%%%%%%%%%%%%%%%%%%%%%%%%%%%%%%
%%%%%%%%%%%%%%%%%%%%%%%%%%%%%%%%%%%%%%%%%%%%%%%%%%%%%%%%%%%%%%%%%%%%%%%%%%%%%%
\begin{frame}[fragile=singleslide]
\mode<presentation>{\frametitle{\insertsubsection}}
\begin{itemize}
\item simulation output
\begin{lstlisting}

             SystemC 2.2.0 --- Dec 15 2008 11:10:07
        Copyright (c) 1996-2006 by all Contributors
                    ALL RIGHTS RESERVED
top.Source send: 'X'
top.Sink recv: 'X'
top.Source send: 'X'
top.Sink recv: 'X'
top.Source send: 'X'
top.Sink recv: 'X'
...
\end{lstlisting}
\item simulation runs infinitely
\end{itemize}
\end{frame}






\subsection{Guards and Actions}
%%%%%%%%%%%%%%%%%%%%%%%%%%%%%%%%%%%%%%%%%%%%%%%%%%%%%%%%%%%%%%%%%%%%%%%%%%%%%%
%%%%%%%%%%%%%%%%%%%%%%%%%%%%%%%%%%%%%%%%%%%%%%%%%%%%%%%%%%%%%%%%%%%%%%%%%%%%%%
\begin{frame}
\mode<presentation>{\frametitle{\insertsubsection\ -- Objectives}}
\begin{itemize}
\item You will learn to ...
\item ... write and use guards.
\item ... write and use actions.
\end{itemize}
\end{frame}




%%%%%%%%%%%%%%%%%%%%%%%%%%%%%%%%%%%%%%%%%%%%%%%%%%%%%%%%%%%%%%%%%%%%%%%%%%%%%%
%%%%%%%%%%%%%%%%%%%%%%%%%%%%%%%%%%%%%%%%%%%%%%%%%%%%%%%%%%%%%%%%%%%%%%%%%%%%%%
\begin{frame}[fragile=singleslide]
\mode<presentation>{\frametitle{\insertsubsection\ -- SourceActor}}
\begin{lstlisting}
static const char MESSAGE [] = "0123456789";

class Source: public smoc_actor {
public:
  smoc_port_out<char> out;
  Source(sc_module_name name) : smoc_actor(name, start),
    count(0), size(strlen(MESSAGE)), message(MESSAGE) {
    start = 
      GUARD(Source::hasToken)  >>
      out(1)                   >>
      CALL(Source::src)        >> start;
  }
private:
  smoc_firing_state start;

  unsigned int count, size;  // variables (functional state)
  const char* message;       //

  bool hasToken() const{ return count<size; } // guard
  void src() { out[0] = message[count++]; }   // action
};

\end{lstlisting}
\end{frame}






%%%%%%%%%%%%%%%%%%%%%%%%%%%%%%%%%%%%%%%%%%%%%%%%%%%%%%%%%%%%%%%%%%%%%%%%%%%%%%
%%%%%%%%%%%%%%%%%%%%%%%%%%%%%%%%%%%%%%%%%%%%%%%%%%%%%%%%%%%%%%%%%%%%%%%%%%%%%%
\begin{frame}[fragile=singleslide]
\mode<presentation>{\frametitle{\insertsubsection\ -- Guards}}
\index{guard}
\index{GUARD@\lstinline{GUARD}}
\begin{itemize}
\item a guard is a \lstinline!const! member function returning a boolean value
\begin{lstlisting}
  bool hasToken() const{
    return count<size;
  } // guard
\end{lstlisting}
\item guards enable/disable transitions (true/false)
\item guards must not change variable values or token in channels
\item refer to guards via \lstinline!GUARD(..)! macro 
\begin{lstlisting}
    start = 
      GUARD(Source::hasToken)  >>
      out(1)                   >>
      CALL(Source::src)        >> start;
\end{lstlisting}
\item use guards for control flow (see below)
\end{itemize}
\end{frame}







%%%%%%%%%%%%%%%%%%%%%%%%%%%%%%%%%%%%%%%%%%%%%%%%%%%%%%%%%%%%%%%%%%%%%%%%%%%%%%
%%%%%%%%%%%%%%%%%%%%%%%%%%%%%%%%%%%%%%%%%%%%%%%%%%%%%%%%%%%%%%%%%%%%%%%%%%%%%%
\begin{frame}[fragile=singleslide]
\mode<presentation>{\frametitle{\insertsubsection\ -- Variables}}
\index{variables}
\begin{itemize}
\item variables ...
\item ... are \lstinline!private! class member of an actor
\item ... can be used to store data
\item ... represent a functional state of an actor (in contrast to FSM state)
\begin{lstlisting}
  unsigned int count, size;
  const char* message;
\end{lstlisting}
\end{itemize}
\end{frame}






%%%%%%%%%%%%%%%%%%%%%%%%%%%%%%%%%%%%%%%%%%%%%%%%%%%%%%%%%%%%%%%%%%%%%%%%%%%%%%
%%%%%%%%%%%%%%%%%%%%%%%%%%%%%%%%%%%%%%%%%%%%%%%%%%%%%%%%%%%%%%%%%%%%%%%%%%%%%%
\begin{frame}[fragile=singleslide]
\mode<presentation>{\frametitle{\insertsubsection\ -- Actions}}
\index{actions}
\begin{itemize}
\item actions ...
\item ... are used to read/write data on input/output ports
\item ... modify variables
\begin{lstlisting}
  void src() {
    out[0] = message[count++];
  }
\end{lstlisting}
\item guards access variables read-only (mandatory \lstinline!const! modifier)
\item actions are allowed to modify variables
\end{itemize}
\end{frame}






%%%%%%%%%%%%%%%%%%%%%%%%%%%%%%%%%%%%%%%%%%%%%%%%%%%%%%%%%%%%%%%%%%%%%%%%%%%%%%
%%%%%%%%%%%%%%%%%%%%%%%%%%%%%%%%%%%%%%%%%%%%%%%%%%%%%%%%%%%%%%%%%%%%%%%%%%%%%%
\begin{frame}[fragile=singleslide]
\mode<presentation>{\frametitle{\insertsubsection\ -- Multiple Access}}
\index{port}
\begin{itemize}
\item you can write/read data more than once (overwrite/re-read)
\item e.g. read input twice
\begin{lstlisting}
  void sink() {
    char squareInput = in[0] * in[0];

    char x = in[0];
    char y = in[0]; // re-read
    assert(x == y);
  }
\end{lstlisting}
\item e.g. write a default value first
\begin{lstlisting}
  void src() {
    out[0] = 'X';    // default
    if(count<size){
      out[0] = message[count++]; //overwrite
    }
  }
\end{lstlisting}
\end{itemize}
\end{frame}










%%%%%%%%%%%%%%%%%%%%%%%%%%%%%%%%%%%%%%%%%%%%%%%%%%%%%%%%%%%%%%%%%%%%%%%%%%%%%%
%%%%%%%%%%%%%%%%%%%%%%%%%%%%%%%%%%%%%%%%%%%%%%%%%%%%%%%%%%%%%%%%%%%%%%%%%%%%%%
\begin{frame}[fragile=singleslide]
\mode<presentation>{\frametitle{\insertsubsection}}
\begin{itemize}
\item simulation output (using Sink actor from previous example)
\begin{lstlisting}
top.Sink recv: "0"
top.Sink recv: "1"
top.Sink recv: "2"
top.Sink recv: "3"
top.Sink recv: "4"
top.Sink recv: "5"
top.Sink recv: "6"
top.Sink recv: "7"
top.Sink recv: "8"
top.Sink recv: "9"
SystemC: simulation stopped by user.
\end{lstlisting}
\item Source actor sends a finite number of characters only
\item simulation terminates when no actor can be activated
\end{itemize}
\end{frame}





%%%%%%%%%%%%%%%%%%%%%%%%%%%%%%%%%%%%%%%%%%%%%%%%%%%%%%%%%%%%%%%%%%%%%%%%%%%%%%
%%%%%%%%%%%%%%%%%%%%%%%%%%%%%%%%%%%%%%%%%%%%%%%%%%%%%%%%%%%%%%%%%%%%%%%%%%%%%%
%\begin{frame}[fragile=singleslide=singleslide]
%\mode<presentation>{\frametitle{\insertsubsection}}
%\begin{itemize}
%\item ...
%\end{itemize}
%\begin{lstlisting}
%
%\end{lstlisting}
%\end{frame}








\section{MJPEG Example}
\section{Programming Guidelines}
\subsection{Models of Computation}
\subsection{How-to}

\begin{frame}
  dummy
\end{frame}

\end{document}

