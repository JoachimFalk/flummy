%%%%%%%%%%%%%%%%%%%%%%%%%%%%%%%%%%%%%%%%%%%%%%%%%%%%%%%%%%%%%%%%%%%%%%%%%%%%%%
%%%%%%%%%%%%%%%%%%%%%%%%%%%%%%%%%%%%%%%%%%%%%%%%%%%%%%%%%%%%%%%%%%%%%%%%%%%%%%
\begin{frame}
\mode<presentation>{\frametitle{\insertsubsection\ -- Objectives}}
\begin{itemize}
\item data flow graphs may contain cyclic dependencies
\item initial token are a common way to break cyclic dependencies
\end{itemize}
\begin{itemize}
\item You will learn to ...
\item ... initialize queus.
\end{itemize}
\end{frame}


%%%%%%%%%%%%%%%%%%%%%%%%%%%%%%%%%%%%%%%%%%%%%%%%%%%%%%%%%%%%%%%%%%%%%%%%%%%%%%
%%%%%%%%%%%%%%%%%%%%%%%%%%%%%%%%%%%%%%%%%%%%%%%%%%%%%%%%%%%%%%%%%%%%%%%%%%%%%%
\begin{frame}
\mode<presentation>{\frametitle{\insertsubsection\ -- Cyclic Dependencies}}
\begin{figure}
\centering
\resizebox{0.9\columnwidth}{!}{\input{init-token-fig.tex}}
\end{figure}
\begin{itemize}
\item each actor forwards input tokens to output port
\item network graph contains a cyclic dependency
\item without initialization none of the actors would become active
\item we may use an initial token to break the cyclic dependency
\end{itemize}
\end{frame}





%%%%%%%%%%%%%%%%%%%%%%%%%%%%%%%%%%%%%%%%%%%%%%%%%%%%%%%%%%%%%%%%%%%%%%%%%%%%%%
%%%%%%%%%%%%%%%%%%%%%%%%%%%%%%%%%%%%%%%%%%%%%%%%%%%%%%%%%%%%%%%%%%%%%%%%%%%%%%
\begin{frame}[fragile=singleslide]
\mode<presentation>{\frametitle{\insertsubsection\ -- Example}}
\begin{lstlisting}
class Forward: public smoc_actor {
public:
  smoc_port_in<int> in;
  smoc_port_out<int> out;

  Forward(sc_module_name name) : smoc_actor(name, start){
    start = 
      in(1)                    >>
      out(1)                   >>
      CALL(Forward::forward)   >> start;
  }
private:
  smoc_firing_state start;

  void forward() {
    std::cout << this->name() << " forward: \""
              << in[0] << "\"" << std::endl;
    out[0] = in[0];
  }
};
\end{lstlisting}
\end{frame}





%%%%%%%%%%%%%%%%%%%%%%%%%%%%%%%%%%%%%%%%%%%%%%%%%%%%%%%%%%%%%%%%%%%%%%%%%%%%%%
%%%%%%%%%%%%%%%%%%%%%%%%%%%%%%%%%%%%%%%%%%%%%%%%%%%%%%%%%%%%%%%%%%%%%%%%%%%%%%
\begin{frame}[fragile=singleslide]
\mode<presentation>{\frametitle{\insertsubsection\ -- Example}}
\begin{lstlisting}
class NetworkGraph: public smoc_graph {
protected:
  // actors
  Forward         ping;
  Forward         pong;
public:
  // network graph constructor
  NetworkGraph(sc_module_name name)
    : smoc_graph(name),
      // create actors
      ping("Ping"),
      pong("Pong")
  {
    smoc_fifo<int> initFifo(1);
    initFifo << 42;
    connectNodePorts(ping.out, pong.in);
    connectNodePorts(pong.out, ping.in, initFifo);
  }
};
\end{lstlisting}
\end{frame}





%%%%%%%%%%%%%%%%%%%%%%%%%%%%%%%%%%%%%%%%%%%%%%%%%%%%%%%%%%%%%%%%%%%%%%%%%%%%%%
%%%%%%%%%%%%%%%%%%%%%%%%%%%%%%%%%%%%%%%%%%%%%%%%%%%%%%%%%%%%%%%%%%%%%%%%%%%%%%
\begin{frame}[fragile=singleslide]
\mode<presentation>{\frametitle{\insertsubsection\ -- Simulation}}
\begin{lstlisting}
             SystemC 2.2.0 --- Dec 15 2008 11:10:07
        Copyright (c) 1996-2006 by all Contributors
                    ALL RIGHTS RESERVED
top.Ping forward: "42"
top.Pong forward: "42"
top.Ping forward: "42"
top.Pong forward: "42"
...
\end{lstlisting}
\end{frame}

%%%%%%%%%%%%%%%%%%%%%%%%%%%%%%%%%%%%%%%%%%%%%%%%%%%%%%%%%%%%%%%%%%%%%%%%%%%%%%
%%%%%%%%%%%%%%%%%%%%%%%%%%%%%%%%%%%%%%%%%%%%%%%%%%%%%%%%%%%%%%%%%%%%%%%%%%%%%%
\begin{frame}[fragile=singleslide]
\mode<presentation>{\frametitle{\insertsubsection\ -- Syntax}}
\begin{itemize}
\item explicit creation of a FIFO initializer TODO: name it QUEUE??
\item give queue size as constructor parameter
\begin{lstlisting}
    smoc_fifo<int> initFifo(1);
\end{lstlisting}
\item push initial value to initializer
\begin{lstlisting}
    initFifo << 42;
\end{lstlisting}
\item pass initializer when constructing the queue
\begin{lstlisting}
    connectNodePorts(pong.out, ping.in, initFifo);
\end{lstlisting}
\item initializer may be reused for creating identical initialized queues
\end{itemize}
\end{frame}


