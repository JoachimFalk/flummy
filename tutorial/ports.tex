%%%%%%%%%%%%%%%%%%%%%%%%%%%%%%%%%%%%%%%%%%%%%%%%%%%%%%%%%%%%%%%%%%%%%%%%%%%%%%
%%%%%%%%%%%%%%%%%%%%%%%%%%%%%%%%%%%%%%%%%%%%%%%%%%%%%%%%%%%%%%%%%%%%%%%%%%%%%%
\begin{frame}
\mode<presentation>{\frametitle{\insertsubsection\ -- Objectives}}
\begin{itemize}
\item You will learn to ...
\item ... use ports.
\item ... connect actors via channels.
\end{itemize}
\end{frame}




%%%%%%%%%%%%%%%%%%%%%%%%%%%%%%%%%%%%%%%%%%%%%%%%%%%%%%%%%%%%%%%%%%%%%%%%%%%%%%
%%%%%%%%%%%%%%%%%%%%%%%%%%%%%%%%%%%%%%%%%%%%%%%%%%%%%%%%%%%%%%%%%%%%%%%%%%%%%%
\begin{frame}[fragile=singleslide]
\mode<presentation>{\frametitle{\insertsubsection\ -- Sink Actor}}
\index{port|(}
\begin{lstlisting}
class Sink: public smoc_actor {
public:
  // ports:
  smoc_port_in<char> in;

  Sink(sc_module_name name)   // actor constructor
    : smoc_actor(name, start) {
    // FSM definition:
    start =
      in(1)                 >>
      CALL(Sink::sink)      >> start;
  }
private:
  smoc_firing_state start;  // FSM states

  void sink() {
    std::cout << this->name() << " recv: \""
              << in[0] << "\"" << std::endl;
  }
};
\end{lstlisting}
\end{frame}






%%%%%%%%%%%%%%%%%%%%%%%%%%%%%%%%%%%%%%%%%%%%%%%%%%%%%%%%%%%%%%%%%%%%%%%%%%%%%%
%%%%%%%%%%%%%%%%%%%%%%%%%%%%%%%%%%%%%%%%%%%%%%%%%%%%%%%%%%%%%%%%%%%%%%%%%%%%%%
\begin{frame}[fragile=singleslide]
\mode<presentation>{\frametitle{\insertsubsection\ -- Input Port}}
%\index{actor!input port|see{input port}}
\index{input port}
\begin{itemize}
\item create an input port
\item ports have a data type (e.g. \lstinline!char!)
\begin{lstlisting}
  smoc_port_in<char> in;
\end{lstlisting}
\item a single object of the particular data type is termed ``a token''
\item declare to read one token in FSM transition
\begin{lstlisting}
    start =
      in(1)                 >>
      CALL(Sink::sink)      >> start;
\end{lstlisting}
\item access data in action
\begin{lstlisting}
  void sink() {
    std::cout << this->name() << " recv: \'"
              << in[0] << "\'" << std::endl;
  }
\end{lstlisting}
\end{itemize}
\end{frame}




%%%%%%%%%%%%%%%%%%%%%%%%%%%%%%%%%%%%%%%%%%%%%%%%%%%%%%%%%%%%%%%%%%%%%%%%%%%%%%
%%%%%%%%%%%%%%%%%%%%%%%%%%%%%%%%%%%%%%%%%%%%%%%%%%%%%%%%%%%%%%%%%%%%%%%%%%%%%%
\begin{frame}[fragile=singleslide]
\mode<presentation>{\frametitle{\insertsubsection\ -- Source Actor}}
\begin{itemize}
\item using output ports is similar
\end{itemize}
\begin{lstlisting}
class Source: public smoc_actor {
public:
  // ports:
  smoc_port_out<char> out;

  Source(sc_module_name name)
    : smoc_actor(name, start) {
    start = 
      out(1)                   >>
      CALL(Source::src)        >> start;
  }
private:
  smoc_firing_state start;  // FSM states

  void src() {
    std::cout << this->name() << " send: \'X\'" << std::endl;
    out[0] = 'X';
  }
};
\end{lstlisting}
\end{frame}






%%%%%%%%%%%%%%%%%%%%%%%%%%%%%%%%%%%%%%%%%%%%%%%%%%%%%%%%%%%%%%%%%%%%%%%%%%%%%%
%%%%%%%%%%%%%%%%%%%%%%%%%%%%%%%%%%%%%%%%%%%%%%%%%%%%%%%%%%%%%%%%%%%%%%%%%%%%%%
\begin{frame}[fragile=singleslide]
\mode<presentation>{\frametitle{\insertsubsection\ -- Output Port}}
%\index{actor!output port|see{output port}}
\index{output port}
\index{port|)}
\begin{itemize}
\item create an output with data type \lstinline!char!
\begin{lstlisting}
smoc_port_out<char> out;
\end{lstlisting}
\item declare to write one token in FSM transition
\begin{lstlisting}
  start = 
    out(1)                   >>
    CALL(Source::src)        >> start;
\end{lstlisting}
\item access data in action
\begin{lstlisting}
void src() {
  std::cout << this->name() << " send: \'X\'" << std::endl;
  out[0] = 'X';
}
\end{lstlisting}
\end{itemize}
\end{frame}




%%%%%%%%%%%%%%%%%%%%%%%%%%%%%%%%%%%%%%%%%%%%%%%%%%%%%%%%%%%%%%%%%%%%%%%%%%%%%%
%%%%%%%%%%%%%%%%%%%%%%%%%%%%%%%%%%%%%%%%%%%%%%%%%%%%%%%%%%%%%%%%%%%%%%%%%%%%%%
\begin{frame}[fragile=singleslide]
\mode<presentation>{\frametitle{\insertsubsection\ -- Connect Actors}}
\index{connectNodePorts@\lstinline{connectNodePorts}|(}
\begin{lstlisting}
class NetworkGraph: public smoc_graph {
public:
  NetworkGraph(sc_module_name name)  // network graph constructor
    : smoc_graph(name),
      source("Source"),             // create actors
      sink("Sink") {
    connectNodePorts(source.out, sink.in); // connect actors
  }
private:
  Source         source;   // actors
  Sink           sink;
};

int sc_main (int argc, char **argv) {
  smoc_top_moc<NetworkGraph> top("top"); // create network graph

  sc_start();   // start simulation (SystemC)
  return 0;
}
\end{lstlisting}
\end{frame}




%%%%%%%%%%%%%%%%%%%%%%%%%%%%%%%%%%%%%%%%%%%%%%%%%%%%%%%%%%%%%%%%%%%%%%%%%%%%%%
%%%%%%%%%%%%%%%%%%%%%%%%%%%%%%%%%%%%%%%%%%%%%%%%%%%%%%%%%%%%%%%%%%%%%%%%%%%%%%
\begin{frame}[fragile=singleslide]
\mode<presentation>{\frametitle{\insertsubsection\ -- FIFO Queues}}
\index{connectNodePorts@\lstinline{connectNodePorts}|)}
\begin{itemize}
\item connect a pair of ports (input, output) using a FIFO queue
\item connected ports have to use the same data type
\item queues have default size ``1'' (one data token)
\begin{lstlisting}
    connectNodePorts(source.out, sink.in);
\end{lstlisting}
\item set queue size explicitly
\begin{lstlisting}
    connectNodePorts<23>(source.out, sink.in);
\end{lstlisting}
\item more advanced channel features (later)
\begin{itemize}
\item channel types
\item initial data tokens
\end{itemize}
\end{itemize}
\end{frame}




%%%%%%%%%%%%%%%%%%%%%%%%%%%%%%%%%%%%%%%%%%%%%%%%%%%%%%%%%%%%%%%%%%%%%%%%%%%%%%
%%%%%%%%%%%%%%%%%%%%%%%%%%%%%%%%%%%%%%%%%%%%%%%%%%%%%%%%%%%%%%%%%%%%%%%%%%%%%%
\begin{frame}[fragile=singleslide]
\mode<presentation>{\frametitle{\insertsubsection}}
\begin{itemize}
\item simulation output
\begin{lstlisting}

             SystemC 2.2.0 --- Dec 15 2008 11:10:07
        Copyright (c) 1996-2006 by all Contributors
                    ALL RIGHTS RESERVED
top.Source send: 'X'
top.Sink recv: 'X'
top.Source send: 'X'
top.Sink recv: 'X'
top.Source send: 'X'
top.Sink recv: 'X'
...
\end{lstlisting}
\item simulation runs infinitely
\end{itemize}
\end{frame}





