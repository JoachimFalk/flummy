\section{Experimental Results}\label{sec:systemoc-casestudy}

In this section we will compare our simulation framework with other approaches to represent executable specifications.
As a case study, we use a two dimensional \emph{inverse discrete cosinus transform} (IDCT) for 8x8 blocks, as displayed in Figure~\ref{fig:ng-idct2d}.
The execution times for the IDCT in different simulation environments were measured on an Intel Pentium~IV CPU with 2.80 GHz and 512 MB of main memory via the linux \code{time} command.
In Table~\ref{tbl:systemoc-comparison} the user CPU time needed by the simulator for executing $10000$ IDCT iterations is shown.

\begin{table}[h]
\caption{\label{tbl:systemoc-comparison}%
Table comparing the simulation performance for 10000 8x8 blocks of the two dimensional \emph{inverse discrete cosinus transform} (IDCT) implemented in different execution environments.
%Visual representation of the \emph{firing FSM} of the \code{SqrLoop} actor $a_2$ shown in Figure~\ref{fig:ng-sqrroot}.
%The \code{SqrLoop} actor controls the number of iterations performed by Newton's square root algorithm.
}
\centering
%%\hrule
%%\rule{0mm}{2mm}\\
\begin{tabular}{|r|r|r|r|r|}\hline
      & C++               & \SysteMoC          & SystemC            & Ptolemy/CAL      \\\hline\hline
 Time & $9.5 \mathrm{ms}$ & $2.06\mathrm{sec}$ & $2.65\mathrm{sec}$ & $33\mathrm{sec}$ \\\hline
\end{tabular}
%%\rule{0mm}{2mm}\\
%%\hrule
\end{table}

\begin{figure}[h]
\centering
\resizebox{\columnwidth}{!}{\input{ng-idct-1d-2d-fig.tex}}
%\includegraphics[width = 4in]{ng-sqrroot}
%\input{ng-sqrroot-fig.tex}
\caption{\label{fig:ng-idct2d}%
Network graph of the two dimensional \emph{inverse discrete cosinus transform} (IDCT) for 8x8 blocks, e.g., as used in the JPEG and MPEG algorithm.
The two dimensional IDCT is composed out of two one dimensional IDCTs, Which are themselves composed out of primitive arithmetic operations like, e.g., \code{Scale} or \code{AddSub}.
% Netzwerkgraph einer zweidimensionalen \emph{IDCT} f�r 8x8 Makrobl�cke.
% Diese IDCT besteht aus den \emph{Aktorinstanzen} $A_1$ - $A_7$
% verbund mittels der Kan�le $c_{1}$ - $c_{34}$.
% Funktional besteht die zweidimensionale IDCT aus
% zwei eindimensionalen IDCT, den Aktoren $A_{3}$ und $A_{5}$,
% welche sequentiell alle Zeilen $A_3$ bzw. alle Spalten $A_5$ eines Makroblocks abarbeiten.
% Eingebette sind diese beiden eindimensionalen IDCT in den
% Aktor $A_{2}$ zur Aufspaltung eines Makroblocks in seine Zeilen, einem
% Aktor $A_{4}$ zum transponieren der Zeilen und Spalten und einem
% Aktor $A_{6}$ zur erneuten Zusammenfassung der Spalten in einen Makroblock.
% Stimuliert wird das System mittels der Quell- bzw. Senkaktoren
% $A_{1}$ bzw. $A_{7}$.
}
\end{figure}

It can be seen that our approach clearly outperforms interpretative approach of Ptolemy using the \emph{CAL actor language}.
However, the SystemC design is of two orders of magnitude slower a pure C++ implementation.
On the other hand, due to the reduced number of events our \SysteMoC{} than a implementation based on SystemC threads and FIFOs.
Based on our unique approach to model the communication behavior, we will integrate static scheduling analysis into the \SysteMoC{} elaboration phase.%, instead of only using this analysis techniques on the extracted XML representation of the design.
That way, we expect to provide a simulation environment with a simulation performance similar to SystemC TLM designs.
